\section{Intercepting \JS Functionality}
\label{cost-benefit:intercepting-js}

This Section presents a technique for interposing on, and optionally preventing
access to, \WAPI features.  The technique is novel in the way it attempts to
minimize the effect on existing code that expects the now-removed feature to be
present.  This approach is used both in building cost-benefit measurements
described in this chapter, and in the implementation of the browser-hardening
extension described in Chapter~\ref{current-web}.


\subsection{Removing Features from the DOM}
\label{cost-benefit:intercepting-js:featremove}
Each webpage and iframe gets its own global \texttt{window} object.  Changes
made to this global object are shared across all scripts on the same page, but
not between pages.  Furthermore, changes made to this global object are seen
immediately by all other script running in the page.  If one script, for
example, deletes or overwrites the \texttt{window.alert} function, no other
scripts on the page will be able to use the \texttt{alert} function, and there
is no way they can recover it.

As a result, earlier code can arbitrarily modify the execution environment seen
by later code.  Browsers allow extensions to inject \JS code into pages and
frames before any page controlled code
runs\footnote{Section~\ref{current-web:extension-deployment} discusses in much
greater detail how and when extensions can modify page execution environments.
This brief discussion is only intended to be enough to support the discussion
of how \WAPI features can be interposed on.} Since this extensions can run
before any scripts included by the page, extensions can modify the browser
environment for all code executed in any page.  The challenge in removing a
feature from the browser environment is not to \emph{just} prevent pages from
reaching the feature, but to do so \emph{in way that still allows the rest of
the code on the page to execute without introducing errors}.

For example, to disable the \texttt{getElementsByTagName} feature, one could
simply remove the \texttt{getElementsByTagName} method from the
\texttt{window.document} object. However, this will result in exceptions to
be thrown if future code attempts to call the now-removed method.

\lstset{numbers=left,xleftmargin=2em,frame=single,framexleftmargin=1.5em}
\begin{figure}[h]
\begin{lstlisting}[language=javascript]
var ps, p5;
ps = document.getElementsByTagName("p");
p5 = ps[4];
p5.setAttribute("style", "color: red");
alert("Success!");
\end{lstlisting}
\caption{Trivial \JS code example, changing the color of the text in a paragraph.}
\label{fig:trivial-js}
\end{figure}


Consider the code in \ref{fig:trivial-js}:  removing the
\texttt{window.document.getElementsByTagName} method will cause an error
on line one, as the code is attempting to call the now-missing property as if
were a function.  Replacing \texttt{getElementsByTagName} with a new, empty
function solves the problem on line one, but only pushes the error to line
two, unless the function returned an array of at least length five. Even after
accounting for that result, one would need to expect that the
\texttt{setAttribute} method was defined on the fourth element in that array.
One could further imagine that other code on the page may be predicated on
other properties of that return value, and fail when those expectations are not
met.


\subsection{ES6 Proxy Configuration}
Our technique solves this problem through a novel use of a new capability
introduced in ES6, the \texttt{Proxy} object.  The \texttt{Proxy} object can
intercept operations and optionally pass them along to another object.
Relevant to this work, proxy objects also allow code to trap on general
language-operations.  Proxies can register functions that fire when the
proxy is called like a function, indexed into like an array, has its properties
accessed like an object, and operated on in other ways.

We take advantage of the \texttt{Proxy} object's versatility in two ways.
First, we use it to prevent websites from accessing browser features,
without breaking existing code.  This use case is described in detail in
Subsection~\ref{cost-benefit:intercepting-js:proxy-general}.  Second, we
use the \texttt{Proxy} object to enforce policies on runtime created objects.
This use case is described in further detail in
Subsection~\ref{cost-benefit:intercepting-js:proxy-non-singletons}.


\subsection{Proxy-Based Approach}
\label{cost-benefit:intercepting-js:proxy-general}

Our technique uses the \texttt{Proxy} object to solve the general problem
demonstrated in Section~\ref{cost-benefit:intercepting-js:featremove}.  We
create a specially configured a proxy object that registers callback functions
for \emph{all} possible \JS operations, and have those callback functions
return a reference to the same proxy object.  We also handle cases where \WAPI
properties and functions return scalar values (instead of functions, arrays
or higher order objects), by having the proxy to coerce to \texttt{0},
empty string, or \texttt{undefined}, depending on the context. Thus configured,
the proxy object can validly take on the semantics of any variable in any \JS
program.

Returning to the example in \ref{fig:trivial-js}, replacing
\texttt{getElementsByTagName} with our proxy will execute cleanly and the alert
dialog on line four will successfully appear.  On line one, the proxy object's
function handler will execute, resulting in the proxy being stored in the
\texttt{ps} variable.  On line two, the proxy's \texttt{get} handler will
execute, which also returns the proxy, resulting in the proxy again being
stored in \texttt{p5}.  Calling the \texttt{setAttribute} method causes the
proxy object to be called twice, first because of looking up the
\texttt{setAttribute}, and then because of the result of that look up being
called as a function.  The end result is that the code executes correctly, but
without accessing the original \texttt{getElementsByTagName} functionality.

The complete proxy-based approach to graceful degradation can be found in
the source code of our browser extension\footnote{\ExtensionSourceUrl}, which
is discussed in detail in Chapter~\ref{current-web}.

Most state changing features in the browser are implemented through methods
which we interpose on using the above described method.  However, this approach
does not work for the small number of features implemented through property sets.
For example, assigning a string to \texttt{document.location} redirects
the browser to the \gls{url} represented by the string.  When the property is
being set on a singleton object in the browser, as is the case with the
\texttt{document} object, we interpose on property sets by assigning a new
``set'' function for the property on the singleton using
\texttt{Object.defineProperty}.


\subsection{Sets on Non-Singleton Objects}
\label{cost-benefit:intercepting-js:proxy-non-singletons}
A different approach is needed for property sets on non-singleton objects.
Property sets on \WAPI defined objects can be imposed on by using \texttt{Object.defineProperty}
to overwrite the ``get'' and ``set'' attributes of the property.  However,
this approach does not allow for capturing the value being set in the ``set''
case.  Therefor, our approach only allows for blocking sets on non-singleton,
\WAPI defined objects.  It can't be used to make more general, runtime policy
decisions, where decision logic would need to be made at execution time on
whether to block or allow the ``set'' operation.

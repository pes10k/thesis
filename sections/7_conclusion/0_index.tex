\chapter{Conclusion}
\label{conclusion}

The modern web is the result of a long series of largely uncoordinated
iterative changes, driven by a combination of lineage\footnote{e.g. the
adoption of Hypercard's event model in the early \gls{dom} standards.}, market
competition\footnote{e.g. the approximately 10-day design and implementation
cycle allowed for the development of \JS}, well-intentioned but incorrect
efforts to predict where the web would go\footnote{e.g. the entire \gls{svg}
\WAPI standard, intended to compete with Flash's vector graphics system, but
which received little traction}, the inability or refusal of standards
committees to work together for the benefit of system cohesion\footnote{e.g.
the choice of the \gls{css} committee to use kebab-case for style properties,
v.s.  the \gls{dom} committee's choice to use camel-case names.}, and efforts
to push the browser as a near-replacement for the operating
system\footnote{e.g. the large number of \WAPI standards for interacting with
hardware, largely adopted because of the ``Firefox OS'' and ``Chrome OS''
projects}, among many other well intentioned reasons.

Given this chaotic, unplanned, and organic development history, its worth
celebrating how successful and useful the web has been.  The web is almost
certainly the worlds largest, open application platform, and despite the myriad
of security and privacy issues the web has suffered over its history, its worth
noting that in many ways the web platform has proved more secure than its
competitors.

However, to say the web has been a huge success (for application users and
developers alike) is not to say the system cannot be improved.  As this
dissertation has hopefully demonstrated, web users face a large number of
unnecessary privacy and security risks.

This dissertation aimed to improve the state of web privacy and security
by applying a cost-benefit analysis to one important part of the browser,
the \WAPI, and seeing how those findings can be used to make the platform
safer for users.  Chapter~\ref{measurement} presented a technique for measuring
what parts of the \WAPI are actually being used on the web, along with the
results of applying that technique to the \ATK.  Chapter~\ref{cost-benefit}
built on these measurements to systematically measure the costs and benefits
posed by each of the standards in the \WAPI, to distinguish
highly beneficial functionality from functionality that placed web users
at unnecessary, uncompensated risk.

Chapter~\ref{current-web} considered how these cost-benefit measurements could
be used to improve privacy and security on the web as it exists today, and
demonstrated the possibility for improvement through the development of a
publicly available browser extension that enforces access-controls on
the \WAPI.

Finally, Chapter~\ref{future-web} considered how these cost-benefit
measurements could be used in the development of alternative web application
system, in order to further protect web users, and demonstrated the feasibility
of this approach through \CDF, a system for developing and deploying safer web
applications, using commodity web browsers.

The author hopes that collectively the findings and data presented in this
dissertation can play a small part in guiding standards committees, future
researchers, and browser vendors in the development of a safer, more privacy
preserving web.

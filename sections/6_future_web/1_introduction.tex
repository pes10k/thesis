\section{Introduction}
\label{future-web:introduction}

The final step this dissertation makes in improving web privacy and security is
to explore other ways web-like applications could be developed and deployed, in
light of the findings discussed in earlier chapters.  Chapter~\ref{current-web}
presented ways of improving web security and privacy while maintaining
compatibility with existing websites. This chapter considers the further
improvements that could be achieved with a system designed from the start with
the findings from Chapters~\ref{measurement} and \ref{cost-benefit}.

This chapter presents \CDF, an alternative method for describing interactive
websites. The design requires site authors to describe sites using a
declarative, statically checkable format, that trades a loss in
author-expressiveness for gains in client-enforceable security guarantees.  The
design is as a proof of concept, to demonstrate that many kinds of websites
users enjoy on the modern web can be implemented with only a subset of the
functionality in the browser, and in a manner that allows the client to enforce
a greater number of protections and guarantees.

The rest of this chapter is organized as follows:
Section~\ref{future-web:design} presents the high level design of the system.
Section~\ref{future-web:implementation} describes an implementation of \CDF~'s
design that leverages the engineering and omnipresence of commodity web
browsers.  Section~\ref{future-web:evaluation} gives an evaluation of the
system's capabilities and security guarantees, a discussion of the limitations
of the current design, and the types of applications the system could not
implement, in its current state.  Section~\ref{future-web:conclusions}
discusses this work's place in the overarching goals of this dissertation.

\section{Organization}
\label{intro:organization}

The remainder of this dissertation is organized in the following manner.

Chapter \ref{background} provided background material on attacks and defenses
related to browser security.

Chapter \ref{measurement} describes both an automated method for measuring
browser feature use on the web, and the results of applying that technique
to the Alexa 10 (as it was at the time of measurement).  This chapter
also includes the description of a method for distinguishing user-serving
feature use from non-user-serving (e.g. advertising and tracking related)
feature use.

Chapter \ref{cost-benefit} presents a method for measuring the costs and
benefits of the \WAPI standards implemented in modern web browsers, and
the results of applying that methodology to a modern, representative web
browser.

Chapter \ref{current-web} describes efforts to apply the findings from
chapters \ref{measurement} and \ref{cost-benefit} to the web as it exists today,
both in the form of a publicly-released browser extension that is being
used by around 1,000 real-world users, and as patches to an existing, privacy
oriented browser.  This chapter also includes a usability measurement of the
extension-approach, comparisons with other popular wbe security and privacy
related tools, and security vulnerabilities that were discovered in popular
browser as a result of this work.

Chapter \ref{future-web} presents the design of an alternative system for developing
and deploying web applications that, building on the findings presented in
chapters \ref{measurement} and \ref{cost-benefit}, is intended to provide
site authors with the expressiveness needed to design, interactive modern web
sites, but while providing client-enforced security and privacy guarantees to
web users.

Chapter \ref{conclusion} concludes with some high level findings from the
work, and possible areas for further research.

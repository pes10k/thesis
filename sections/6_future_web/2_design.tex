\section{Design}
\label{future-web:design}

\CDF is an alternative method of creating modern, interactive websites, but with greater security and
privacy guarantees than the current HTML-and-\JS system provides.  The principal
features in the design of \CDF are as follows.

\textbf{First}, \CDF prevents websites from running arbitrary code in the client browser.
Instead, \CDF authors create interactive websites by
composing trusted, client-controlled implementations of interactive web functionality
using an easily checked, declarative syntax.
\textbf{Second}, \CDF only uses a subset of browser features, allowing websites
to access only the ``core'' or most popular and frequently used tools for
creating interactive web sites.
\textbf{Third}, \CDF places stricter constraints on web documents than current
HTML-and-\JS applications enforce,
to better protect user privacy and security.

\begin{table}[ht]
    \begin{tabular}{ l c c }
    \toprule
        Capability                      &   HTML + JS    & CDF \\
    \midrule
    Load static media from remote
        and local domains             &   \checkmark    & \checkmark  \\
    Load non-client controlled \JS  &   \checkmark    & -           \\
    Can express common web
        design idioms                 &   \checkmark    & \checkmark  \\
    Server control over
        HTTP \texttt{referer} and
        related privacy settings      &   \checkmark    & -           \\
    Client guarantees over
        HTTP \texttt{referer} and
        related privacy settings      &   -             & \checkmark  \\
    Read sensitive values from
        cookies, local storage, etc.  &   \checkmark    & -           \\
    Sub-page / AJAX requests and
        updates                       &   \checkmark    & \checkmark  \\
    Allow form submissions and
        AJAX updates to remote
        domains                       &   \checkmark    & -           \\
    HTML5 multimedia
        (\texttt{<audio>},
        \texttt{<video>})             &   \checkmark    & \checkmark  \\
    Supports common browser plugins
        (Flash, Java, Silverlight)  &   \checkmark    & -           \\
    Advanced \JS tools
        (WebGL, \texttt{<canvas>},
        ASM.js)                       &   \checkmark    & -           \\
    Client side storage
        (IndexDB, localStorage, etc)  &   \checkmark    & -           \\
    Offline Applications            &   \checkmark    & \checkmark  \\
    \bottomrule
    \end{tabular}
    \caption{Feature comparison between HTML and CDF document formats.}
    \label{table:cdf-html-comparison}
  \end{table}


\ref{table:cdf-html-comparison} provides a comparison of the capabilities
and guarantees made by current HTML-and-\JS based applications, contrasted
against \CDF documents.  The following subsections provide more detail
about each aspect of the design of \CDF.


\subsection{Trusted Feature Implementation}
\label{future-web:design:trusted-feature-implementation}
\CDF's main method for improving user security and privacy is by preventing websites from
executing arbitrary \JS in the browser.  The current
\JS-based system for providing interactive websites is the cause
of many web browser security problems.  Web browsers must trust that code
will carry out some non-malicious purpose when executing it, and that a given
set of \JS instructions will benefit the user (by, for example, setting up
a website's user interface elements) instead of harming the user (e.g. by fingerprinting
the user, accessing a browser feature with a known security flaw, or sending
a session token to a remote server).

Instead of the difficult to secure \JS approach, \CDF provides a set of
trusted, client-side implemented interactive primitives, and allows websites
to compose them using a declarative, easy to verify syntax.  \CDF authors can,
for example, tie a \emph{mouse click} event to a \emph{document attribute
change} event, not by writing code directly, but through the structure of the
document.  \CDF clients include their own trusted libraries that handle
generating code and executing the relevant functionality on the clients, without
trusting code provided by the website.

The result is that \CDF documents
are composed from functionality implemented in trusted, client-controlled libraries.
These libraries are designed to compose safely, and pages can only access them through
a simple, declarative syntax.  This is in contrast to the typical \JS based approach, where websites
can execute arbitrary code, and web browsers must judge if the resulting
behavior seems safe through heuristics like XSS filters and code origin
reputation systems.


\subsection{\CDF Feature Selection}
\label{future-web:design:feature-selection}
\CDF also protects user security and privacy by reducing the browser's attack
surface by preventing websites from accessing browser functionality that
is either rarely used, or predominantly used for advertising and tracking.

Modern web browsers implement a huge array of features, which site authors
access through \JS APIs (either the DOM or the Web API).  Some features
are closely related to the web's core purpose of interactive
documents.  Examples of such features include the \textit{DOM Level 2: Core}
standard~\cite{dom2corew3c}, used to manipulate and inspect an HTML document
with \JS, or the \textit{XMLHttpRequest} standard~\cite{ajaxwhatwg}, which allows
websites to make sub-document updates to the origin domain without refreshing
the page.

Other browser features are more esoteric and only loosely related to the goal
of providing interactive documents.  For example, modern web browsers
implement the \textit{Web Audio} standard~\cite{webaudio2018standard}, which
allows websites to perform full audio synthesis, the \emph{Ambient Light}
standard~\cite{ambientlightapi}, which allow websites to access any light
sensors on the device, and the \emph{WebRTC} standard~\cite{webrtcw3c}, which
allows web browsers to create peer-to-peer networks.

Research done by Snyder et al~\cite{snyder2016browser} found that a small
number of browser features, primary related to document-manipulation and updating,
were frequently used on the web, while the majority of \JS-accessed
features were never used by websites.  The same work found that even more
browser features became rarely used when filtering out advertising and tracking
related usage.

\CDF improves user security and privacy by only allowing websites access
to the frequently used, document-manipulation related features in the browser.
By preventing websites from accessing features that do not generally provide
user serving interests (either because those features are primary used for
advertising and tracking, or because the features are rarely used at all),
\CDF brings web browsers into closer alignment with the security principal of
least privilege.  The attack surface exposed to websites is
dramatically reduced, with minimal impact to the user experience.


\subsection{Document Constraints}
\label{sec:design:doc-constraints}
HTML-based applications include several other design aspects that make them
difficult to secure.  HTML and \JS based applications allow scripts to be
loaded from remote locations from any part of the HTML document, enabling
many XSS attacks.  HTML documents can contain full sub-documents
through the use of \texttt{<iframe>} elements, enabling drive by downloads and related
attacks.  And HTML applications generally include a ``referer'' header
when requesting remote resources, enabling some forms of user tracking.

\CDF improves user security and privacy by tightly-controlling what kinds of
resources documents can fetch, and what information is sent during the fetch
request.  CDF documents cannot include arbitrary code (either
inline or hosted remotely), include sub-documents, or send information
generated in the client directly to remote domains.

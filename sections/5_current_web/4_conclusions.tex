\section{Conclusions}
\label{current-web:conclusions}

This Chapter builds on the findings described in Chapters~\ref{measurement} and
\ref{cost-benefit} to build a publicly available tool that allows web users to
restrict what parts of the \WAPI websites can access.  This Chapter also
presents evaluations of the usability of the tool under common usage scenarios,
some quantification of the security benefits of using the tool under those
scenarios, and some insights that were only gained when the tool began being
used by real-world, unaffiliated users.

The most significant results of this study, as it relates to this
dissertation's overarching goal of improving privacy and security on the web,
are three insights.  First, that restricting website access to the \WAPI is
a realistic and implementable way of protecting the security and privacy of
users on the web today, and with trade-offs that real world users are willing
to accept.  Second, that with relatively minor changes, browser vendors
could allow the security and privacy benefits of this \WAPI blocking technique
to be enjoyed at even lower cost.  And third, that standards authors should
consider whether all websites need access to new \WAPI functionality, or if
users would be better served by a restrict-by-default, opt-in-when-needed model.

This Chapter has focused on applying the findings from
Chapters~\ref{measurement} and \ref{cost-benefit} to the web as it is currently
designed.  Chapter~\ref{future-web} explores what security and privacy
benefits can be achieved by applying these findings to a deeper redesign of
how web applications are designed and deployed.

\section{Browser Feature Growth}

% However, the earliest interactions
% for the \gls{www} specified only static documents and form submission systems,
% with interactivity limited to applets and plugins.

% It wasn't until Netscape introduced \JS to the web in 1996 in Netscape 2.0
% that the web as a modern application platform began to develop. Since then,
% the web has picked up a large, and quickly growing, set of capabilities,
% designed to allow web site authors to deliver new and more complex applications
% to users.

% This feature growth was initially driven by competition between
% browser vendors.  Netscape and Microsoft attempted to gain market advantage
% by implementing unique functionality, to entice web authors
% to target their browsers, and ensure a larger user base for the browser.
% More recently, this feature growth is driven primarily by competition between the
% web and ``native'' application platforms, or systems provided by operating
% systems, that may provide better performance, or additional functionality, then
% web sites can provide.  Each of these efforts to expand the functionality
% of the browser has been accompanied by a different marketing slogan (e.x.
% DHTML, Web 2.0, HTML5, the browser as OS).

% The end result is that the web has changed from a system of static document
% retrieval, to a platform for delivering applications that are every bit
% as sophisticated as applications delivered and deployed through more
% conventional means.  Where websites could once be thought of as simple
% collections of images, text and styling instructions (with possible
% interactive elements at the periphery), websites now have access to a
% range of functionality that rivals ``native'' applications.  Websites can
% access graphics cards, create peer-to-peer networks, access IO devices
% (e.x. the file system, microphone, ambient light sensor, video camera, VR
% devices) and perform high resolution timing operations, among countless other
% capabilities.  Browser vendors, working with standards bodies like the \gls{w3c},
% aggressively add to the list.
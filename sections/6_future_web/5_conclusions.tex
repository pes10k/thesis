\section{Conclusions}
\label{future-web:conclusions}
This chapter presents an alternate system to designing and deploying web
applications that emphasizes user privacy and security.  The system, titled
\CDF, achieve these gains in two primary ways: first by restricting the set of
\WAPI features websites have access to, and second, but using a statically
verifiable, typed document system, instead of the \gls{html} and arbitrary-\JS
model used on the web today.

This system builds on several findings discussed earlier in this dissertation:
that most of the \WAPI is not used by websites on the internet
(Chapter~\ref{measurement}), that allowing websites to access this
rarely-needed functionality imposes unnecessary risks to users privacy and
security  (Chapter~\ref{cost-benefit}), and that most of the types of
functionality users enjoy online can be provided by limiting websites to a
subset of.  ``core'', low risk \WAPI features (Chapters~\ref{cost-benefit} and
\ref{current-web}).

While not intended to be used by web developers and users as is, the \CDF
system presented in this chapter demonstrates that most of the benefits of
modern web applications can be implemented and enjoyed at much lower risk to
web users than the current \gls{html} and \JS system entails.

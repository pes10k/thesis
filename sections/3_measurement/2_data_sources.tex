\section{Data Sources}
\label{measurement:data-sources}

This work relied on several prior-existing sets of data.  This Section
describes these existing data sets, and how we used them to measure what
\JS features are used on the modern web.


\subsection{Alexa Website Rankings}
\label{measurement:data-sources:website-popularity-rankings}
The Alexa rankings are a well known ordering of websites by traffic.
Typically, researchers uses Alexa's rankings of the worldwide million most
popular sites. However, Alexa also provides other data about these sites,
including popularity rankings at country granularity, breakdowns of which
sub-sites (by fully qualified domain name) are most popular, and a breakdown by
page load and by unique visitor of how many monthly visitors each site gets.

We used the 10,000 top ranked sites from Alexa's list of the one-million
most popular sites as representative of the web in general.  This set of 10,000
websites represent approximately one third of all web visits.


\subsection{Web API Features}
\label{measurement:data-sources:method-web-standards}
As mentioned in Section~\ref{intro:terms}, this work uses the term
\textbf{feature} to denote a browser capability that is accessible through
calling a \JS function or setting a property on a \JS object.

We determined the set of \JS-exposed features by reviewing the \gls{webidl}
definitions included in the \FF version \FFversion source code. \gls{webidl} is
a language that define \JS interfaces implemented in a browsers.
These \gls{webidl} files are included in the \FF source.

In the common case, Firefox's \gls{webidl} files define a mapping between a
\JS accessible method or property and the C++ code that implements
the underlying functionality\footnote{In addition to mapping
\JS to C++ methods and structures, \gls{webidl} can also define \JS
to \JS methods, as well as intermediate structures that are not
exposed to the browser.  In practice though, the primary role of \gls{webidl}
is to define a mapping between \JS API endpoints and
the underlying implementations, generally in C++.}. We examined each of the 757
\gls{webidl} files in the \FF and extracted \numfeatures relevant methods and
properties implemented in the browser.


\subsection{Web API Standards}
Web standards are documents defining functionality that web browser vendors
should implement.  They are generally written and formalized by organizations
like the W3C, though occasionally standards organizations delegate
responsibility for writing standards to third parties, such as the Khronos
group who maintains the current \textit{WebGL} standard.  As mentioned in
Section~\ref{intro:terms}, we use the term \textbf{standard} to refer to these
sets of features, generally grouped by a common purpose.

There are also web standards that cover non-\JS aspects of the browser (such
as parsing rules, what tags and attributes can be used in HTML documents,
etc.). This work focuses only on web standards that define \JS exposed
functionality.

We identified \numstandards standards implemented in \FF.  We associated each
of these to a standards document.  We also found 65 API endpoints implemented
in \FF that are not found in any web standard document, which we associated
with a catch-all \textit{Non-Standard} categorization.

In the case of extremely large standards, we identify sub-standards, which
define a subset of related features intended to be used together.  For example,
we treat the subsections of the \textit{HTML}~\cite{whatwg2018html} standard
that define the basic \textit{Canvas API}, or the \textit{WebSockets API}, as
their own standards.

We treated these signifigant sub-standards as seperate units of analysis
becuase they have their own coherent purpose in the \WAPI, independent
of their parent, containing standard; many of these signifigant subs-standards
were implemented in browsers independent of the parent standard (i.e. browser
vendors added support for ``websockets'' long before they implemented the full
``HTML5'' standard).

Some features appear in multiple web standards.  For example, the
\texttt{Node.prototype.insertBefore} feature appears in the \textit{Document
Object Model (DOM) Level 1 Specification}~\cite{dom1w3c}, \textit{Document
Object Model (DOM) Level 2 Core Specification}~\cite{dom2corew3c} and
\emph{Document Object Model (DOM) Level 3 Core
Specification}~\cite{dom3corew3c} standards.  In such cases, the feature is
attributed to the earliest containing standard.


\subsection{Historical Firefox Builds}
We determined when features were implemented in \FF by
examining the \numfirefoxes versions of \FF that were released between 2004
and when this work was conducted in 2016, and finding the earliest version
that each of the \numfeatures feature appeared.
We treat the release date of the earliest version of
\FF that a feature appears in as the feature's ``implementation date''.

Most standards do not have a single implementation date, since in some case
it took months or years for all of the features in a standard to be
implemented in \FF.  We therefore treated the introduction of a standard's
most popular feature as the standard's implementation date. For ties
(especially relevant when no feature in a standard is used), we
used the date of the earliest implemented feature.


\subsection{Blocking Extensions}
\label{measurement:data-sources:data-extensions}
Finally, this work drew from commercial and crowd-sourced browser extensions,
which are popularly used to modify the browser environment.

This work used two such browser extensions, Ghostery and AdBlock Plus.
Ghostery is a browser extension that allows users to increase their privacy
online by modifying the browser to not load resources or set cookies associated
with cross-domain passive tracking, as determined by the extension's
maintainer, Ghostery, Inc..

This work also uses the AdBlock Plus extension, which modifies
the browser to not load resources the extension associates with
advertising, and to hide elements in the page that are advertising related.
AdBlock Plus draws from a crowdsourced list of rules and URLs to determine
if a resource is advertising-related.

This work used each extension's default configuration, including the default
rule sets for which elements and resources to block.  No changes were made to
the configuration or implementation of either extension.

\section{Results}
\label{cost-benefit:results}

This section presents the results of applying the methodology discussed in
Section~\ref{cost-benefit:methodology} to \FFWithVersion.  The section first describes
the benefit of each \WAS, and follows with the cost measurements.

\begin{table}[ht]
  \captionsetup{font=footnotesize,singlelinecheck=off}
  \centering
  %   \rowcolors{2}{gray!25}{white}
  \resizebox{\textwidth}{!}{
  \begin{tabular}{ l | l r r r | r r | r | l }
    \toprule
      Standard Name                                     & Abbreviation & \#
    \ATK & Site Break & Agree & \# CVEs & \# High or & \% ELoC & Enabled \\
                                                        &              & Using
      & Rate & \% & & Severe & & attacks \\
    \midrule
      WebGL                                               &  WEBGL       & 852   & \textless1\% & 93\%  & 31  & 22 & 27.43 & \cite{laperdrix2016beauty,alaca2016device,ho2014tick,cao2017cross} \\ % 20751 \\
      HTML: Web Workers                                   &  H-WW        & 856   & 0\%          & 100\% & 16  & 9  & 1.63  & \cite{ho2014tick,gras2017aslr} \\ % 1238 \\
      WebRTC                                              &  WRTC        & 24    & 0\%          & 93\%  & 15  & 4  & 2.48  & \cite{englehardt2016online,alaca2016device} \\ % 1878 \\
      HTML: The canvas element                            &  H-C         & 6935  & 0\%          & 100\% & 14  & 6  & 5.03  & \cite{englehardt2016online,laperdrix2016beauty,alaca2016device,acar2014web,kotcher2013cross,ho2014tick,cao2017cross} \\ % 3810 \\
      Scalable Vector Graphics                            &  SVG         & 1516  & 0\%          & 98\%  & 13  & 10 & 7.86  & \\ % 5949 \\
      Web Audio API                                       &  WEBA        & 148   & 0\%          & 100\% & 10  & 5  & 5.79  & \cite{englehardt2016online,alaca2016device} \\ % 4380 \\
      XMLHttpRequest                                      &  AJAX        & 7806  & 32\%         & 82\%  & 11  & 4  & 1.73  & \\ % 1312 \\
      HTML                                                &  HTML        & 8939  & 40\%         & 85\%  & 6   & 2  & 0.89  & \cite{nikiforakis2013cookieless,acar2013fpdetective} \\ % 677 \\
      HTML 5                                              &  HTML5       & 6882  & 4\%          & 97\%  & 5   & 2  & 5.72  & \\ % 4332 \\
      Service Workers                                     &  SW          & 0     & 0\%          & -     & 5   & 0  & 2.84  & \cite{van2016request,gelernter2015cross,van2015clock} \\ % 2150 \\
      HTML: Web Sockets                                   &  H-WS        & 514   & 0\%          & 95\%  & 5   & 3  & 0.67  & \\ % 508 \\
      HTML: History Interface                             &  H-HI        & 1481  & 1\%          & 96\%  & 5   & 1  & 1.04  & \\ % 790 \\
      Indexed Database API                                &  IDB         & 288   & \textless1\% & 100\% & 4   & 2  & 4.73  & \cite{alaca2016device,acar2014web} \\ % 3583 \\
      Web Cryptography API                                &  WCR         & 7048  & 4\%          & 90\%  & 4   & 3  & 0.52  & \\ % 400 \\
      Media Capture and Streams                           &  MCS         & 49    & 0\%          & 95\%  & 4   & 3  & 1.08  & \cite{tian2014all} \\ % 819 \\
      DOM Level 2: HTML                                   &  DOM2-H      & 8956  & 13\%         & 89\%  & 3   & 1  & 2.09  & \\ % 1588 \\
      DOM Level 2: Traversal and Range                    &  DOM2-T      & 4406  & 0\%          & 100\% & 3   & 2  & 0.04  & \\ % 32 \\
      HTML 5.1                                            &  HTML51      & 2     & 0\%          & 100\% & 3   & 1  & 1.18  & \\ % 895 \\
      Resource Timing                                     &  RT          & 433   & 0\%          & 98\%  & 3   & 0  & 0.10  & \\ % 81 \\
      Fullscreen API                                      &  FULL        & 229   & 0\%          & 95\%  & 3   & 1  & 0.12  & \\ % 96 \\
      Beacon                                              &  BE          & 2302  & 0\%          & 100\% & 2   & 0  & 0.23  & \\ % 175 \\
      DOM Level 1                                         &  DOM1        & 9113  & 63\%         & 96\%  & 2   & 2  & 1.66  & \\ % 1258 \\
      DOM Parsing and Serialization                       &  DOM-PS      & 2814  & 0\%          & 83\%  & 2   & 1  & 0.31  & \\ % 235 \\
      DOM Level 2: Events                                 &  DOM2-E      & 9038  & 34\%         & 96\%  & 2   & 0  & 0.35  & \\ % 270 \\
      DOM Level 2: Style                                  &  DOM2-S      & 8773  & 31\%         & 93\%  & 2   & 1  & 0.69  & \\ % 523 \\
      Fetch                                               &  F           & 63    & \textless1\% & 90\%  & 2   & 0  & 1.14  & \cite{van2016request,gelernter2015cross,van2015clock} \\ % 869 \\
      CSS Object Model                                    &  CSS-OM      & 8094  & 5\%          & 94\%  & 1   & 0  & 0.17  & \cite{nikiforakis2013cookieless} \\ % 131 \\
      DOM                                                 &  DOM         & 9050  & 36\%         & 94\%  & 1   & 1  & 1.29  & \\ % 983 \\
      HTML: Plugins                                       &  H-P         & 92    & 0\%          & 100\% & 1   & 1  & 0.98  & \cite{alaca2016device,acar2013fpdetective} \\ % 743 \\
      File API                                            &  FA          & 1672  & 0\%          & 83\%  & 1   & 0  & 1.46  & \\ % 1110 \\
      Gamepad                                             &  GP          & 1     & 0\%          & 71\%  & 1   & 1  & 0.07  & \\ % 60 \\
      Geolocation API                                     &  GEO         & 153   & 0\%          & 96\%  & 1   & 0  & 0.26  & \cite{xu2015ucognito,kim2014exploring} \\ % 198 \\
      High Resolution Time Level 2                        &  HRT         & 5665  & 0\%          & 100\% & 1   & 0  & 0.02  & \cite{gelernter2015cross,andrysco2015subnormal,oren2015spy,van2015clock,kotcher2013cross,ho2014tick,gruss2015practical,gras2017aslr} \\ % 22 \\
      HTML: Channel Messaging                             &  H-CM        & 4964  & 0\%          & 0.025 & 1   & 0  & 0.40  & \cite{weissbacher2015zigzag,son2013postman} \\ % 309 \\
      Navigation Timing                                   &  NT          & 64    & 0\%          & 98\%  & 1   & 0  & 0.09  & \\ % 75 \\
      Web Notifications                                   &  WN          & 15    & 0\%          & 100\% & 1   & 1  & 0.82  & \\ % 622 \\
      Page Visibility (Second Edition)                    &  PV          & 0     & 0\%          & -     & 1   & 1  & 0.02  & \\ % 20 \\
      UI Events                                           &  UIE         & 1030  & \textless1\% & 100\% & 1   & 0  & 0.47  & \\ % 362 \\
      Vibration API                                       &  V           & 1     & 0\%          & 100\% & 1   & 1  & 0.08  & \\ % 67 \\
      Console API                                         &  CO          & 3     & 0\%          & 100\% & 0   & 0  & 0.59  & \cite{ho2014tick} \\ % 452 \\
      CSSOM View Module                                   &  CSS-VM      & 4538  & 0\%          & 100\% & 0   & 0  & 2.85  & \cite{acar2013fpdetective} \\ % 2159 \\
      Battery Status API                                  &  BA          & 2317  & 0\%          & 100\% & 0   & 0  & 0.15  & \cite{englehardt2016online,alaca2016device,nikiforakis2013cookieless,olejnik2015leaking} \\ % 120 \\
      CSS Conditional Rules Module Lvl 3                &  CSS-CR      & 416   & 0\%          & 100\% & 0   & 0  & 0.16  & \\ % 126 \\
      CSS Font Loading Module Level 3                     &  CSS-FO      & 2287  & 0\%          & 98\%  & 0   & 0  & 1.24  & \cite{alaca2016device,acar2013fpdetective} \\ % 942 \\
      DeviceOrientation Event                             &  DO          & 0     & 0\%          & -     & 0   & 0  & 0.06  & \cite{das2016tracking,alaca2016device} \\ % 48 \\
      % Directory Upload                                    &  DU          & 0     & 0\%          & -     & 0   & 0  & 0.94  & \\ % 714 \\
      DOM Level 2: Core                                   &  DOM2-C      & 8896  & 89\%         & 97\%  & 0   & 0  & 0.29  & \\ % 225 \\
      DOM Level 3: Core                                   &  DOM3-C      & 8411  & 4\%          & 96\%  & 0   & 0  & 0.25  & \\ % 194 \\
      DOM Level 3: XPath                                  &  DOM3-X      & 364   & 1\%          & 97\%  & 0   & 0  & 0.16  & \\ % 126 \\
      % Encoding                                            &  E           & 1     & 0\%          & 100\% & 0   & 0  & 0.21  & \\ % 165 \\
      Encrypted Media Extensions                          &  EME         & 9     & 0\%          & 100\% & 0   & 0  & 1.91  & \\ % 1449 \\
      % execCommand                                         &  EC          & 2419  & 0\%          & 100\% & 0   & 0  & 0.00  & \\ % 0 \\
      % Geometry Interfaces Module Level 1                  &  GIM         & 0     & 0\%          & -     & 0   & 0  & 0.71  & \\ % 538 \\
      % HTML: Broadcasting                                  &  H-B         & 1     & 0\%          & 100\% & 0   & 0  & 0.26  & \\ % 204 \\
      HTML: Web Storage                                   &  H-WB        & 7806  & 0\%          & 83\%  & 0   & 0  & 0.55  & \cite{alaca2016device,xu2015ucognito,ho2014tick} \\ % 421 \\
      % HTML Editing APIs                                   &  H-E         & 0     & 0\%          & -     & 0   & 0  & 0.00  & \\ % 0 \\
      % Media Capture from DOM Elements                     &  MCD         & 0     & 0\%          & -     & 0   & 0  & 0.28  & \\ % 212 \\
      Media Source Extensions                             &  MSE         & 1240  & 0\%          & 95\%  & 0   & 0  & 1.97  & \\ % 1495 \\
      % MediaStream Recording                               &  MSR         & 0     & 0\%          & -     & 0   & 0  & 0.55  & \\ % 417 \\
      % Performance Timeline                                &  PT          & 4535  & 0\%          & 95\%  & 0   & 0  & 0.04  & \\ % 32 \\
      % Performance Timeline Level 2                        &  PT2         & 1672  & 0\%          & 98\%  & 0   & 0  & 0.21  & \\ % 159 \\
      % Pointer Lock                                        &  PL          & 18    & 0\%          & 100\% & 0   & 0  & 0.03  & \\ % 29 \\
      % Proximity Events                                    &  PE          & 18    & 0\%          & 100\% & 0   & 0  & 0.01  & \\ % 8 \\
      % Selection API                                       &  SEL         & 2287  & 0\%          & 100\% & 0   & 0  & 0.02  & \\ % 22 \\
      Selectors API Level 1                               &  SLC         & 8611  & 15\%         & 89\%  & 0   & 0  & 0.00  & \\ % 0 \\
      % Shadow DOM                                          &  SD          & 0     & 0\%          & -     & 0   & 0  & 0.34  & \\ % 261 \\
      % The Screen Orientation API                          &  SO          & 38    & 0\%          & 100\% & 0   & 0  & 0.34  & \\ % 261 \\
      Script-based animation timing control          &  TC          & 3437  & 0\%          & 100\% & 0   & 0  & 0.08  & \cite{nikiforakis2013cookieless} \\ % 67 \\
      % Tracking Preference Expression                      &  TPE         & 0     & 0\%          & -     & 0   & 0  & 0.01  & \\ % 10 \\
      % URL                                                 &  URL         & 5     & 0\%          & 100\% & 0   & 0  & 0.59  & \\ % 453 \\
      % User Timing Level 2                                 &  UTL         & 3077  & 0\%          & 100\% & 0   & 0  & 0.44  & \\ % 340 \\
      % W3C DOM4                                            &  DOM4        & 5639  & 0\%          & 100\% & 0   & 0  & 0.45  & \\ % 346 \\
      % WebVTT                                              &  WEBVTT      & 0     & 0\%          & -     & 0   & 0  & 0.44  & \\ % 334 \\
      Ambient Light Sensor API                            &  ALS         & 18    & 0\%          & 89\%  & 0   & 0  & 0.00  & \cite{nikiforakis2013cookieless,olejnik2017stealing} \\ % 0 \\
    \bottomrule
  \end{tabular}
  }
  \caption{
    Data on all \NumStandards measured \WASs, excluding 20 standards with a 0\% break rate, 0 associated CVEs
    and accounting for less than one percent of measured effective lines of code.
  }
  \begin{small}
    \begin{itemize}
      \setlength{\itemsep}{-2pt}
      \item The standard's full name
      \item The abbreviation used when referencing this standard in this work
      \item The number of sites in the Alexa 10k using the standard (Section~\ref{measuement:results})
      \item The portion of measured sites that were broken by disabling the
        standard. (Section~\ref{cost-benefit:methodology:per-standard-benefit})
      \item The agreement between testers' evaluation (Section~\ref{cost-benefit:methodology:per-standard-benefit})
      \item The number of CVEs since 2010 associated with the feature (Section~\ref{cost-benefit:results:costs-cves})
      \item The number of CVEs since 2010 ranked as ``high'' or ``severe'' (Section~\ref{cost-benefit:results:costs-cves})
      \item The percentage of \gls{eloc} for this
        standard, as a percentage of all attributed lines (Section~\ref{cost-benefit:methodology:costs-loc})
      \item Citations for papers describing attacks relying on the standard (Section~\ref{cost-benefit:results:costs-research})
    \end{itemize}
  \end{small}
  \label{table:megatable}
\end{table}


\subsection{Per-Standard Benefit}
\label{cost-benefit:results:results-benefit}

\begin{figure}[ht]
  \centering
  \includegraphics[width=.5\textwidth]{figures/breakrate_histogram.pdf}
  \caption{A histogram giving the number of standards binned by the percentage of sites that broke when removing the standard.}
  \label{fig:feature-benefit}
\end{figure}


As explained in Section~\ref{cost-benefit:methodology:per-standard-benefit}, our workers
conducted up to \NumSitesPerStandard measurements of websites in the \ATK known
to use each specific \WAS. If a standard was observed being used fewer than
\NumSitesPerStandard times within the \ATK, all sites using that standard were
measured. In total, we did two measurements of 1,684 (website, disabled feature)
tuples, one by each worker.

\ref{fig:feature-benefit} gives a histogram of the break rates for each of
the \NumStandards standards measured in this work.  As the graph shows, removing
over 60\% of the measured standards resulted in no noticeable effect on the
user's experience.

In some cases, this was because the standard was never observed being
used\footnote{e.x. the \textit{WebVTT} standard, which allows document
authors to synchronize text changes with media playing on the page.}.
In other cases, it was because the standard is intended to be used in a way
that users do not notice\footnote{e.x. the \textit{Beacon} standard, which allows content
authors to trigger code execution when a user browses away from a website.}.

Other standards caused a large number of sites to break when removed
from the browser.  Disabling access to the \textit{DOM 1} standard (which provides
basic functionality for modifying the text and appearance of a document)
broke an estimated 69.05\% of the web.

A listing of the site break rate for all \NumStandards standards is provided in
in \ref{table:megatable}.

We note that these measurements only cover the interacting with a website
as an unauthenticated user. It is possible that site feature use changes when
users log into websites, since some sites only provide full
functionality to registered users.  These numbers only describe the functionality
sites use before they've established a trust-relationship with the site (e.g.
before they've created an account and logged into a web site).


\subsection{Per-Standard Cost}
\label{cost-benefit:results:results-costs}
As described in Section~\ref{cost-benefit:methodology:per-standard-cost}, we measure the cost of a
\WAS being available in the browser in three ways: first by related research
documenting security and privacy attacks that leverage the standard
(Section~\ref{cost-benefit:results:costs-research}), second by the number of
historical CVEs reported against the standard since 2010
(Section~\ref{cost-benefit:results:costs-cves}), and third with a lower bound estimate of
the number of \gls{eloc} needed to implement the
standard in the browser (Section~\ref{cost-benefit:results:costs-loc}).


\subsubsection{Security Costs - Attacks from Related Research}
\label{cost-benefit:results:costs-research}

We searched the last five years of work published at major research
conferences and journals for research on browser weaknesses related
to \WASs.  These papers either explicitly identify either
\WASs, or features or functionality that belong to a \WAS.  In each case the standard
was either necessary for the attack to succeed, or was used to make the attack
faster or more reliable.  While academic attacks do not aim to discover all possible vulnerabilities,
the academic emphasis on novelty mean that the \WASs implicated in these attacks
allow a new, previously undiscovered way to exploit the browser.

The most frequently cited standard was the
\textit{High Resolution Time Level 2}~\cite{highres2016w3c} standard, which
provides highly accurate, millisecond-resolution timers.  Seven
papers published since 2013 leverage the standard to break the isolation protections provided
by the browser, such as learning information about the environment the browser
is running in~\cite{ho2014tick,oren2015spy,gruss2015practical}, learning information about
other open browser windows~\cite{andrysco2015subnormal,kotcher2013cross,gruss2015practical}, and gaining
identifying information from other domains~\cite{van2015clock}.

Other implicated standards include the \textit{Canvas} standard, which
was identified by researchers as allowing attackers to persistently
track users across websites~\cite{acar2014web}, learn about the browser's
execution environment~\cite{ho2014tick} or obtain information from other
browsing windows~\cite{kotcher2013cross}, and the \textit{Media Capture and Streams}
standard, which was used by researchers to perform ``cross-site request forgery, history sniffing, and information
stealing'' attacks~\cite{tian2014all}.

In total we identified \NumAttackPapers papers leveraging \NumAttackStandards
standards to attack the privacy and security protections of the web browser.
Citations for these papers are included in \ref{table:megatable}.


\subsubsection{Security Costs - CVEs}
\label{cost-benefit:results:costs-cves}

\input{figures/cve_breakrate}
\begin{figure}[t]
  \centering
  \includegraphics[width=.5\textwidth]{figures/cve_breakrate_severe.pdf}
  \caption{A scatter plot showing the number of ``high'' or ``severe'' CVEs filed against each standard since 2010, by how many sites in the Alexa 10k break when the standard is removed.}
  \label{fig:cve-breakrate-severe}
\end{figure}


Vulnerability reports are not evenly distributed across browser standards.
Figure \ref{fig:cve-breakrate-severe} presents this comparison of
standard benefit (measured by the number of sites that require the standard
to function) on the y-axis, and the number of severe CVEs historically associated with
the standard on the x-axis.  A plot of all CVEs (not just high and severe ones),
is included in the appendix as Figure \ref{fig:cve-breakrate}.
It shows the same general relationships between break rate and CVEs as
Figure \ref{fig:cve-breakrate-severe}, and is included for completeness.

Points in the upper-left of the graph depict standards that are high benefit,
low cost, i.e. standards that are frequently required on the web but have
rarely (or never) been implicated in CVEs.  For example, consider the
\textit{Document Object Model (DOM) Level 2 Events Specification} standard,
denoted by \textbf{DOM2-E} in Figure \ref{fig:cve-breakrate-severe}.  This
standard defines how website authors can associate functionality with page
events such as button clicks and mouse movement.  This standard is highly
beneficial to browser users, being required by 34\% of pages to function
correctly.  Enabling the standard comes with little risk to web users,
being associated with zero CVEs since 2010.

Standards in the lower-right section of the graph, by contrast, are low benefit,
high cost standards, when using historical CVE counts as an estimate of security
cost.   The \textit{WebGL Specification} standard, denoted by \textbf{WEBGL}
in Figure \ref{fig:cve-breakrate-severe}, is an example of such a low-benefit,
high-cost standard.  The standard allows websites to take advantage of graphics
hardware on the browsing device for 3D graphics and other advanced image
generation.  The standard is needed for less than 1\% of web sites
in the \ATK to function correctly, but is implicated in 22 high or severe CVEs since
2010.  How infrequently this standard is needed on the web, compared with
how often the standard has previously been the cause of security
vulnerabilities, suggests that the standard poses a high security
risk to users going forward, with little attenuating benefit.

As \ref{fig:cve-breakrate} and \ref{fig:cve-breakrate-severe} show, some standards have historically
put users at much greater risk than others.  Given that for many of these
standards the risk has come with little benefit to users, these standards
are good candidates for disabling when visiting untrusted websites.


\subsubsection{Security Costs - Implementation Complexity}
\label{cost-benefit:results:costs-loc}

\begin{figure}[ht]
  \centering
  \includegraphics[width=.5\textwidth]{figures/loc_breakrate.pdf}
  \caption{A scatter plot showing the LOC measured to implement each standard, by how many sites in the Alexa 10k break when the standard is removed.}
  \label{fig:loc-breakrate}
\end{figure}


We further found that the cost of implementing standards in the browser are
not equal, and that some standards have far more complex implementations than
others (with complexity measured as the \gls{eloc} uniquely needed to
implement a given standard). Figure \ref{fig:loc-breakrate} presents a
comparison of standard benefit (again measured by the number of sites that
require the standard to function) and the exclusive lines of code needed to implement the standard, using the method
described in section \ref{cost-benefit:results:costs-loc}.

Points in the upper-left of Figure \ref{fig:loc-breakrate} depict standards
that are frequently needed on the web for sites for function correctly,
but which have relatively non-complex implementations.  One example of
such a standard is the \textit{Document Object Model (DOM) Level 2 Core
Specification} standard, denoted by \textbf{DOM2-C}.  This standard
provides extensions the browser's basic document modification methods,
most popularly, the \texttt{Document.prototype.createDocumentFragment}
method, which allows websites to quickly create and append sub-documents
to the current website.  This method is needed for 89\% of websites to
function correctly, suggesting it is highly beneficial to web users to have it
enabled in their browser.  The standard comes with a low security cost to users as well;
our technique identifies only 225 exclusive lines of code that are in
the codebase solely to enable this standard.

Points in the lower-right of the figure depict standards that provide
infrequent benefit to browser users, but which are responsible for a great
deal of complexity in the browser's code base.  The \textit{Scalable Vector
Graphics (SVG) 1.1 (Second Edition)} standard, denoted by \textbf{SVG},
is an example of such a high-cost, low-benefit standard.  The standards
allows website authors to dynamically create and interact with embedded SVG
documents through \JS.  The standard is required for core functionality in
approximately 0\% of websites on the \ATK, while adding a large amount of
complexity to the browser's code base (at least 5,949 exclusive lines of
code, more than our technique identified for any other standard).


\subsection{Threats to Validity}
The main threat to validity in this experiment is the accuracy of our
human-executed casual browsing scenario. With respect to internal validity, the
high agreement between the two users performing tasks on the same sites lends
credence to the claim that the users were able to successfully exercise most or
all of the functionality that a casual browser might encounter. The students
who worked on this project spent over 500 hours combined performing these
casual browsing tasks and recording their results, and while they were
completely separated while actively browsing, they spent a good deal of time
comparing notes about how to fully exercise the functionality of a website
within the 70 second time window for each site.

External validity, i.e. the extent to which our results can be generalized, is also
a concern. However, visiting a website for 70 or fewer seconds encapsulates
80\% of all web page visits according to~\cite{liu2010understanding}, thus
accurately representing a majority of web browsing activity, especially when
visiting untrusted websites. Furthermore, while our experiment does not
evaluate the \JS functionality that is only available to authenticated users,
we posit that protection against unknown sites---the content aggregators,
pop-up ads, or occasionally consulted websites that a user does not interact
with enough to trust---are precisely the sites with which the user should
exercise the most caution.
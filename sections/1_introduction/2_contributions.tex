\section{Contributions}
\label{intro:contributions}

The underlying motivation, and core thesis, of this work, is that privacy
and security on the web can be improve, with only small effects on usability,
by restricting what functionality websites can access.  This conclusion
is built to in four step, through the following following contributions:

\begin{itemize}
    \item A web-scale measurement of how the \WAPI is used on the web today.
          This study contained an automated measurement of browser functionality
          used on sites in the Alexa 10k (at time of measurement), with
          a novel method of distinguishing between features used
          for core-site-functionality, and features used for
          non-user-serving purposes (e.g. advertising and tracking).
    \item A comprehensive measurement of the costs and benefits of each standard
          in the \WAPI.  This work models a standard's
          benefit as the number of sites on the web that require the standard
          to carry out the site's core functionality.  The work models
          a standard's cost in three ways: as the number of recent publicity
          disclosed vulnerabilities relating to the standard, the number
          of lines of code uniquely needed to implement the standard, and
          the number of papers describing attacks that leverage the standard
          in top security conferences and journals.
    \item Findings learned and vulnerabilities discovered in the development
          and maintenance of a browser extension to restrict \WAPI access on
          the web.  This work also presents useability
          measurements taken during an in-lab evaluation of the tool, and
          broader findings from real-world use of the extension.
    \item The design and evaluation of \gls{cdf}, a system of developing and deploying
          web applications with functionality similar to most modern websites,
          but providing client-enforced protections and dataflow
          guarantees.
\end{itemize}

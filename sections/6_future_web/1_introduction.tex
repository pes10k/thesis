\section{Introduction}
\label{future-web:introduction}

A final step this dissertation makes in improving web privacy and security is
to explore other ways web-like applications could be developed and deployed, in
light of the findings discussed earlier in earlier chapters.  While
Chapter~\ref{current-web} presented ways of improving web security and privacy
while maintaining compatibility with existing websites, this chapter
considers the further improvements that could be achieved with a system
designed from the start with the findings from Chapter~\ref{measurement} and
Chapter~\ref{cost-benefit}.

This chapter presents \CDF, an alternative method for describing interactive
websites. The design requires site authors to describe sites using a
declarative, statically checkable format, that trades a loss in
author-expressiveness for gains in client-enforceable security guarantees.
The design is intended to serve as a proof of concept that many of the kinds
of websites users enjoy on the modern web can be achieved with only a subset of
the functionality in the browser, and in a manner that allows the client to
enforce a greater number of protections and guarantees.

The rest of this chapter is organized as follows:
Section~\ref{future-web:design} presents the high level design of the system.
Section~\ref{future-web:implementation} describes an implementation of that
design way that leverages the engineering and omnipresence of commodity
web browsers.  Section~\ref{future-web:evaluation} gives an evaluation of the
capabilities and security guarantees provided by the system, and
Section~\ref{future-web:implementation} discusses the limitations of the
current design, and the types of applications the design would not be able to
deliver.  Section~\ref{future-web:conclusions} discusses this work's place in
the overarching goals of this dissertation.

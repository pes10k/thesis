\section{Introduction}
\label{current-web:current-web}

The third step towards the over-arching goal of improving web security and
privacy is to consider how the per-standard cost and benefit measurements
described in Chapter~\ref{cost-benefit} can be used to improve the web as it
exists today.  More specifically, to explore how the data and techniques
discussed in the previous chapters can be used to better protect web users
visiting websites, as they are constructed now.

This chapter describes two such efforts.  The first is the design and development
of a browser-extension that imposes access controls on which \WAPI features
web sites can access.  The second approach is through modifications to the Brave
browser,a privacy oriented web browser designed (in part) to prevent tracking online.
In both cases, the goal is to modify the DOM to prevent websites from gaining
access to features that risk user's security and privacy, though the general
approach differs in each case. Our browser extension attempts to limit web
sites to a minimal set of features, judged to be high benefit and low cost.
The Brave browser has identified a small set of \WAPI features that
are frequently used to track users online, and (by default) prevents sites
from using those features.

This chapter discusses lessons learned by working with both approaches, including
the discovery of a significant vulnerability in the way the Firefox and Chrome
browsers implement the WebExtension standard, which significantly limits
the security and privacy guarantees our extension can enforce.  This discovered
vulnerability affects many other privacy-oriented WebExtensions, and most
security tools are still vulnerable at the time this work is being written.

The rest of this chapter is organized as follows:  Section~\ref{current-web:extension}
describes the design of a browser extension we developed based on the findings
discussed in Chapter~\ref{cost-benefit}, along with a useability analysis of
the extension.  Section~\ref{current-web:extension-deployment} presents
issues and discoveries that were made after the browser extension was released
to the public, and began being used by approximately 1k users.

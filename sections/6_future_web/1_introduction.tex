\section{Introduction}
\label{future-web:introduction}

The final step this dissertation makes in improving web privacy and security is
to explore other ways web-like applications could be developed and deployed, in
light of the findings discussed in previous chapters.
 Chapter~\ref{current-web} presented ways of improving web security and privacy
while maintaining compatibility with existing websites. This chapter considers
the further improvements that could be achieved with a system designed from the
start with the findings from Chapters~\ref{measurement} and \ref{cost-benefit}.

This chapter presents \CDF, an alternative method for describing interactive
websites. The design requires site authors to describe sites using a
declarative, statically checkable format, that trades a loss in
author-expressiveness for gains in client-enforceable security guarantees.  The
design is as a proof of concept, to demonstrate that many kinds of websites
users enjoy on the modern web can be implemented with only a subset of the
functionality in the browser, and in a manner that allows the client to enforce
a greater number of protections and guarantees.

The discussed system is intended demonstrate two observations informed by the
chapters: first, that there is a large class of websites that do not need
the expressiveness and full feature set of the existing \gls{html} and \JS
system, and second, that there are security and privacy benefits to be had
in moving this class of websites to alternative web application systems.
We emphasize that the system described in this chapter is not intended to
be adopted as is, or that it makes strong claims to any specific
security and privacy benefits.  It is instead intended as a demonstration
that other points on the security-functionality tradeoff curve are possible,
and that the costs (in terms of developer expressiveness) of moving to those
other, more secure and private, points may be less than expected.

The rest of this chapter is organized as follows:
Section~\ref{future-web:design} presents the high level design of the system.
Section~\ref{future-web:implementation} describes an implementation of \CDF~'s
design that leverages the engineering and omnipresence of commodity web
browsers.  Section~\ref{future-web:evaluation} gives an evaluation of the
system's capabilities and security guarantees, a discussion of the limitations
of the current design, and the types of applications the system could not
implement, in its current state.  Section~\ref{future-web:conclusions}
discusses this work's place in the overarching goals of this dissertation.

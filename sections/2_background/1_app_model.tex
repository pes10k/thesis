\section{The Web Application Model}
\label{background:web-application-model}

This section provides a brief overview of how web applications are designed,
and the role of the \WAPI in building modern web applications.
This description is not intended to be comprehensive, but to provide enough
context so that the rest of this dissertation can be understood
by readers without experience in web development.

Browsers allow \JS code to interact with websites in several steps. First,
browsers parse the received \gls{html} into a \JS accessible data-structure,
that roughly-mirrors the tree-based structure of the original document.
This structure allows \JS code to access and modify the document-tree,
using \JS properties and functions.  These tree-modifying \JS features are
collectively called the \gls{dom}, and are standardized across browsers.

Browsers also provide websites access to a large number of \JS features
that are not directly related to modifying the document-tree.  These
features range widely in purpose, including allowing sites to
access device hardware, perform high resolution timing operations, and
perform network \gls{io}.  The term ``Web \gls{api}'' refers to the union of
functionality related to modifying the rendered \gls{html} document, and
these additional \JS features.

Modern web applications are thus the combination of two sources of code:
first, the \WAPI functionality implemented in the browser, and second,
the website's \JS code that uses the \WAPI to implement the site's functionality.

Web applications have unique properties that make them difficult to secure.
First, all \JS code has access to nearly all functionality in the \WAPI.
There are a few exceptions, where users are
prompted to allow access to certain functionalities, relating
to tasks like accessing hardware devices and geo-locating the user.  These
restrictions apply to a tiny fraction of features; websites have
access by default to the vast majority of the \WAPI.

Second, the web application model lacks a formal
way of allowing code units to interact.  Instead, all code units are
executed in a common namespace, and code units collaborate by accessing and
modifying a single global variable (implemented in the browser as
\texttt{window}), or by modifying the globally accessible representation of the
\gls{html} document (implemented in the browser as \texttt{window.document}).
This model makes it difficult to execute a code unit without allowing
it to read and modify the execution environment.  All code executed in the
page (whether that code was intended by the page author, loaded by a third
party to implement an advertising system, or maliciously include as part of
an \gls{xss} attack) gets equal access to the capabilities, secrets and data
available to the website or application.


\section{Client Side Browser Defenses}
\label{background:related-browser-defs}

There are variety of techniques which ``harden'' the browser against attacks
via limiting what \JS is allowed to run within the browser. These defenses can
be split into two categories: those configured by the user, and those
configured by the website author.

In the user-configured category, both Adblock and NoScript prevent \JS from
running based on the site serving it.  Adblock~\cite{adblockplus} primary
function is to block ads for aesthetic purposes, but it can also prevent
infection by malware being served in those
ads~\cite{forbes-malware,engadget-malware}.  Adblock blocks feature use by
preventing the loading of resources from certain domains.

NoScript~\cite{noscriptwebsite} prevents \JS on an
all-or-nothing basis, based on the code's origin.  By default, NoScript prevents
\JS execution from all origins, rendering a large swath of the web unusable.  
In its default configuration, NoScript does allows \JS execution from a built
in set of trusted origins.  This built in allow-list has resulted in a proof of
concept exploit via purchasing expired, allowed domains~\cite{noscript_whitelist}.
Beyond these popular tools, IceShield~\cite{heiderich2011iceshield}
dynamically detects suspicious \JS calls within the browser, and modifies the
\gls{dom} to prevent attacks.

The Tor Browser~\cite{dingledine2004tor} disables by default or prompts the
user before using a number of features.  Regarding \JS, they disable
SharedWorkers~\cite{webworkersw3c}, and prompt before using calls from HTML5
Canvas, the GamePad API, WebGL, the Battery API, and the Sensor
API~\cite{tor-features}.  These particular features are disabled because they
enable techniques which violate the Tor Browser's security and privacy goals.

On the website author side, \gls{csp} allows limiting of the
functionality of a website, but rather than allowing browser users to decide
what will be run, \gls{csp} allows web developers to constrain code on their own
sites so that potential attack code cannot access functionality deemed
unnecessary or dangerous~\cite{stamm2010reining}.  Conscript is another
client-side implementation which allows a hosting page to specify policies for
any third-party scripts it includes~\cite{meyerovich2010conscript}.  There are
also a number of technologies selected by the website author but enforced on
the client side,  including Google Caja~\cite{google13caja} and
GATEKEEPER~\cite{guarnieri09gatekeeper}.

There are existing models for enforcing policies to limit functionality outside
of the web browser as well.  Mobile applications use a richer permission model
where permission to use certain features is asked of the user at either install
or run-time~\cite{android-permissions,au2011short}.

\section{Measurement Methodology}
\label{measurement:methodology}

To understand browser feature use on the open web, we conducted a survey of the
Alexa 10k, visiting each site ten times and recording which browser features
were used.  We visited each site five times with an unmodified browsing
environment, and five times with popular tracking-blocking and
advertising-blocking extensions installed.  This section describes
the goals of this survey, followed by how we instrumented the browser to
determine which features are used on a given site, and then concludes with how
we used our instrumented browser to measure feature use on the web in general.


\subsection{Goals}
The goal of our automated survey is to determine which browser features
are used on the web as it is commonly experienced by users.  This requires us
to take a broad-yet-representative sample of the web, and to exhaustively
determine the features used by those sites.

To do so, we built a browser extension to measure which features are used when
a user interacts with a website. We then chose a representative sample of the
web to visit.  Finally, we developed a method for interacting with these sites in an
automated fashion to elicit the same functionality that a human web
user would experience.  Each of these steps is described in detail
in the proceeding subsections.

This automated approach only attempts to measure the ``open web'', or the
subset of webpage functionality that a user encounters \textit{without} logging
into a website.  Users may encounter different types of functionality when
interacting with websites they have created accounts for and established
relationships with, but such measurements are beyond the scope of this paper.
We note this restriction, only measuring functionality used by non-authenticated
portions of websites, as a limitation of this paper and a possible area for
future work.


\subsection{Measuring Extension}
\label{measurement:methodology:measureextension}

We instrumented a recent version of the \FF web
browser (version \FFversion) with a custom browser extension to
records each time a \JS feature has been used on a visited page.  Our extension
injects \JS into each page after the browser has created the DOM for that page,
but before the page's content has been loaded. By injecting our instrumenting
\JS into the browser before the page's content has been fetched and rendered, we
can modify the methods and properties in the DOM before it becomes
available to the requested page.

The \JS that the extension injects into each requested page modifies
the DOM to count when an instrumented method is called or that an
instrumented property is written to.  How the extension measures these method calls
and property writes is detailed in the following two subsections.
% Figure~\ref{fig:gremlins} presents a representative diagram of the crawling process.


% \begin{figure}[tb]
%     \centering
% \centerline{
%     \includegraphics[width=\columnwidth]{figures/crawl_gremlins.png}
% }
%     \caption{One iteration of the feature invocation measurement process.}
%     \label{fig:gremlins}
%     \vspace*{-0.1in}
% \end{figure}

\subsubsection{Measuring Method Calls}
The browser extension counts when a method has been invoked by overwriting the method on
the containing object's prototype.  This approach allows us to shim in our own
logging functionality for each method call, and then call the original method
to preserve the original functionality.  We replace each reference to each
instrumented method in the DOM with an extension managed, instrumented
method.

We take advantage of closures in \JS to ensure that web
pages are not able to bypass the instrumented methods by
looking up--or otherwise directly accessing--the original versions of each
method.


\subsubsection{Measuring Property Writes}
Properties were more difficult to instrument.  \JS provides no way to
intercept when a property has been set or read on a client script-created object,
or on an object created after the instrumenting code has finished executing.
However, through the use of the non-standard
\texttt{Object.watch()}\cite{mozillaobjectwatch} method in
\FF, we were able to capture when pages set properties on
the singleton objects in the browser (e.g. \texttt{window},
\texttt{window.document}, \texttt{window.navigator}).  Using this
\texttt{Object.watch()} method allowed the extension to capture and count
all writes to properties on singleton objects in the DOM.

There are a small number of features in the \texttt{DOM} where we were not
able to intercept property writes.  We were therefore unable to count how frequently
these features were used.  These features, found primarily in older standards,
are properties where writes trigger side effects on the page.  The most significant
examples of such properties are \texttt{document.location} (where writing to
the property can trigger page redirection) and \texttt{Element.innerHTML}
(where writing to the property causes the subtree in the document to be replaced).
The implementation of these features in \FF make them unmeasurable using
our technique.  We note them here as a small but significant limitation of our
measurement technique.


\subsubsection{Other Browser Features}
Web standards define other features in the browser too, such as browser events
and CSS layout rules, selectors, and instructions.  Our extension-based
approach did not allow us to measure the use of these features, and so counts
of their use are not included in this work.

In the case of standard defined browser events (e.g. \texttt{onload},
\texttt{onmouseover}, \texttt{onhover}) the extension could have
captured some event registrations through a combination of watching for
event registrations with \texttt{addEventListener} method calls
and watching for property-sets to singleton objects.  However, we would not
have been able to capture event registrations using the legacy \texttt{DOM0}
method of event registration (e.g. assigning a function to an object's
\texttt{onclick} property to handle click events) on non-singleton objects.
Since we would only have been able to see a subset of event registrations,
we decided to omit events completely from this work.

Similarly, this work does not consider non-\JS exposed functionality
defined in the browser, such as CSS selectors and rules.  While interesting,
this work focuses solely on functionality that the browser allows websites to
access though \JS.


\subsection{Eliciting Site Functionality}
\label{measurement:methodology:eliciting-site-functionality}

Using our feature-detecting browser extension, we were able to measure which
browser features are used on the 10k most popular websites.  The following
subsections describe how we simulated human interaction
with web pages to measure feature use, first with the browser in its default
state, and again with the browser modified with popular advertising and tracking
blocking extensions.


\subsubsection{Default Case}
\label{measurement:methodology:default-case-measurements}

To understand which features are used in a site's execution, we installed the
instrumenting extension described in Section~\ref{measurement:methodology:measureextension}
and visited sites from the Alexa 10k, with the goal of
exercising as much of the functionality used on the page as possible.
While some \JS features of a site are automatically activated on the home page
(e.g. advertisements and analytics), many features will only be used as a result
of user interaction, either within the page or by navigating to different areas
of the site. Here we explain our strategy for crawling and interacting with sites.

In order to trigger as many browser features as possible on a website, we
used a common site testing methodology called ``monkey testing''.  Monkey
testing refers to the strategy of instrumenting a page to click, touch, scroll,
and enter text on random elements or locations on the page.  To accomplish this, we use a
modified version of gremlins.js~\cite{zaninotto2016gremlins}, a library built
for monkey testing front-end website interfaces.  We modified the gremlins.js
library to allow us to distinguish between when the gremlins.js script uses a
feature, and when the site being visited uses a feature.  The former feature usage is omitted from the
results described in this paper.

We started our measurement by visiting the home page of site and allowing
the monkey testing to run for 30 seconds.  Because the randomness of monkey
testing could cause navigation to other domains, we intercepted and prevented
any interactions which might navigate to a different page.  For
navigations that would have been to the local domain, we noted which URLs the
browser would have visited in the absence of the interception.

We then proceeded in a breadth first search of the site's hierarchy using the
URLs that would have been visited by the actions of the monkey
testing.  We selected 3 of these URLs that were on the same domain (or
related domain, as determined by the Alexa data), and visited each, repeating
the same 30 second monkey testing procedure and recording all used features.
From each of these 3 sites, we then visited three more pages for 30 seconds,
which resulted in a total of 13 pages interacted with for a total of 390 seconds
per site.

If more than three links were clicked during any stage of the monkey testing
process, we selected which URLs to visit by  giving preference to URLs where
the path structure of the URL had not been previously seen. In contrast to
traditional interface fuzzing techniques, which have as a goal finding unintended
or malicious functionality~\cite{amalfitano2012using,liu2014decaf}, we were
interested in finding all functionalities that users will commonly interact
with.  By selecting URLs with different path-segments, we tried to visit as many
types of pages on the site as possible, with the goal of
capturing all of the functionality on the site that a user would encounter.
The robustness and validity our strategy are evaluated in
Section~\ref{measurement:validity}.



\subsubsection{Blocking Case}
\label{measurement:methodology:blocking-case-measurements}
In addition to the default case measurements described in
Section~\ref{measurement:methodology:default-case-measurements}, we also re-ran the same measurements
against the Alexa 10k with an ad blocker (AdBlock Plus) and a tracking-blocker
(Ghostery) to generate a second, `blocking', set of measurements. We treat
these blocking extensions as representative of the types of modifications
users make to customize their browsing experience. While a so-modified version of a site
no longer represents its author's intended representation (and may in fact
break the site), the popularity of these content blocking extensions shows that
this blocking case is a common valid alternative experience of a website.


\subsubsection{Automated Crawl}
\label{measurement:methodology:automated-crawl}

\begin{table}[ht]
  \centering
  \begin{tabular}{ l r }
    \toprule
      Domains measured                    &  9,733 \\
      Total website interaction time      &  480 days\footnote{estimated based on the number of pages visited and 30 seconds of page interaction per visit.} \\
      Web pages visited                   &  2,240,484    \\
      Feature invocations recorded        &  21,511,926,733 \\
    \bottomrule
  \end{tabular}
  \caption{Amount of data gathered regarding \JS feature use on the Alexa 10k.}
  \label{fig:results-vanity-stats}
\end{table}


For each site in the Alexa 10k, we repeated the above procedure ten times
to ensure we measured all features used on the page, first five times in the
default case, and then again five times in the blocking case.  By parallelizing this
crawl with 64 \FF installs operating over 4 machines, we were able to complete the crawl in
two days.

We present findings for why
five times is sufficient to induce all types of site functionality in each test case in
Section~\ref{measurement:validity}.  \ref{fig:results-vanity-stats} presents
some high level figures of this automated crawl.  For 267 domains,
we were unable to measure feature usage for a variety of reasons, including
non-responsive domains and sites that contained syntax errors in
their \JS code that prevented execution.
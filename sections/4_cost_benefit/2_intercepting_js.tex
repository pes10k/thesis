\section{Intercepting \JS Functionality}
\label{cost-benefit:intercepting-js}

This Section presents a technique for interposing on, and optionally removing
access to, \WAPI features, that attempts to minimize the effect on
existing code bases that expect the now-removed feature to be present.
This approach is used both in building cost-benefit measurements described
in this chapter, and in the implementation of the browser-hardening extension
described in Chapter~\ref{current-web}.


\subsection{Removing Features from the DOM}
\label{cost-benefit:intercepting-js:featremove}
Each webpage and iframe gets its own global
window object.  Changes made to the global object are shared across all scripts
on the same page, but not between pages.  Furthermore, changes made to this
global object are seen immediately by all other script running in the page.  If
one script deletes or overwrites the \texttt{window.alert}
function, for example, no other scripts on the page will be able to use the
\texttt{alert} function, and there is no way they can recover it.

As a result, code executed earlier can arbitrarily modify the browser environment
seen by code executed later.  Since code run by browser extensions can
run before any scripts included by the page, extensions can modify the
browser environment for all code executed in any page.
The challenge in removing a feature from the browser environment is not to
\textit{just} prevent pages from reaching the feature, but to do so \textit{in way that
still allows the rest of the code on the page to execute without introducing errors}.

For example, to disable the \texttt{getElementsByTagName} feature, one could
simply remove the \texttt{getElementsByTagName} method from the
\texttt{window.document} object. However, this will result in fatal errors
if future code attempts to call that now-removed method.

Consider the code in \ref{fig:trivial-js}:  removing the
\texttt{window.document} \texttt{.getElementsByTagName} method will cause an error
on line one, as the site would be trying to call the now-missing property as if
were a function.  Replacing \texttt{getElementsByTagName} with a new, empty
function would solve the problem on line one, but would cause an error on line
two, unless the function returned an array of at least length five. Even after
accounting for that result, one would need to expect that the
\texttt{setAttribute} method was defined on the fourth element in that array.
One could further imagine that other code on the page may be predicated on
other properties of that return value, and fail when those are not true.

\lstset{numbers=left,xleftmargin=2em,frame=single,framexleftmargin=1.5em}
\begin{figure}[h]
\begin{lstlisting}[language=javascript]
var ps, p5;
ps = document.getElementsByTagName("p");
p5 = ps[4];
p5.setAttribute("style", "color: red");
alert("Success!");
\end{lstlisting}
\caption{Trivial \JS code example, changing the color of the text in a paragraph.}
\label{fig:trivial-js}
\end{figure}


\subsection{ES6 Proxy Configuration}

Our technique solves this problem through a specially constructed version of
the \texttt{Proxy} object.  The \texttt{Proxy} object can intercept operations
and optionally pass them along to another object.  Relevant to this work,
proxy objects also allow code to trap on general language-operations.  Proxies
can register generic handlers that fire when the proxy is called like a
function, indexed into like an array, has its properties accessed like an object,
and operated on in other ways.

We take advantage of the \texttt{Proxy} object's versatility in two ways.
First, we use it to prevent websites from accessing certain browser
features, without breaking existing code.  This use case is described in detail
in Subsection~\ref{cost-benefit:intercepting-js:proxy-general}.  And second, we use the \texttt{Proxy}
object to enforce policies on runtime created objects.  This use case is
described in further detail in Subsection~\ref{cost-benefit:intercepting-js:proxy-non-signletons}


\subsection{Proxy-Based Approach}
\label{cost-benefit:intercepting-js:proxy-general}

We first use the \texttt{Proxy} object to solve the problems described in
\ref{cost-benefit:intercepting-js:featremove}.  We create a specially configured a proxy object
that registers callback functions for \textit{all} possible \JS
operations, and having those callback functions return a reference to the
same proxy object.  We also handle cases where Web API properties and
functions return scalar values (instead of functions, arrays or higher order
objects), by programming the proxy to evaluate to \texttt{0}, empty string,
or \texttt{undefined}, depending on the context. Thus configured, the proxy
object can validly take on the semantics of any variable in any \JS program.

By replacing \texttt{getElementsByTagName} with our proxy, the code in
\ref{fig:trivial-js} will execute cleanly and the alert dialog on line
four will successfully appear.  On line one, the proxy object's function handler
will execute, resulting in the proxy being stored in the \texttt{ps} variable.
On line two, the proxy's \texttt{get} handler will execute, which also returns
the proxy, resulting in the proxy again being stored in \texttt{p5}.
Calling the \texttt{setAttribute} method causes the proxy object
to be called twice, first because of looking up the \texttt{setAttribute},
and then because of the result of that look up being called as a function.  The
end result is that the code executes correctly, but without
accessing any browser functionality beyond the core \JS language.

The complete proxy-based approach to graceful degradation can be found in
the source code of our browser extension\footnote{\ExtensionSourceUrl}.

Most state changing features in the browser are implemented through methods
which we block or record using the above described method.  This approach
does not work for the small number of features implemented through property sets.
For example, assigning a string to \texttt{document.location} redirects
the browser to the URL represented by the string.  When the property is
being set on a singleton object in the browser, as is the case with the
\texttt{document} object, we interpose on property sets by assigning a new
``set'' function for the property on the singleton using
\texttt{Object.defineProperty}.

\subsection{Sets on Non-Singleton Objects}
\label{cost-benefit:intercepting-js:proxy-non-signletons}
A different approach is needed for property sets on non-singleton objects.
Property sets cannot be imposed on through
altering an object's \texttt{Prototype}, and non-singleton objects
can not be modified with \texttt{Object.defineProperty} at
instrumentation time (since those objects do not yet exist). We instead interpose
on methods that yield non-singleton objects.

We modify these methods to return
\texttt{Proxy} objects that wrap these non-singleton objects, which we use
to control access to set these properties at run time. For example,
consider the below code example, using the \emph{Web Audio} API.

\lstset{numbers=left,xleftmargin=2em,frame=single,framexleftmargin=1.5em}
\begin{figure}[h]
\begin{lstlisting}[language=javascript]
var context = new window.AudioContext();
var gainNode = context.createGain();
gainNode.channelCount = 1;
\end{lstlisting}
\caption{Example of setting a property on a non-singleton object in the \WA.}
\label{fig:non-singleton-property-js}
\end{figure}


In this example, we are not able to interpose on the \texttt{gainNode.} \texttt{channelCount}
set, as the \texttt{gainNode} object does not exist when we modify the
\texttt{DOM}.  To address these cases, we further modify the
\texttt{AudioContext.property.createGain} to return a specially created
proxy object, instead of a \texttt{GainNode} object.  This, specially crafted
proxy object wraps the \texttt{GainNode} object, allowing us to interpose
on property sets.  Depending on the current policy, we either ignore the
property set or pass it along to the original \texttt{GainNode} object.


\subsection{Security Implications}
\label{cost-benefit:intercepting-js:proxy-security}

There are some code patterns where the proxy approach described here could have
a negative impact on security, such as when security sensitive computations are
done in the client, relying on functionality provided by the \WAPI, and where the
results of those calculations are critical inputs to other security sensitive
operations.  We expect that such cases are rare, given common web application
design practices. Even so, in the interest of safety, we whitelist the
\emph{WebCrypto} API by default, and discuss the security
and privacy tradeoffs here.

As discussed above, our proxy-based approach for interposing on \WAPI features
replaces references to the functionality being blocked with a new function that
returns the proxy object.  In most cases where the feature being replaced is
security relevant, this should not negatively effect the security of the system.  For
example, if the \texttt{encrypt} method from the \emph{Web Crypto} were
replaced with our proxy object, the function would not return an unencrypted
string, but instead the proxy object.  While this would break a system that
expected the cryptographic operation to be successful, it would
``fail-closed''; sensitive information would \textbf{not} be returned where
encrypted information was expected.

Conversely, if \texttt{getRandomValues} is used to generate a nonce,  the
returned proxy object would coerce to an empty string. While the security
repercussions of this silent failure could possibly be grave,
\cite{snyder2016browser} observed that the vast majority of calls to
\texttt{getRandomValues} on the open web could be considered privacy-invasive,
as they were used as part of the Google Analytics tracking library. Even so,
the potential harm to users from a silent failure is too great, resulting in
our decision to whitelist \emph{WebCrypto}. As our proposed contextual
protection extension can implement arbitrary policies, we look forward to
debate among experts and users as to what a sensible set of defaults should be
in this situation.


% Second, we note that common web application practices already emphasize
% not performing security sensitive operations in the browser, but instead to
% move them to a more trusted location in the system (i.e. the server).
% Applications following these practices would be not have their security impacted
% by this already-unlikely issue.

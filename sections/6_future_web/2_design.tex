\section{Design}
\label{future-web:design}

\CDF is an alternative system for creating modern, interactive websites, that
provides greater security and privacy guarantees than the current HTML-and-\JS
system.  The principal features in the design of \CDF are as follows.

\textbf{First}, \CDF prevents websites from running arbitrary \JS on the
client.  Instead, \CDF authors create interactive websites by composing
trusted, client-controlled implementations of interactive functionality using
an easily checked, declarative syntax.  \textbf{Second}, \CDF only uses a small
subset of browser features, allowing websites to access only the ``core'', or
most popular and frequently used parts of the \WAPI, when creating web sites.
\textbf{Third}, \CDF places stricter constraints on the kinds of web documents
than current HTML-and-\JS applications enforce, to restrict possible data
flows through the application.

\begin{table}[ht]
    \begin{tabular}{ l c c }
    \toprule
        Capability                      &   HTML + JS    & CDF \\
    \midrule
    Load static media from remote
        and local domains             &   \checkmark    & \checkmark  \\
    Load non-client controlled \JS  &   \checkmark    & -           \\
    Can express common web
        design idioms                 &   \checkmark    & \checkmark  \\
    Server control over
        HTTP \texttt{referer} and
        related privacy settings      &   \checkmark    & -           \\
    Client guarantees over
        HTTP \texttt{referer} and
        related privacy settings      &   -             & \checkmark  \\
    Read sensitive values from
        cookies, local storage, etc.  &   \checkmark    & -           \\
    Sub-page / AJAX requests and
        updates                       &   \checkmark    & \checkmark  \\
    Allow form submissions and
        AJAX updates to remote
        domains                       &   \checkmark    & -           \\
    HTML5 multimedia
        (\texttt{<audio>},
        \texttt{<video>})             &   \checkmark    & \checkmark  \\
    Supports common browser plugins
        (Flash, Java, Silverlight)  &   \checkmark    & -           \\
    Advanced \JS tools
        (WebGL, \texttt{<canvas>},
        ASM.js)                       &   \checkmark    & -           \\
    Client side storage
        (IndexDB, localStorage, etc)  &   \checkmark    & -           \\
    Offline Applications            &   \checkmark    & \checkmark  \\
    \bottomrule
    \end{tabular}
    \caption{Feature comparison between HTML and CDF document formats.}
    \label{table:cdf-html-comparison}
  \end{table}


\ref{table:cdf-html-comparison} provides a comparison of the capabilities
and guarantees made by current HTML-and-\JS based applications, contrasted
against \CDF documents.  The following subsections provide more detail
about each aspect of the design of \CDF.


\subsection{Trusted Feature Implementation}
\label{future-web:design:trusted-feature-implementation}
\CDF's main method for improving user security and privacy is by preventing
websites from executing arbitrary \JS on the client.  The current \JS-based
system for providing interactive websites is the cause of many web browser
security problems.  Browsers must trust that code will carry out some
non-malicious purpose when executing it, and that a given set of \JS
instructions will benefit the user (by, for example, setting up a website's
user interface elements) instead of harming the user (e.g. by fingerprinting
the user, accessing a browser feature with a known security flaw, or sending a
session token to a malicious destination).

Instead of attempting to difficult task of verifiying that \JS code is benign
before executing in, \CDF takes the simpler approach of not allowing
applications to provide their own \JS.  \CDF instead provides a set of trusted,
client-side implemented interactive primitives, which web authors can compose
into higher-level functionality with a declarative, easy to verify syntax.
\CDF authors can, for example, tie a \emph{mouse click} event to a
\emph{document attribute change} event, not by writing code directly, but
through the structure of the document.  \CDF clients include their own trusted
libraries that handle generating code and executing the relevant functionality
on the clients, without trusting code provided by the website.

The result is that \CDF documents are composed from functionality implemented
in trusted, client-controlled libraries.  These libraries are designed to
compose safely, and pages can only access them through a simple, declarative
syntax.  This is in contrast to the typical \JS based approach, where websites
can execute arbitrary code, and web browsers must judge if the resulting
behavior seems safe through heuristics like \gls{xss} filters and code origin
reputation systems.


\subsection{Feature Selection}
\label{future-web:design:feature-selection}
\CDF also protects user security and privacy by reducing the browser's attack
surface by preventing websites from accessing browser functionality that
is either rarely used, or predominantly used for non-user-serving purposes
(e.g. browser fingerprinting).

As demonstrated in chapters~\ref{measurement} and \ref{cost-benefit}, modern
web browsers implement a huge array of \WAPI features.  While some of this
functionality is closely related to the web's most-frequent purpose of delivering
interactive documents, other functionality is never used, used in only rare
niche situations, or used predominantly for malicious purposes.

\CDF uses these findings to improve user security and privacy by restricting
websites to only frequently-used, document-manipulation related \WAPI features.
By preventing websites from accessing features that are not generally
used for user serving interests (either because those features are
primary used for advertising and tracking, or because the features are rarely
used at all), \CDF brings web browsers into closer alignment with the security
principal of least privilege.  The attack surface exposed to websites is
dramatically reduced, with minimal impact to the user experience.


\subsection{Document Constraints}
\label{sec:design:doc-constraints}
Current \gls{html} applications include several other design aspects that make
them difficult to secure.  To name only a few such examples: \gls{html} and
\JS applications allow scripts to be loaded from remote locations from any
part of the \gls{html} document, enabling many \gls{xss} attacks.  \gls{html}
documents can contain complete sub-documents through the use of
\texttt{<iframe>} elements, enabling drive by downloads and related attacks.
And \gls{html} applications generally include a ``referer'' header when
requesting remote resources, enabling some forms of user tracking.

\CDF improves user security and privacy by tightly-controlling what kinds of
resources documents can fetch, and what information is sent during the fetch
request.  \CDF documents cannot include arbitrary code (either
inline or hosted remotely), include sub-documents, or send information
generated in the client directly to remote domains.

\section{Data Sources}
\label{measurement:data-sources}

This work relied on several prior-existing sets of data.  This section proceeds by
detailing how we determined which websites are popular and how often
they are visited, how we determined the current \JS-exposed feature set
of web browsers, what web standards those features belong to and when
they were introduced, and which browser extensions we used to distinguish
user-serving from non-user-serving feature use.


\subsection{Alexa Website Rankings}
\label{measurement:data-sources:website-popularity-rankings}
The Alexa rankings are a well known ordering of websites ranked by traffic.
Typically, research which uses Alexa relies on their ranked list of the
worldwide top one million sites. Alexa also provides other data about these sites.
In addition to a global ranking of each of these sites,
there are local rankings at country granularity, breakdowns of which
sub-sites (by fully qualified domain name) are most popular, and a breakdown by
page load and by unique visitor of how many monthly visitors each site gets.

We used the 10,000 top ranked sites from Alexa's list of the one-million
most popular sites, and which collectively represent approximately one third of
all web visits, as representative of the web in general.


\subsection{Web API Features}
\label{measurement:data-sources:method-web-standards}
As mentioned in Section~\ref{intro:terms}, this work uses the term \textbf{feature}
to denote a browser capability that is accessible through calling a
\JS call or setting a property on a \JS object.

We determined the set of \JS-exposed features by reviewing the WebIDL definitions
included in the \FF version \FFversion source
code. WebIDL is a language that defines the \JS features web browsers
provided to web authors.  In the case of \FF, these WebIDL files are
included in the browser source.

In the common case, Firefox's WebIDL files define a mapping between a
\JS accessible method or property and the C++ code that implements
the underlying functionality\footnote{In addition to mapping
\JS to C++ methods and structures, WebIDL can also define \JS
to \JS methods, as well as intermediate structures that are not
exposed to the browser.  In practice though, the primary role of WebIDL
in Firefox is to define a mapping between \JS API endpoints and
the underlying implementations, generally in C++.}. We examined each of the 757
WebIDL files in the \FF and extracted \numfeatures relevant methods and
properties implemented in the browser.


\subsection{Web API Standards}
Web standards are documents defining functionality that web browser vendors
should implement.  They are generally written and formalized by organizations
like the W3C, though occasionally standards organizations delegate
responsibility for writing standards to third parties, such as the Khronos
group who maintains the current \textit{WebGL} standard.  As mentioned in
Section~\ref{intro:terms}, we use the term \textbf{standard} to refer to these
sets of features, generally grouped by a common purpose.

There are also web standards that cover non-\JS aspects of the browser (such
as parsing rules, what tags and attributes can be used in HTML documents,
etc.). This work focuses only on web standards that define \JS exposed
functionality.

We identified \numstandards standards implemented in \FF.  We associated each
of these to a standards document.
We also found 65 API endpoints implemented in \FF that are not
found in any web standard document, which we associated with a catch-all
\textit{Non-Standard} categorization.

In the case of extremely large standards, we identify sub-standards,
which define a subset of related features intended to be used together.
For example, we treat the subsections of the \textit{HTML}~\cite{whatwg2018html}
standard that define the basic \textit{Canvas API}, or the \textit{WebSockets API},
as their own standards.

Because these sub-standards have their own coherent purpose,
it is meaningful to discuss them independently of their parent standards. Furthermore,
many have been implemented in browsers independent
of the parent standard (i.e. browser vendors added support for
``websockets'' long before they implemented the full ``HTML5''
standard).

Some features appear in multiple web standards.  For example, the
\texttt{Node.prototype.insertBefore} feature appears in the
\textit{Document Object Model (DOM) Level 1 Specification}~\cite{dom1w3c},
\textit{Document Object Model (DOM) Level 2 Core Specification}~\cite{dom2corew3c}
and \emph{Document Object Model (DOM) Level 3 Core Specification}~\cite{dom3corew3c}
standards.  In such cases, the feature is attributed to its earliest published
standard.


\subsection{Historical Firefox Builds}
We determined when features were implemented in Firefox by
examining the \numfirefoxes versions of \FF that have been released since 2004
and testing when each of the \numfeatures features first appeared.
We treat the release date of the earliest version of
\FF that a feature appears in as the feature's ``implementation date''.

Most standards do not have a single implementation date, since it could take
months or years for all features in a standard to be implemented in \FF.  We therefore
treat the introduction of a standard's currently most popular feature as the
standard's implementation date. For ties (especially relevant when no feature
in a standard is used), we default to the earliest feature available.

%This approach differs from, and is more accurate than, other popular approaches,
%such as the CanIUse website\cite{deveria2015caniuse}, which tracks when
%some standards and features became available in web browsers, but only for a
%small, curated subset of features and standards.


\subsection{Blocking Extensions}
\label{measurement:data-sources:data-extensions}
Finally, this work pulls from commercial and crowd-sourced browser extensions,
which are popularly used to modify the browser environment.

This work uses two such browser extensions, Ghostery and AdBlock Plus.  Ghostery is a
browser extension that allows users to increase their privacy online
by modifying the browser to not load resources or set cookies associated with
cross-domain passive tracking, as determined by the extension's maintainer, Ghostery, Inc..

This work also uses the AdBlock Plus browser extension, which modifies
the browser to not load resources the extension associates with
advertising, and to hide elements in the page that are advertising related.
AdBlock Plus draws from a crowdsourced list of rules and URLs to determine
if a resource is advertising-related.

This work uses the default configuration for each browser extension, including
the default rule sets for which elements and resources to block.
No changes were made to the configuration or implementation of either extension.
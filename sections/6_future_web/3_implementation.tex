\section{Implementation}
\label{future-web:implementation}

We implemented \CDF in two parts, first as a document specification, and second
as several additions to the browser's trusted base: a \emph{parser} that
converts \CDF documents into trusted \HTML and \JS, a \emph{HTTP proxy} that
converts \CDF documents for use in web browsers, and a set of \emph{trusted \JS
libraries} that run in the browser to implement the interactive aspects of \CDF
applications.

The described system was implemented to allow \CDF documents to be run in web
browsers today, with no additions or modifications needed to any recently
released browser.  The same design could be implemented by modifying a browser
to be able to parse and understand \CDF documents ``natively'', though at the
cost of a much greater engineering task.

We also adopted \gls{css} as is, to handle the presentation
of \CDF applications.  We did so to minimize the engineering effort needed
to implement the \CDF concept, and because of the relative lack of security
issues associated with \gls{css} compared to \JS.  While privacy issues
have been raised concerning recent \gls{css} features~\cite{crookedstylesheets},
such issues are beyond this scope of this work, other than to note
that similar sub-setting approaches could be implemented in future-\CDF-like
systems to address such attacks.

This section gives a high level explanation of one possible implementation
of \CDF~'s design.  Documentation for creating \CDF documents, including
type specifications, nesting rules, and the interactivity primitives
included in \CDF can be found in an open source implementation and accompanying
documentation\footnote{https://github.com/bitslab/cdf.}.

\subsection{Document Format}
\CDF uses \gls{json} strings to represent documents.  \CDF documents are
trees of typed objects.  Types in \CDF fall into one of four categories.

\begin{itemize}[itemsep=-2pt]
  \item \textbf{Elements.} The structure and text of the document.
  \item \textbf{Events.} New input from the network or the user.
  \item \textbf{Behaviors.} Descriptions of what should happen when an Event has
                   triggered.
  \item \textbf{Deltas.} Changes to be applied to the document.
\end{itemize}

Each type defines the configuration it can receive (e.g. the
URL that a \texttt{image} object can refer to), and the types it
accepts as children in the tree.  For example, \texttt{text} objects can be
children of \texttt{button} objects (to create labels on buttons), but
\texttt{button} objects cannot be children of \texttt{text} objects.
Since the types in \CDF are all well-defined, they can be strictly checked to
ensure they will have predictable effects when rendered in the client.

Some types accept configuration parameters (e.g. the class names to add to the
element when rendered in \gls{html}, or the local \gls{url} to post a form's
information to).  These configuration parameters are also strictly typed, and
are checked for safety and correctness before being rendered in the
client.

Types are designed to emphasize predictable information flow and user privacy.
For example, in \CDF \texttt{form} elements are only allowed to send information to
the origin domain, while in HTML applications, \texttt{<form>} elements
can be configured to send information to any domain.


\subsection{Trusted Base Additions}
We implemented the \CDF design through three additions to the current trusted
web browser trusted base.  These additions, in tandem, enforce the security and
privacy properties discussed in Section~\ref{future-web:design}.


\subsubsection{Parser}
The first addition \CDF makes to the browser's trusted base is a \CDF parser.
The role of the \CDF parser is to take strings and either identify them as
invalid \CDF documents, or to render an equivalent and safe \gls{html} and \JS
string that can be rendered in the browser. The parser also provides debugging
information as a convenience to \CDF authors.

If the parser is given a valid \CDF document, it converts it into a combination
of \HTML tags, escaped text, \texttt{<script>} tags referencing \JS libraries
that are part of the \CDF trusted base, and \texttt{<script>} tags containing
parameters to be passed to those trusted libraries.  Invalid documents
``fail closed'', and return an error code and no output.


\subsubsection{HTTP Proxy}
\label{future-web:implementation:proxy}
The second addition \CDF makes to the browser's trusted base is a \gls{http}
proxy that sits between the browser and the internet.  The proxy passes
requests from the browser to the destination server unchanged.  Once the server
responds, the proxy examines the response.  If the response appears to be a
\CDF document, the \gls{http} proxy extracts the body of the request and provides it
to the parser.  If the parser accepts the response as a valid \CDF document,
the proxy passes the parser-generated \HTML and \JS back to the client.  If the
parser rejects the server's response as invalid \CDF, the proxy instead passes
back an error message to the client, informing the user that the server
provided an invalid document.


\subsubsection{Client \JS Libraries}
\label{future-web:implementation:client-js}
The third addition to the browser's trusted base is in a small number of \JS
libraries (14) that implement the interactive elements of each page.  These
libraries handle all the client-side logic and functionality needed for all of
the event, behavior and delta types used in the system, plus some plumbing code
needed to route the parameters extracted by the parser to the correct library
implementations.

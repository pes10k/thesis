\section{Browser Hardening Extension}
\label{current-web:extension}

This section presents efforts at using the cost-benefit measurements from
Section~\ref{cost-benefit} to design and develop a browser extension that
allows users to control website access to \WAPI functionality.  The extension
has been released to the public and is being used by almost 1k users at the
time of this writing (555 Firefox
users~\footnote{\url{https://addons.mozilla.org/en-US/firefox/addon/webapi-manager/}}
and 417 Chrome
users~\footnote{\url{https://chrome.google.com/webstore/detail/webapi-manager/hojoljbhkebfjalcbnfmoiljfidcmcmj}}).
This section describes how the extension was designed, evaluated and
implemented before it was publicly released.
Section~\ref{current-web:extension-deployment} discusses lessons and findings
that were gained only once the extension was available for the public.

The source of the extension has been released publicly~\footnote{\ExtensionSourceUrl},
and the project continues to be refined with the help and input of other
developers and privacy activists.


\subsection{Implementation}
Our browser extension uses the same \WAS disabling technique described in
Section~\ref{cost-benefit:intercepting-js} to dynamically control the \WAPI
functionality exposed to websites.  The extension allows users to
apply hardened configurations that we designed (based on the findings
discussed in Section~\ref{cost-benefit} and detailed in
Section~\ref{current-web:extension:configurations}), or design and deploy
their own hardened configurations by selecting any permutation of the
measured \WASs to disable (along with several additional \WASs that were deployed
since the cost-benefit measurements were performed).

We emphasize the dynamic nature of the hardened browser configurations for
several reasons.  First, if a given standard was found to be vulnerable to new
attacks in the future, security sensitive users could update their hardened
configurations to remove it.  Likewise, if other features became more popular
or useful to users on the web, future hardened configurations could be updated
to allow those standards.  The extension enables users to define their own
cost-benefit balance in the security of their browser, rather than prescribing
a specific configuration.

Finally, the tool allows users to create per-origin attack-surface policies,
so that trusted sites can be granted access to more \JS-accessible
features and standards than unknown or untrusted websites.  Similar to, but
finer grained than, the origin based policies of tools like NoScript, this
approach allows users to better limit websites to the least privilege
needed to carry out the sites' desired functionality.


\subsection{Trade-offs and Limitations}
Deploying our approach as a browser extension entailed significant trade-offs.
On the benefit side, browser extensions are easy for users to install and update,
and our browser extension is compatible with current popular web browsers,
minimizing the amount of additional engineering work needed to get the approach
implemented and usable by security and privacy concerned users.  Additionally,
browser extensions are powerful enough to (in principal) successfully protect
users from most vulnerabilities that depend on accessing a \JS-exposed method
or data structure (of which there are many, as documented in
Section~\ref{cost-benefit:results:costs-cves}).  The WebExtension standard, which
standardizes an common interface for writing cross-browser extensions,
defines hooks that allow for disabling large portions of high-risk
functionality, which could protect users from not-yet-discovered bugs, in a way
that ad-hoc fixing of known bugs could not.

However, the choice to deploy as an extension also entails significant
limitations.  First, there are categories of browser exploits
that our extension-based approach cannot guard against.  For example, our
approach cannot provide protection against exploits that rely on \WAPI
functionality that is reachable through means other than \JS.
The extension would not provide protection against, for example,
exploits in the browser's CSS parser, TLS code, or image parsers (since the
attacker would not require \JS to access such code-paths).

Second, the choice to implement as a browser extension made our approach
vulnerable to a weakness common to all \gls{dom}-modifying browser extensions
privacy and security tools that was discovered in this work (discussed in
detail in Section~\ref{current-web:extension-deployment}).

Third, the extension approach does not have access to some
information that could be used to make more sophisticated decisions about
when to allow a website to access a feature.  An alternate approach that
modified the browser could use factors such as the state of the stack at call
time (e.g. distinguishing between first-and-third party calls to a \WAS), or
where a function was defined (e.g. whether a function was defined in \JS code
delivered over TLS from a trusted website).  Because such information is not
exposed through \JS, our extension is not able to take advantage of
such information.


\subsection{Usability Evaluation}
This section presents an evaluation of the usability of our \WAPI-standard
blocking extension.  The goal of this evaluation was to measure how the
extension both improved and harmed users' browsing experiences, to see if the
extension's cost (measured in ``number of sites that no longer function
correctly'') would be worth the extension's benefits (measured as ``security
risk reduction from blocking \WAPI standards'').

The extension allows users to develop custom, per-site configurations of which
\WAPI standards to block.  Since there were \NumStandards standards considered
in Section~\ref{cost-benefit}, there are \(2^{\NumStandards}\) possible
configurations of the extension, yielding an impossible number of
configurations to test for even a single site, let alone a large enough sample
of websites to represent the web as a whole.

Instead, we created and evaluated two plausible extension configurations, based
on the data described in Section~\ref{cost-benefit}, to represent users with
different privacy needs, and thus users willing to accept different
security/usability trade-offs.  The following subsections present an analysis
of the usability of these two selected configurations.


\subsubsection{Selecting Configurations}
\label{current-web:extension:configurations}
\begin{table}[th]
    \centering
    % \rowcolors{2}{gray!25}{white}
    \resizebox{10cm}{!}{%
      \begin{tabular}{ l | c c c }
        \toprule
          Standard &
          Conservative &
          Aggressive &
          Lite \\
        \midrule
          Beacon                                        & X & X & X \\
          DOM Parsing and Serialization                 & X & X & \\
          Fullscreen API                                & X & X & \\
          High Resolution Time Level 2                  & X & X & \\
          HTML: Web Sockets                             & X & X & \\
          HTML: Channel Messaging                       & X & X & \\
          HTML: Web Workers                             & X & X & \\
          Indexed Database API                          & X & X & \\
          Performance Timeline Level 2                  & X & X & \\
          Resource Timing                               & X & X & \\
          Scalable Vector Graphics 1.1                  & X & X & X \\
          UI Events Specification                       & X & X & \\
          Web Audio API                                 & X & X & X \\
          WebGL Specification                           & X & X & X \\
          Ambient Light Sensor API                      &   & X & X \\
          Battery Status API                            &   & X & X \\
          CSS Conditional Rules Module Level 3          &   & X & \\
          CSS Font Loading Module Level 3               &   & X & \\
          CSSOM View Module                             &   & X & \\
          DOM Level 2: Traversal and Range              &   & X & \\
          Encrypted Media Extensions                    &   & X & \\
          execCommand                                   &   & X & \\
          Fetch                                         &   & X & \\
          File API                                      &   & X & \\
          Gamepad                                       &   & X & \\
          Geolocation API Specification                 &   & X & X \\
          HTML: Broadcasting                            &   & X & \\
          HTML: Plugins                                 &   & X & \\
          HTML: History Interface                       &   & X & \\
          HTML: Web Storage                             &   & X & \\
          Media Capture and Streams                     &   & X & \\
          Media Source Extensions                       &   & X & \\
          Navigation Timing                             &   & X & \\
          Performance Timeline                          &   & X & \\
          Pointer Lock                                  &   & X & \\
          Proximity Events                              &   & X & \\
          Selection API                                 &   & X & \\
          The Screen Orientation API                    &   & X & \\
          Timing control for script-based animations    &   & X & \\
          URL                                           &   & X & \\
          User Timing Level 2                           &   & X & \\
          W3C DOM4                                      &   & X & \\
          Web Notifications                             &   & X & X \\
          WebRTC 1.0                                    &   & X & X \\
          WebVTT                                        &   &   & X \\
          Geometry Interfaces                           &   &   & X \\
          Vibration                                     &   &   & X \\
          WebVR                                         &   &   & X \\
          WebUSB                                        &   &   & X \\
          WebSpeech                                     &   &   & X \\
        \bottomrule
      \end{tabular}
    }
    \caption{Listing of which standards were disabled in the evaluated
    conservative and aggressive hardened browser configurations.}
    \label{table:browser-configs}
  \end{table}

To address the combinatorial impossibility of evaluating all possible extension
configurations on all sites, we created two configurations to represent
configurations that users might create.  We call these configurations the
\textbf{conservative} and \textbf{aggressive} configurations, each intending to
represent users with different privacy-functionality trade-offs.

\ref{table:browser-configs} lists the standards we blocked for the conservative
and aggressive hardened browser configurations.  Our \textbf{conservative}
configuration focuses on removing features that are infrequently needed by
websites to function, and would be fitting for users who desire more security
than is typical of a commodity web browser, and are tolerant of a slight loss
of functionality.  Our \textbf{aggressive} configuration focuses on removing
further attack surface from the browser, even when that necessitates breaking
more websites.  This configuration would fit highly security sensitive
environments, where users are willing to accept breaking a higher percentage of
websites in order to gain further security.

We selected these profiles based on the data discussed in
Section~\ref{measurement:results}, Section~\ref{cost-benefit:results}, and
prioritizing not affecting the functionality of the most popular sites on the
web.  We further chose to not restrict the \emph{Web Crypto} standard, to avoid
affecting the possibly-security-sensitive code.

We note that these are just two possible configurations, and that users
(or trusted curators, IT administrators, or other sources) could
use this method to find the security / usability trade-off that best fit their needs.


\subsubsection{Configuration Evaluation}
\label{current-web:extension:configuration-evaluations}
\begin{table}[t]
  \centering
  % \rowcolors{2}{gray!25}{white}
  \resizebox{\columnwidth}{!}{
    \begin{tabular}{ l | r r }
      \toprule
        Statistic &
        Conservative &
        Aggressive \\
      \midrule
        Standards blocked              & 15      & 45      \\
        Previous CVEs \#               & 89      & 123     \\
        Previous CVEs \%               & 52.0\%  & 71.9\%  \\
        LOC Removed \#                 & 37,848  & 53,518  \\
        LOC Removed \%                 & 50.00\% & 70.76\% \\
        \% Popular sites broken        & 7.14\%  & 15.71\% \\
        \% Less popular sites broken   & 3.87\%  & 11.61\% \\
      \bottomrule
    \end{tabular}
  }
  \caption{Cost and benefit statistics for the evaluated conservative and aggressive browser configurations.}
  \label{table:config-eval}
\end{table}

We evaluated the usability and the security gains the hardened browser
configurations provided.  \ref{table:config-eval} shows the results of this
evaluation.  As expected, blocking more standards resulted in a more secure
browser, but at some cost to usability (measured by the number of broken
sites).

Our evolution was carried out similarly to the per-standard measurement
technique described in
Section~\ref{cost-benefit:methodology:per-standard-benefit}.  First, we created
two sets of test sites, \textbf{popular} sites (the 200 most popular sites in
the \ATK that are in English and not pornographic) and \textbf{less popular
sites} (a random sampling of sites from the \ATK ranked 201 or lower).
This yielded 175 test sites in the popular category, and 155 in the less
popular category.

Second, we had two evaluators visit each of these 330 websites under three
browsing configurations, for 60 seconds each.  Our decision to use 60 seconds
per page is based on prior research~\cite{liu2010understanding} finding that
users on average spend under a minute per page.

These evaluators first visited each site in an unmodified
Firefox browser, to determine the author-intended functionality of the website.
Second, they visited in a Firefox browser in the above mentioned conservative
configuration.  Finally, they visited a third time in the aggressive hardened
configuration.

For the conservative and aggressive tests, the evaluators recorded how the
modified browser configurations affected each page, using the 1--3 scale
described in Section~\ref{cost-benefit:methodology:per-standard-benefit}.  Our
evaluators independently gave each site the same 1--3 ranking 97.6\% of the
time for popular sites, and 98.3\% of the time for less popular sites, giving
us a high degree of confidence in their evaluations.  The ``\% Popular sites
broken'' and ``\% Less popular sites broken'' rows in \ref{table:config-eval}
give the results of this measurement.

To further increase our confidence in the reported site-break rates, our
evaluators recorded, in text, what broken functionality they encountered.  We
then randomly sampled and checked these textual descriptions, to verify that
our evaluators were experiencing similar broken functionality.  The consistency
we observed through this sampling supports the internal validity of the
reported site break rates.

As \ref{table:config-eval} shows, the trade off between gained security and
lessened usability is non-linear.  The conservative configuration disabled code
paths associated with 52\% of previous CVEs, and removed 50\% of \gls{eloc},
while affecting the functionally of only 3.87\%-7.14\% of sites on the
internet.  Similarly, the aggressive configuration disabled 71.9\% of code
paths associated with previous CVEs and over 70\% of \gls{eloc}, while
affecting the usability of 11.61\%-15.71\% of the web.


\subsubsection{Usability Comparison}
\label{current-web:extension:usability-comparison}
\begin{table}[t]
    \centering
    % \rowcolors{2}{gray!25}{white}
    \resizebox{.5\textwidth}{!}{
      \begin{tabular}{ l | r r r }
        \toprule
          & \% Popular    & \% Less popular  & Sites tested \\
          & sites broken  & sites broken     & \\
        \midrule
          Conservative Profile & 7.14\%  & 3.87\%  & 330 \\
          Aggressive Profile   & 15.71\% & 11.61\% & 330 \\
          Tor Browser Bundle   & 16.28\% & 7.50\%  & 100 \\
          NoScript             & 40.86\% & 43.87\% & 330 \\
        \bottomrule
      \end{tabular}
    }
    \caption{Comparison of the useability of the feature-access-control imposing
    extension, compared against versus other popular browser security tools.}
    \label{table:usability-comparison}
  \end{table}


We compared the usability of our conservative and aggressive configurations
against \gls{tbb} and NoScript, to measure how the \WAPI-blocking approach
compared to other popular security and privacy tools.  We found that the
conservative configuration had the highest usability of all four tested tools,
and that the aggressive hardened configuration was roughly comparable to the
default configuration of the \gls{tbb}.  The results of this comparison are
given in \ref{table:usability-comparison}.

This comparison does not imply which method is the most secure.  The types of
security problems addressed by each of these approaches are largely intended to
solve different types of problems, and all three compose well (i.e., one could
use a cost-benefit method to determine which \WASs to enable \textit{and}
harden the build environment and route traffic through the Tor network
\textit{and} apply per-origin rules to script execution). However, since
\gls{tbb} and NoScript are widely used security tools, comparing against them
gives a good baseline for usability for security conscious users.

We tested the usability using the same technique we used for the conservative
and aggressive browser configurations, described in Section
\ref{current-web:extension:configurations}. The same two evaluators visited the
same 175 popular and 155 less popular sites, but compared the page in
an unmodified Firefox browser with the default configuration of the NoScript
extension.

The same comparison was carried out for default \FF against
the default configuration of the \gls{tbb}\footnote{Smaller sample sizes were
used when evaluating \gls{tbb} because of time constraints, not for fundamental
methodological reasons.}.  The evaluators again reported very similar scores in
their evaluation, reaching the same score 99.75\% of the time when evaluating
NoScript and 90.35\% when evaluating the Tor Browser.  We expect the lower
agreement score for the \gls{tbb} is a result of our evaluators being routed
differently through the Tor network, and receiving different versions of the
website based on the location of their exit nodes.\footnote{We chose to
\emph{not} assign the Tor exit node to a fixed location during this evaluation
to accurately recreate the experience of using the default configuration of the
\gls{tbb}.}

As \ref{table:usability-comparison} shows, the usability of our conservative
and aggressive configurations is as good as or better than other popularly used
browser security tools.  This suggests that, while our \WASs blocking
approach has some affect on usability, it is a cost security-sensitive users
would accept.


\subsubsection{Allowing Features for Trusted Applications}
We further evaluated our approach by attempting to use several popular,
complex \JS applications in a browser in the \textbf{aggressive} hardened
configuration.  We then created application-specific configurations to allow
these applications to run, but with access to only the minimal set of
features needed for functionality.

This process of creating site-specific feature configurations
is roughly analogous to granting trusted applications additional
capabilities (in the context of a permissions based system), or allowing trusted
domains to run \JS code (similar to how NoScript functions).

We built these application specific configurations using a tool-assisted, trial
and error process: first, we visited the application with the browser extension
in ``logging'' mode, which caused the extension to log blocked functionality.
Next, when we encountered a part of the web application that did not function
correctly, we reviewed the extension's log to see what blocked functionality
seemed related to the error.  We then iteratively enabled the related blocked
standards and revisited the application, to see if enabling the standard
allowed the app to function correctly.  We repeated the above steps until the
app worked as desired.

This process is would be beyond what typical web users would be
capable of, or interested in, doing.  Users who were interested in improving the
security of their browser, but not interested in creating hardened app
configurations themselves, could subscribe to trusted, expert curated policies,
similar to how users of AdBlock Plus receive community created rules from
EasyList.

For the first example application-specific configuration, we watched videos on
YouTube, by first searching for videos on the site's homepage, clicking on a
video to watch, watching the video on its specific page, and then expanding the
video's display to full-screen.  This required enabling three standards
that are blocked in our \textbf{aggressive} configuration: the \textit{File
API} standard~\footnote{YouTube uses methods defined in this standard to create
URL strings referring to media on the page.}, the \textit{Media Source
Extensions} standard~\footnote{YouTube uses the
\texttt{HTMLVideoElement.prototype.getVideoPlaybackQuality} method from this
standard to calibrate video quality based on bandwidth.}, and the
\textit{Fullscreen API} standard. Once we enabled these three standards,
we were able to search for and watch videos on the site, while still
having 39 other standards disabled.

Second, we used the Google Drive application to write and save a text document,
formatting the text using the formatting features provided by the website
(creating bulleted lists, altering justifications, changing fonts and text
sizes, embedding links, etc.).  Doing so required enabling two standards that
are by default blocked in our \textbf{aggressive} configuration: the
\textit{HTML: Web Storage} standard~\footnote{Google Drive uses functionality
from this standard to track user state between pages.} and the \textit{UI
Events} standard~\footnote{Google Drive uses this standard for finer-grained
detection of where the mouse cursor is clicking in the application's
interface.}.  Allowing Google Docs to access these two additional standards,
but leaving the other 40 standards disabled, allowed us to create rich text
documents without any user-noticeable affect in site functionality.

Third and finally, we used the Google Maps application to map a route between
Chicago and New York.  We did so by first searching for ``Chicago, IL'',
allowing the map to zoom in on the city, clicking the ``Directions'' button,
searching for ``New York, NY'', and then selecting the ``driving directions''
option.  Once we enabled the \textit{HTML: Channel Messaging}
standard~\footnote{Which Google Maps uses to enable communication between
different sub-parts of the application.} we were able to use the site as
normal.

\summary
Over the last two decades, the web has grown from a system for delivering
static documents, to the world's most popular application platform.  As the
web has become more popular and successful, browser vendors have added more
and more functionality into the web platform.
While some of this functionality has proven very useful and allowed site
authors to create applications that users enjoy, a large subsets of functionality
in the browser goes largely unused.  Another sizable subset of functionality
has been leveraged by malicious parties to harm browser users.

This dissertation presents an effort to improving web privacy and security
by applying a cost-benefit analysis to the \WAPI, as its implemented in
popular web browsers.  The goal of the work is to apply the principal of
``least privilege'' to the web, and restrict websites to functionality
they need to carry out user-serving ends.  The work pursues that
end through a novel method of measuring the costs and benefits associated with
each standard in the \WAPI, identifying different high-benefit and low-risk
subsets of the \WAPI, and evaluating a variety of approaches for
restricting websites to these safer subsets.

This dissertation covers four distinct research efforts, each of which contribute
to the overall goal of improving privacy and security on the web.  First,
this dissertation describes an automated technique for measuring \WAPI use on the web by
instrumenting the DOM in a commodity web browser, automating the browser
to interact with websites in a manner that elicits most of the same
feature use as human users encounter, and recording what functionality is
triggered during this execution.  This section also presents the results of
applying this automated recording methodology to the entire Alexa 10k, both
in default browser configurations, and with popular blocking extensions installed.

Second, this work presents a systematic measurement of the costs and benefits of
each standard in the \WAPI.  This work models a standard's benefit
as the percentage of sites on the web that require the standard to carry out
their core, user-serving functionality, and models a standard's cost as
the security risk the standard poses to users (measured as the number of
recent vulnerabilities relating to the standard's implementation, the additional
complexity the standard's implementation brings to the browser code base, and
the number of papers in recent top security conferences and journals that
leverage the standard).

Third, this work presents a method to apply these cost-benefit measurements can
to the web as it exists today, to try and improve user's privacy and security.
This technique entails the creation of a \WAPI-blocking browser extension,
that restricts which features current websites are able to access.  This section
also presents findings from making this tool available to general web users.

Finally, this work describes an alternate system for designing
web applications that provides client-enforced privacy and security guarantees.
The design of this system builds on the previously discussed per-standard
cost-benefit methodology to determine which \WAPI features sites generally need.

Each of these work support the overarching finding that privacy
and security on the web can be improved with only a small cost to the user
experience. In contrast to the the current practice of giving every site access
to every feature in the browser (with only minor exceptions), this work presents
a data driven approach to restricting websites to a subset of safer, user-serving
functionality.  This dissertation further shows that the privacy and security
benefits of enforcing this ``least privilege'' approach to the \WAPI would
be meaningful, and real world deployment of these techniques shows that at
least some web users find the approach useful in protecting their privacy and
security.

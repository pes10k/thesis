\section{Measurement Methodology}
\label{measurement:methodology}

To understand browser feature use on the open web, we conducted a survey of the
Alexa 10k, visiting each site ten times and recording which browser features
were used during each visit.  We visited each site five times with an
unmodified browsing environment, and five times with popular tracking-blocking
and advertising-blocking extensions installed.  This Section describes the
goals of this survey, followed by the browser was instrumented to determine
which features were used on a site, and then concludes with how we used
our instrumented browser to measure feature use across the entire Alexa 10k.


\subsection{Goals}
The goal of this automated survey was to determine which browser features
are used on the web as it is commonly experienced by users.  This required us
to take a broad-yet-representative sample of the web, and to exhaustively
determine the features used by those sites.

To do so, we built a browser extension to measure which features are used when
a user interacts with a website.  We then chose a representative sample of the
web to visit.  Finally, we developed a method for interacting with these sites
in an automated fashion to elicit the same functionality that a human web user
would experience.  Each of these steps is described in detail in the proceeding
subsections.

This automated approach only attempts to measure the ``open web'', or the
subset of webpage functionality that a user encounters \textit{without} logging
into a website.  Users may encounter different types of functionality when
interacting with websites they have created accounts for and established
relationships with, but we did not attempt to gather such such measurements in
this work.  As a result, care should be taken before generalizing the following
results to browsing in general.


\subsection{Measuring Extension}
\label{measurement:methodology:measureextension}

We instrumented a recent version of \FF (version \FFversion) with a custom
browser extension that records each time a \JS feature is used.  The extension
achieves this by injecting \JS into each page after the browser has created the
\gls{dom} for that page, but before the page's content has been loaded. By
injecting our instrumenting \JS into the browser before the page's content has
been fetched and rendered, we can modify the methods and properties in the
\WAPI before the visited page's code is able to execute.

The injected \JS modifies the page's environment count when an instrumented
method is called or that an instrumented property is written to.  How the
extension measures these method calls and property writes is detailed in the
following two subsections.


\subsubsection{Measuring Method Calls}
The browser extension counted method invocations by overwriting the method on
the containing object's prototype.  This approach allowed us to shim in
logging functionality for each method call, and then call the original method
to preserve the original functionality.  We replace each reference to each
instrumented method in the \WAPI with an extension managed, instrumented
method.

Our method used \JS closures to ensure that web pages were not able to bypass
the our instrumented methods by looking up--or otherwise directly
accessing--the original versions of each method.


\subsubsection{Measuring Property Writes}
Properties were more difficult to instrument.  \JS provides no clean way to
intercept when a property has been set or read on a client script-created object,
or on an object created after the instrumenting code has finished executing, with
out affecting the execution of the page.  However, through the use of the
non-standard \texttt{Object.watch()}~\cite{mozillaobjectwatch} method in \FF,
we were able to capture when pages set properties on the singleton objects in
the browser (e.g. \texttt{window}, \texttt{window.document},
\texttt{window.navigator}).  The \texttt{Object.watch()} method allowed
the extension to capture and count all writes to properties on singleton
objects in the \WAPI.

There are a small number of features in the \gls{dom} where we were not able to
intercept property writes, and were therefore unable to count how frequently
these features were used.  These features, found primarily in older standards,
are properties where writes trigger side effects on the page.  The most
significant examples of such properties are \texttt{document.location} (where
writing to the property can trigger page redirection) and
\texttt{Element.prototype.innerHTML} (where writing to the property causes the
subtree in the document to be replaced).  The implementation of these features
in \FF make them unmeasurable using our technique.  They are noted here as a
small but significant limitation of our measurement technique.


\subsubsection{Other Browser Features}
Web standards define other features in the browser too, such as browser events
and \gls{css} layout rules, selectors, and instructions.  Our extension-based
approach did not allow us to measure the use of these features, and so counts
of their use are not included in this work.

In the case of standard defined events (e.g. \texttt{onload},
\texttt{onmouseover}, \texttt{onhover}) the extension could have
captured some event registrations through a combination of watching for
event registrations with \texttt{addEventListener} method calls
and watching for property-sets to singleton objects.  However, we would not
have been able to capture event registrations using the legacy \texttt{DOM0}
method of event registration (e.g. assigning a function to an object's
\texttt{onclick} property to handle click events) on non-singleton objects.
Since we would only have been able to see a subset of event registrations,
we decided to omit events completely from this work.

Similarly, this work does not consider non-\JS exposed functionality defined in
the browser, such as \gls{css} selectors and rules.  While interesting, this
work focuses solely on functionality that the browser allows websites to access
though \JS.


\subsection{Eliciting Site Functionality}
\label{measurement:methodology:eliciting-site-functionality}
We then measured which browser features were used on the most popular 10k
websites with our feature-detecting extension.  The following subsections
describe how we simulated human interaction with web pages to measure feature
use, first with the browser in its default state, and again with the browser
modified with popular advertising and tracking blocking extensions.


\subsubsection{Default Case}
\label{measurement:methodology:default-case-measurements}
To understand which features are used in a site's execution, we installed the
instrumenting extension described in
Section~\ref{measurement:methodology:measureextension} and visited sites from
the Alexa 10k, with the goal of exercising as much of the functionality used on
the page as possible.  While some \JS features of a site are automatically
activated on the home page (e.g. advertisements and analytics), many code paths
will only be used as a result of user interaction, either within the page or by
navigating to different areas of the site. Thus, an accurate measurement
of feature use requires interacting with an crawling each site.  This
sub-section describes the strategy used for crawling and interacting with sites.

In order to trigger as many browser features as possible on a website, we used
a common site testing methodology called ``monkey testing''.  Monkey testing
refers to instrumenting a page to click, touch, scroll, and enter text on
random elements or locations on the page.  To accomplish this, we used a
modified version of gremlins.js~\cite{zaninotto2016gremlins}, a library built
for monkey testing front-end website interfaces.  We modified the gremlins.js
library to distinguish between when the gremlins.js script used a
feature, and when the visted site used a feature.  The former feature
use was omitted from further consideration.

Each measurement started by visiting the site's home page and allowing
the monkey testing to run for 30 seconds.  Because the randomness of monkey
testing could cause navigation to other domains, we intercepted and prevented
any interactions which might navigate to a different page.  For
navigations that would have been to the local domain, we noted which URLs the
browser would have visited in the absence of the interception.

We then executed a breadth first search of the site using the
URLs that would have been visited by the actions of the monkey
testing.  We selected 3 of these URLs that were on the same domain (or
related domain, as determined by the Alexa data), and visited each, repeating
the same 30 second monkey testing procedure and recording all used features.
From each of these 3 sites, we then visited three more pages for 30 seconds,
which resulted in a total of 13 pages interacted with for a total of 390 seconds
per site.

If more than three links were clicked during any stage of the monkey testing
process, we selected which URLs to visit by giving preference to URLs where the
path structure URL had not been previously visited. In contrast to traditional
interface fuzzing techniques, which have as a goal finding unintended or
malicious functionality~\cite{amalfitano2012using,liu2014decaf}, we aimed to
find just the features site visitors would use. Selecting URLs with different
path-segments was a heuristic based approach to visit as many types of pages on
the site as possible, with the goal of capturing all of the functionality on
the site that a user would encounter.  This work discusses the robustness and
validity of this strategy in Section~\ref{measurement:validity}.


\subsubsection{Blocking Extension Measurements}
\label{measurement:methodology:blocking-case-measurements} We then repeated the
same measurement technique with ad-blocking and tracking-blocking extensions in
place (AdBlock Plus and Ghostery, respectively), to generate a second,
`blocking', set of measurements. We treated these blocking extensions as
representative of the types of modifications users make to customize their
browsing experience. While a so-modified version of a site no longer represents
its author's intended representation (and may in fact break the site), the
popularity of these content blocking extensions shows that this blocking case
is a common valid alternative experience of a website.


\subsubsection{Automated Crawl}
\label{measurement:methodology:automated-crawl}

\begin{table}[ht]
  \centering
  \begin{tabular}{ l r }
    \toprule
      Domains measured                    &  9,733 \\
      Total website interaction time      &  480 days\footnote{estimated based on the number of pages visited and 30 seconds of page interaction per visit.} \\
      Web pages visited                   &  2,240,484    \\
      Feature invocations recorded        &  21,511,926,733 \\
    \bottomrule
  \end{tabular}
  \caption{Amount of data gathered regarding \JS feature use on the Alexa 10k.}
  \label{fig:results-vanity-stats}
\end{table}


For each site in the Alexa 10k, we repeated the above procedure ten times to
attempt to measured all features used on the page: five times in the default
case, and then five times in the blocking case.  The crawl took two days, using
64 simultaneous \FF instances executing on 4 machines.

Section~\ref{measurement:validity} discusses why five crawls per site, per
condition, were sufficient to induce all relevant functionality.
\ref{fig:results-vanity-stats} presents some high level figures of this
automated crawl.  For 267 domains, we were unable to measure feature usage for
a variety of reasons, including non-responsive domains and sites that contained
syntax errors in their \JS code that prevented execution.

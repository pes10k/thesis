\chapter{Measuring Web API Use}
\section{Introduction}
The web is the world's largest open application platform.  While initially
developed for simple document delivery, it has grown to become the
most popular way of delivering applications to users.  Along with this growth
in popularity has been a growth in complexity, as the web has
picked up more capabilities over time.

This growth in complexity has been guided by both browser vendors and web
standards.  Many of these new web capabilities are provided through new
\JS APIs (referred to in this paper as \textbf{features}).  These
capabilities are organized into collections of related
features which are published as standards documents (in this paper, we
refer to these collections of APIs as \textbf{standards}).

To maximize compatibility between websites and web browsers,
browser vendors rarely remove features from browsers.  Browser vendors
aim to provide website authors with new features without breaking
sites that rely on older browser features.  The result is an ever growing
set of features in the browser.

Many web browser features have been controversial and even actively opposed
by privacy and free software activists for imposing significant costs
on users, in the form of information leakage or loss of control.
The \emph{WebRTC}~\cite{webrtcw3c} standard has been
criticized for revealing users' IP addresses~\cite{webrtcprivacy2015}, and
protestors have literally taken to the streets~\cite{emeprotests2016} to oppose the
\emph{Encrypted Media Extensions}~\cite{eme} standard. This standard aims to
give content owners much more control over how their content is experienced
within the browser. Such features could be used to prevent users from exerting
control over their browsing experience.

Similarly, while some aspects of web complexity are understood (such as the
number of resources web sites request~\cite{butkiewicz2011understanding}),
other aspects of complexity are not, such as how
much of the available functionality in the browser gets used, by which
sites, how often, and for what purposes.  Other related questions include
whether recently introduced features are as popular as older features, whether
popular websites use different features than less popular sites, and how the use
of popular extensions, like those that block advertisements and
online tracking, impact which browser features are used.

This paper answers these questions by examining the use of browser features on
the web.  We measure which browser features are frequently used by site authors,
and which browser features are rarely used, by examining the \JS feature
usage of the ten thousand most popular sites on the web.  We find, for example, that
50\% of the JavaScript provided features in the web browser are never
used by the ten thousand most popular websites.

We additionally measure the browser feature use in the presence of popular ad and
tracking blocking extensions, to determine how they effect browser feature use.
We find that installing
advertising and tracking blocking extensions not only reduces
the amount of \JS users execute when browsing the web, but changes the kinds of
features browsers execute. We identify a set of browser
features (approximately 10\%) that are used by websites, but which ad and
tracking blockers prevent from executing more than 90\% of the time.
Similarly, we find that over 83\% of features available in the browser are
executed on less than 1\% of websites in the presence of these popular extensions.

We have published data described in this work.  This data includes the JavaScript
feature usage of the Alexa 10k in both a default browser configuration and with ad
and tracking blocking extensions in place, as well as the mappings of JavaScript
features to standards documents.  The database with these measurements, along
with documentation describing the database's schema, is available
online~\cite{snyderp2016webapidata}.

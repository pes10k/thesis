\section{Conclusions}
\label{measurement:conclusions}

This chapter presents an automated technique for measuring what parts of the
\WAPI is used in typical low-trust, non-authenticated browsing scenarios, both
when using a stock browser, and when using a browser with popular ad and
tracking blocking extensions installed.  This chapter also presents the results
of applying the automated measurement technique to the \ATK.

The most significant results of this work, as it relates to this
dissertation's overarching goal of improving privacy and security on the web,
are two related insights.  First, this work shows there are significant
portions of the \WAPI that are not used when browsing non-trusted websites
(over 50\%, as measured by \WAPI features).  This strongly suggests that there
is little benefit for browser users in enabling these features, at least until
users have authenticated with the site, or otherwise established some trust.

Second, this work also documents that there are other parts of the \WAPI
that websites often want to use, but which are blocked by advertising
and tracking extensions.  This indicates that some \WAPI functionality
is primary used for purposes that the users of those web browsers do not
approve of.  These frequently-blocked features also seem to provide little
benefit to browser users, at least in the case of non-trust, unauthenticated
browsing scenarios.

Chapter~\ref{cost-benefit} builds on these findings by measuring the costs
and benefits each standard in the \WAPI carries with it.
Chapters~\ref{current-web} and \ref{future-web} then use these findings to
build tools and systems to better protect the privacy and security of web
users.

\section{Real-World Extension Deployment}
\label{current-web:extension-deployment}

The previous section described the design of the \WAPI blocking extension, and
how that design was based on the cost-benefit measurements from
Chapter~\ref{cost-benefit}.  This section describes discoveries that were
made once the extension was released to the public, and further
developed with the help of other privacy and security focused developers.


\subsection{Discovered Vulnerability in WebExtension Implementations}
While working on an issue that was reported against the blocking extension,
we discovered a security-related vulnerability in the WebExtension implementations
in Firefox and Chrome.  This issue allows determined websites to access browser
functionality blocked by our extension, and a comprehensive defense against
it required modifications to our approach that breaks more websites than
we originally accounted for.  This same WebExtension weakness also affects
other security and privacy improving extensions (for example PrivacyBadger).

This section provides some background information on what the WebExtension
standard is, describes the vulnerability in WebExtension implementations that
prevents extension like ours from making strong security and privacy guarantees
(without making further page-breaking DOM modifications), and describes how
authors of privacy and security enhancing extensions can move forward.


\subsubsection{The WebExtension Standard}
The \textit{WebExtension}~\cite{webext2018standard} standard is a standard
for writing browser-modifying extensions, in a way that is compatible
with all major browsers.  All of the major browser vendors have pledged
support for the standard, but Firefox and Chrome have the most complete
and popular implementations.  Though the standard is largely based on
the original Chrome Extension API, the standard is now managed by the \gls{w3c}.

The WebExtension standard allows authors to modify the browsing environment
in many ways.  Most relevant to our extension is the ability to inject
\JS into frames, before the frames's own \JS has executed.  This allows the
the extension to modify the environment the website executes in.  Our extension
uses this technique to add access controls to \WAPI features, using a method
described in greater detail in Section~\ref{cost-benefit:intercepting-js}.
Many other privacy and security enhancing extensions (e.g. PrivacyBadger) use
this same technique to prevent pages from accessing parts of the \WAPI, to
protect users (for example to prevent the page from fingerprinting users).


\subsubsection{How the Vulnerability Works}
The vulnerability we discovered allows pages to access unmodified versions
of the browser environment, even after the extension's \JS has run.  The
vulnerability works by exploiting how frames interact in Chrome and Firefox,
and when the WebExtension's script injection hooks run.


\subsubsubsection{Frames in HTML Documents}
The term \textbf{frame} refers to an execution environment in the browser.
Most commonly these are pages loaded by the browser, with each browser
tab depicting a single frame.  For example, loading \texttt{example.org} in
a browser tab will create a single frame, showing the \gls{html} document
returned by \texttt{example.org}.  Opening a second tab and loading
\texttt{other-example.org} will likewise create a new frame, this time
depicting the HTML returned from \texttt{other-example.org}.

Importantly, each frame gets its own \gls{dom} instance, each with its own
version of each \WAPI feature.  \WAPI feature is implemented by a \JS object,
with functions represented by objects inheriting from \texttt{Function.prototype}.
Changing a function in one frame will have no effect on similar objects
in other frames.  Put differently, deleting the `Document.prototype.getElementById'
object in the \texttt{example.org} frame will prevent other code executing in
the \texttt{example.org} frame from querying for elements by their \texttt{id},
but that change will be invisible to code running in the \texttt{other-example.org}
frame.

Generally frames are not able to communicate with each other.  The above
mentioned \texttt{example.org} frame has no way to access the \gls{dom} of the
\texttt{other-example.org} frame.  There are cases where frames are access
resources from other frames though.  A common example of such a case is if
a frame creates a child frame that is rendering content from the same domain
as the the parent frame.  In such cases, \JS running in the parent frame
can access the \gls{dom} of the child frame through the child frame's
\texttt{contentWindow} property.

% \begin{figure}[t!]
  \centering
  \lstset{numbers=left,xleftmargin=2em,frame=single,framexleftmargin=1.5em}
  \begin{lstlisting}[language=javascript]
  // Overwrite the "toBlob" method on canvas objects in
  // this frame.
  window.HTMLCanvasElement.prototype.toBlob = () => "blocked";

  let canvasElm = document.getElementByTagName("canvas");
  canvasElm.toBlob(); // Returns "blocked".

  let childIFrame = document.createElement("iframe");
  childIFrame.location = "//" + document.location.hostName;
  document.body.appendChild(childIFrame);

  // Extract the "toBlob" method from the DOM of the child frame.
  let childWindow = childIFrame.childWindow;
  let childToBlob = childWindow.HTMLCanvasElement.prototype.toBlob;

  // Returns the contents of the canvas element as a base64 string,
  // the functionality we tried to block above.
  childToBlob.call(canvasElm);
  \end{lstlisting}
  \caption{\JS code demonstrating the discovered WebExtension vulnerability.}
  \label{fig:webex-vulnerability-js}
\end{figure}



\subsubsubsection{Injecting Script from a WebExtension}
The WebExtension standard provides several opportunities for extensions to
inject \JS into frames.  Relevant to this vulnerability is that scripts can
register to run at \texttt{loading} time, which corresponds to a point when
the \gls{dom} for the frame has been prepare, but no page contents have
executed yet. The WebExtension standard defines this as the
\texttt{document\_start} hook.

Because the WebExtension standard guarantees that \texttt{document\_start} will
run before any other content is executed in the frame, many extensions use
this opportunity to inject script into a frame, to modify the DOM of the frame
to achieve some security or privacy improvement.  Our extension uses this
hook to inject \JS that interposes on the features in the standards the user
has blocked.  Since the WebExtension standard guarantees that this will happens
before any page content runs, the extension can be sure that will only see
the \gls{dom} as its been modified by the extension, can that page code
will not be able to access the original, non-interposed-on versions of the
blocked features.

Many other extensions do something similar, and use the \texttt{document\_start}
hook to achieve security or privacy goals. PrivacyBadger, for example, uses
this opportunity to replace \WAPI function associated with canvas fingerprinting
with new functions that record the fingerprinting attempt.


\subsubsubsection{Exploiting the Vulnerability}
The vulnerability arises because of how the above two issues interact.  One
might expect that because the \texttt{document\_start} hook runs before
page content, then page content will only be able to access the \gls{dom}
after its been modified.  However, this is not the case.  While frames are unable
to access their own \gls{dom} before its modified by extensions, frames can
access the \gls{dom} of child frames before the \texttt{document\_start}
hook fires in the child frame.

This occurs because, while the \texttt{document\_start} hook is guaranteed to
fire before a frame's content runs, there is a period of time between
when the child frame's \gls{dom} is created, and the \texttt{document\_start}
hook fires.  If a parent frame accesses the child frame's \texttt{contentWindow}
property during this interim period, the parent frame will be able to access
the child frame's \gls{dom} before its modified.  The parent frame
can then extract references to blocked functionality and execute them
in the context of the parent frame.  The end result of this is pages are able
to bypass extensions that attempt to restrict access to \WAPI features.


\subsubsection{Addressing the Vulnerability}
We've tried to address this vulnerability in the WebExtension standard in
several ways.  First, we filed bugs with both
Firefox~\footnote{\url{https://bugzilla.mozilla.org/show_bug.cgi?id=1424176}} and
Chromium~\footnote{\url{https://bugs.chromium.org/p/chromium/issues/detail?id=793217,
notifying them of the issue.  Firefox has acknowledged the issue but so far
the issue has not been address.  The issue is still waiting for triage in
Chromium's system.

Second, we notified other similar privacy and security minded
projects that use the same WebExtension approach (i.e. Brave and PrivacyBadger)
of the issue.  In both cases, the developers acknowledged the issue, but are
waiting on action from the browser vendors before pursing the matter further.




\subsection{Feature-level Granularity}
\subsection{New Standard Adoption}
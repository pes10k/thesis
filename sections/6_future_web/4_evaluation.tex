\section{Evaluation}
\label{future-web:evaluation}

We tested the usability and expressiveness of \CDF by implementing several
popular types of web applications in the the system.  We selected
these applications (a blog modeled on \url{http://www.vogue.com/},
an online-banking site based on \url{https://www.bankofamerica.com/}, a social
media site modeled on \url{https://twitter.com/}, and a collaborative web application similar to
HotCRP~\cite{hotCRPhomepage}) to represent the range of sites that web users
commonly interact with.  In each case we were able to replicate the
user-facing functionality of each page.

This section evaluates the security benefits of \CDF's approach for describing
interactive websites.  For each issue, we briefly
describe a vulnerability common in current web applications, and then
describe how \CDF improves the situation.


\subsection{Cross-Site Scripting}
Cross-Site scripting (XSS) refers to when attackers are able to inject \JS code
into an HTML document, so that the code is executed by all site visitors, trusted
as if the code came from the site author.  The technique is used for many
malicious purposes, including extracting session tokens from the client or
redirecting the user to a domain the attacker controls.

\CDF protects the client from XSS attacks.  First and most
significantly it removes the ability for a document to describe any kind of
\JS code directly.  Instead of arbitrary code, \CDF documents can only describe
a composition of trusted, safe types.  While a malicious attacker could
possibly corrupt a target server to present visitors a different
composition of types than the application author intended, \CDF's types
constrain the functionality that can be described to only safe activities.  \CDF
does not include, for example, anyway to access cookie values or
redirect the client in \JS to another location, common goals of XSS attacks.


\subsection{Page Alteration / Defacement}
When application authors do not adequately sanitize or validate the inputs
users provide to their site, they risk giving users the ability to deface, or
otherwise unexpectedly alter, the presentation of their website.  This
can lead to a blurring of the line between a message provided by the page
author (which may be trusted by site visitors) and other web site visitors
(which may be untrusted).  This may happen when a naive application
author concatenates the user's input, represented
as a string, into a larger string the author is using for the returned
content.

\CDF's type system makes this kind of error more difficult to make.
The \CDF author must construct pages as trees of instances of types.
Overall page structure and styling cannot be modified from within
an individual child node in the document tree. In cases
where page authors are taking inputs from users, and anticipate that input
to be in the form of an unstructured piece of text (such as a comment
on an article), the page author would do so by setting the user's input string
as the content of a \texttt{text} element.  When \CDF then
renders the document to send to visitors of the site, the \CDF parser escapes all
content in \texttt{text} instances to ensure that the content cannot change
the structure of the page (such as by including \JS code or altering the balance
of tags on the page).

While \CDF does not make this kind of attack impossible (it is possible
to conceive of ways that a sufficiently naive page author would construct a
vulnerable document), it makes the attack much more difficult to pull off.
Instead of becoming relatively easy for page authors to be affected by this
kind of attack, \CDF instead makes it difficult and less likely.


\subsection{Limited Trusted Base}
A further source of vulnerability in \HTML documents is that they allow
attackers to take advantage of a greatly expanded trusted base, in the form
of browser plugins like Java and Flash, and in the form of infrequently used
Web API features.  As the frequent rate
of browser updates shows, securing just the browser is an extremely difficult
task.  Needing to trust the browser \emph{in addition to} closed source, third
party plugins with long histories of exploitability makes the problem
of securing the web dramatically more difficult.

\CDF further reduces the attack surface by removing the ability of \CDF
documents to include or refer to plugins. As previously discussed, \CDF does
not include any way to represent an \texttt{<object>}, \texttt{<embed>} or
\texttt{<iframe>} tag on the page, nor does it have a \texttt{<script>} type
that could be used to include the same client side. Earlier in the web's
evolution, popular features like audio and video could only be provided by
these third party plugins. Now that the web has matured and all popular
browsers support standards for audio and video with hardware-accelerated
playback, the absolute necessity of these extensions is limited. The \CDF
specification makes it impossible for \CDF page authors to reference or
interact with any plugins that might be on the system.

\subsection{Client Side Fingerprinting}
Web users who have not authenticated or intentionally identified themselves
to a website expect to be semi-anonymous.  Once a user discards any identifying
tokens they've received from a website (e.g. deleting their
browser cookies) they have a reasonable expectation that they are no longer
known to the site.  Such assumptions are even built into the state-less nature of
HTTP, and required the addition of cookies to add state into the web.

Malicious websites violate this assumption through client side
fingerprinting, or by including \JS code in their pages to
take a large number of quasi-identifiying measurements, and combine them
in such a way that site visitors can be uniquely identified.  These
quasi-identifiers are not sensitive to users deleting their cookies,
modifying their user agent string, or taking other similar steps, making it
difficult for users to regain their privacy.

While not all of these techniques rely on client executed \JS
code, many do, such as canvas based fingerprinting
\cite{mowery2012pixel,acar2014web}, identifying the \JS
engine being used\cite{mowery2011fingerprinting} or font and plugin enumeration
\cite{eckersley2010unique}.  \CDF prevents these client-side fingerprinting
techniques by removing the ability of page authors to
include code that takes the relevant measurements.  For
example, there is no way for a \CDF author to construct a \CDF document that
will query the versions of what plugins are installed on the system, or to
use the \texttt{<canvas>} tag to take semi-uniquely-identifying measures of the
visitors browser.  By removing the ability of document authors to include
arbitrary \JS in their pages, and by making it impossible to create
documents that take the same identifying measures, \CDF prevents client-side
fingerprinting and increases the amount of anonymity users can expect.


\subsection{Predictable Information Flow}
\label{sec:eval-info-flow}
A final threat to the privacy and security of web users is that it is difficult,
if not impossible, for the average user to predict what information they are
sharing when they visit a website, and where that information is being sent.
A user may visit a website on a domain they trust---and not intending to trust
any other domains in doing so---only to later learn that the site (maliciously
or unknowingly) notified a third party that they visited the site.

\CDF addresses this issue in three ways.  First, the most popular
and intrusive tracking systems used today rely, at least in part, on
\JS run on the client.  Inclusion of third party tracking libraries is inexpressible in \CDF, and thus
the user automatically gains a great deal of privacy-preservation.

Second, the \CDF parser sets the \texttt{Content-Security-Policy} of all
documents to \texttt{referrer never} though an included \texttt{<meta>} tag
element, instructing browsers to not send a referrer header when requesting
remote resources, further protecting the privacy of the user.

Finally, it is extremely difficult for web users to be sure where their content
will be sent when they submit a form on a webpage, regardless of whether that
form is to perform a search or login in a sensitive website.  Even inspecting
the source \HTML of the page being viewed is no guarantee, since \JS could
have manipulated where the form values will be sent.  \CDF removes these uncertainties
by only allowing forms and sub-page requests to send to the current domain.

Tracking pixels which load from third party domains with unique per-user IDs in
their URL are still usable in \CDF. While this allows some level of tracking to
persist in \CDF, a third party providing Google Analytics style functionality
would need to synchronize the user IDs with every colluding site, rather than
rely on \JS, cookies, and ``referer'' headers to reconstruct user browsing history.
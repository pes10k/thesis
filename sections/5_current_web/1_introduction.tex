\section{Introduction}
\label{current-web:current-web}

The third step towards this dissertation's over-arching goal of improving web
security and privacy is to consider how the per-standard cost and benefit
measurements described in Chapter~\ref{cost-benefit} can be used to improve the
web as it exists today.  More specifically, to explore how the data and
techniques discussed in the previous chapters can be used to better protect web
users, given how web sites are currently deployed.

This chapter describes one such effort, through the design and development
of a browser-extension that imposes access controls on which \WAPI features
web sites can access.  The tool's goal is to modify the \gls{dom} to prevent
websites from accessing to features that risk user's security and privacy. The
extension attempts to limit web sites to a minimal set of features, judged to
be high benefit and low cost.

This chapter discusses the design of this tool, a usability evaluation
of the tool, and a discovered vulnerability the Firefox and Chrome
implementations of the WebExtension standard, which significantly limits the
security and privacy guarantees our extension can enforce.  This
vulnerability affects many other privacy-oriented WebExtensions, and most
of these tools are still vulnerable at the time this work is being written.

The rest of this Chapter is organized as follows:
Section~\ref{current-web:extension} describes the design of a browser extension
based on the findings discussed in Chapter~\ref{cost-benefit},
along with a usability analysis of the extension.
Section~\ref{current-web:extension-deployment} presents issues and discoveries
that were made after the browser extension was released to the public and
began being used by approximately 1k users.

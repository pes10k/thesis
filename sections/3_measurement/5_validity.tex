\section{Validity}
\label{measurement:validity}

This study measures the features executed over repeated,
automated interactions with a website.  We treat these
automated measurements as representative of the features that would be executed
when a human visits the website.

Thus, our work relies on our automated measurement technique
triggering (at least) the browser functionality a human user's browser will
execute when interacting with the same website. This section explains how we
verified this assumption to be reasonable.

\subsection{Internal Validation}
\label{measurement:validity:internal-validation}

\begin{table}[ht]
  \centering
  \begin{tabular}{ l r }
    \toprule
      Round \# &
      Avg. New Standards \\
    \midrule
      2 & 1.56 \\
      3 & 0.40 \\
      4 & 0.29 \\
      5 & 0.00 \\
    \bottomrule
  \end{tabular}
  \caption{Average number of new standards encountered on each subsequent automated crawl of a domain.}
  \label{fig:internal-validation-table}
\end{table}

As discussed in Section~\ref{measurement:methodology:default-case-measurements}, we applied our automated
measurement technique to each site in the Alexa 10k ten times, five times in
an unmodified browser, and five times with blocking extensions in place.
We measured five times in each condition with the goal of capturing the full
set of functionality used on the site, since the measurement's
random walk technique means that each subsequent measurement may encounter
different, new parts of the site.

A natural question then is whether five measurements are sufficient
to capture all potentially encountered features per site, or whether additional
measurements are necessary.  To ensure that five measurements were
sufficient, we examined how many new standards were encountered on
each round of measurement.  If new standards were still being encountered in the
final round of measurement, it would indicate we had not measured enough,
and that our data painted an incomplete picture of the types of features used
by each site.

\ref{fig:internal-validation-table} shows the results of this verification.
The first column lists each round of measurement, and the second
column lists the number of new standards encountered
that had not yet been observed in the previous rounds (averaged
across the entire Alexa 10k).  As the table shows, the average number of new
standards observed on each site decreased with each measurement,
until the 5th measurement for each site, at which point we did not observe any
new features being executed on any site.

From this we concluded that five rounds was sufficient for
each domain, and that further automated measurements of these sites were
unlikely to observe new feature usage.


\subsection{External Validation}
\label{measurement:validity:external-validation}

\begin{figure}[h]
  \centering
  \includegraphics[width=0.5\textwidth]{figures/measurement_external_validation.pdf}
  \caption{Histogram of the number of standards encountered on a domain under manual interaction that were not encountered under automated interaction.}
  \label{fig:external-validation-figure}
\end{figure}

We also tested whether our automated technique observed the same feature
use as human web users encounter.   We randomly chose 100 sites to visit
from the Alexa 10k and interacted with each for 90 seconds in a casual
web browsing fashion.  This included reading articles and blog posts, scrolling
through websites, browsing site navigation listings, etc.

We interacted with
the home page of the site (the page directed to from the raw domain) for 30
seconds, then clicked on a prominent link we thought a typical human browser
would choose (such as the headline of a featured article) and interacted
with this second page for 30 more seconds.  We then repeated the process a
third time, loading a third page that was interacted with for another 30
seconds.

After omitting  pornographic and non-English sites, we completed this process
for 92 different websites. We then compared the features used during manual
interaction with our automated measurements of the same sites.
\ref{fig:external-validation-figure} provides a histogram of this
comparison, with the x-axis showing the number of new standards
observed during manual interaction that \textit{were not} observed during
the automated interaction.  As the graph shows, in the majority of cases
(83.7\%), no features were observed during manual interaction that the
automated measurements did not catch.

The graph also shows a few outliers,
including a very significant one, where manual interaction triggered
standards that our automated technique did not.  On closer inspection, this
outlier was due to the site updating its content between when we performed the
automated measurement and the manual measurement.  The outlier site,
\texttt{buzzfeed.com}, is a website that changes its front page
content hour to hour.  The site further features subsections that are unlike
the rest of the site, and can have widely differing functionality,
resulting in very different standard usage over time. We checked to see if
standards were used under manual evaluation of the outlier that were not observed
during automated testing on the rest of the Alexa 10k, and did not find any.

From this we conclude that our automated measurement technique did a generally
accurate job of elicit the feature use a human user would
encounter on the web, even if the technique did not perfectly emulate human
feature use in all cases.

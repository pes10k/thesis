\summary
The \gls{www} is possibly the worlds largest open system, allowing information
to be transfer, and individuals to interact, with a speed and ease that would
have been unimaginable only a generation ago.  However, the earliest interactions
for the \gls{www} specified only static documents and form submission systems,
with interactivity limited to applets and plugins.

It wasn't until Netscape introduced \JS to the web in 1996 in Netscape 2.0
that the web as a modern application platform began to develop. Since then,
the web has picked up a dramatically larger, and quickly growing, set of capabilities,
designed to allow web site authors to deliver new and more complex applications
to users.

This feature growth was initially driven by competition between
browser vendors.  Netscape and Microsoft attempted to gain market advantage
by implementing unique functionality, to entice web authors
to target their browsers, and ensure a larger user base for the browser.

More recently, this feature growth is driven primarily by competition between the
web and ``native'' application platforms, or systems provided by operating
systems, that may provide better performance, or additional functionality, then
web sites can provide.


competition between browser vendors, and between the web and other application
platforms, has p
web sites and web browsers have gained a wide range of new capabilities,
both due to competition between browser vendors, and competition with
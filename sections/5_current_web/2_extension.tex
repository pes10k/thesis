\section{Browser Extension}
\label{current-web:extension}

As part of this work, we are also releasing a Firefox browser extension
that allows users to harden their browsers using the same standard
disabling technique described in this paper.  The extension is available
as source code\footnote{\ExtensionSourceUrl}.

\subsection{Implementation}
Our browser extension uses the same \WAS disabling technique described in
Section~\ref{cost-benefit:intercepting-js} to dynamically control the DOM-related
attack surface to expose to websites.  The extension allows users to
deploy the same conservative and aggressive hardened browser configurations
described in Section~\ref{subsec:eval-configs}.  Extension users can also
create their own hardened configurations by selecting any permutation of the
\NumStandards measured \WASs to disable.

Hardened configurations can be adjusted over time, as the
relative security and benefit of different browser features changes.  This
fixed-core-functionality, updated-policies deployment model works well
for popular web-modifying browser extensions (such as AdBlock, PrivacyBadger and
Ghostery).  Our browser-hardening extension similarly allows users to subscribe
to configuration updates from external sources (trusted members of the security
community, a company's IT staff, security-and-privacy advice groups, etc.), or
allows users to create their own configurations.

If a browser standard were found to be vulnerable to new attacks in the future,
security sensitive users could update their hardened configurations to remove it.
Likewise, if other features became more popular or useful to users on the web,
future hardened configurations could be updated to allow those standards.
The extension enables users to define their own cost-benefit balance in the security
of their browser, rather than prescribing a specific configuration.

Finally, the tool allows users to create per-origin attack-surface policies,
so that trusted sites can be granted access to more \JS-accessible
features and standards than unknown or untrusted websites.  Similar to, but
finer grained than, the origin based policies of tools like NoScript, this
approach allows users to better limit websites to the least privilege
needed to carry out the sites' desired functionality.

We discussed our approach with engineers at Mozilla, and we are
investigating how our feature usage measurement and blocking techniques could be
incorporated into Firefox Test Pilot as an experimental feature. This
capability would allow wider deployment of this technique within a genuine
browsing environment, which can also improve the external validity of our
measurements.

\subsection{Tradeoffs and Limitations}
Implementing our approach as a browser extension entails significant tradeoffs.
It has the benefit of being easy for users to install and update, and that
it works on popular browsers already.  The extension approach also protect users
from vulnerabilities that depends on accessing a \JS-exposed
method or data structure (of which there are many, as documented in
Section~\ref{subsubsec:results-costs-cves}), with minimal re-engineering effort,
allowing policies to be updated quickly, as security needs change.  Finally,
the \WAS-blocking, extension approach is also useful for disabling large
portions of high-risk functionality, which could protect users from not-yet-discovered
bugs, in a way that ad-hoc fixing of known bugs could not.

There are several significant downsides to the extension-based approach however.
First is that there are substantial categories of browser exploits that our
extension-based approach cannot guard against.  Our approach does not provide
protection against exploits that rely on browser functionality that is reachable
through means other than \JS-exposed functionality.  The extension would not
provide protection against, for example, exploits in the browser's CSS parser,
TLS code, or image parsers (since the attacker would not require \JS to
access such code-paths).

Additionally, the extension approach does not have access to some
information that could be used to make more sophisticated decisions about
when to allow a website to access a feature.  An alternate approach that
modified the browser could use factors such as the state of the stack at
call time (e.x. distinguishing between first-and-third party calls
to a \WAS), or where a function was defined (e.x. whether a function was defined
in \JS code delivered over TLS from a trusted website).  Because such information
is not exposed to the browser in \JS, our extension is not able to take advantage
of such information.

\subsection{Controlled Evaluation}
TBD

\subsubsection{Selecting Configurations}
\label{current-web:extension:configurations}

To evaluate the utility and usability of our fine grained, standards-focused
approach to browser hardening, we created two hardened browser configurations.

Table \ref{table:browser-configs} lists the standards that we blocked for
the conservative and aggressive hardened browser configurations.
Our \textbf{conservative} configuration focuses on removing features that are infrequently needed by websites to function, and would be fitting for users who desire more
security than is typical of a commodity web browser, and are tolerant of a
slight loss of functionality.  Our \textbf{aggressive} configuration focuses on removing
attack surface from the browser, even when that necessitates breaking more websites.
This configuration would fit highly security sensitive environments, where users
are willing to accept breaking a higher percentage of websites in order to gain further security

We selected these profiles based on the data discussed in Section~\ref{cost-benefit:results}, related
previous work on how often standards are needed by websites~\cite{snyder2016browser},
and prioritizing not affecting the functionality of the most popular sites on the
web.  We further chose to not restrict the \emph{Web Crypto} standard, to
avoid affecting the critical path of security sensitive code.

We note that these are just two possible configurations, and that users
(or trusted curators, IT administrators, or other sources) could
use this method to find the security / usability tradeoff that best fit their needs.


% \subsubsection{Evaluation}
% \label{subsubsec:eval-configs-evaluation}
\begin{table}[t]
  \centering
  % \rowcolors{2}{gray!25}{white}
  \resizebox{\columnwidth}{!}{
    \begin{tabular}{ l | r r }
      \toprule
        Statistic &
        Conservative &
        Aggressive \\
      \midrule
        Standards blocked              & 15      & 45      \\
        Previous CVEs \#               & 89      & 123     \\
        Previous CVEs \%               & 52.0\%  & 71.9\%  \\
        LOC Removed \#                 & 37,848  & 53,518  \\
        LOC Removed \%                 & 50.00\% & 70.76\% \\
        \% Popular sites broken        & 7.14\%  & 15.71\% \\
        \% Less popular sites broken   & 3.87\%  & 11.61\% \\
      \bottomrule
    \end{tabular}
  }
  \caption{Cost and benefit statistics for the evaluated conservative and aggressive browser configurations.}
  \label{table:config-eval}
\end{table}


We evaluated the usability and the security gains these hardened browser
configurations provided.  Table \ref{table:config-eval}
shows the results of this evaluation.  As expected, blocking more standards
resulted in a more secure browser, but at some cost to usability (measured
by the number of broken sites).

Our evolution was carried out similarly to the per-standard measurement
technique described in Section~\ref{subsec:per-standard-benefit}.  First
we created two sets of test sites, \textbf{popular} sites (the 200 most popular
sites in the Alexa 10k that are in English and not pornographic) and
\textbf{less popular sites} (a random sampling of sites from the Alexa 10k that
are rank 201 or lower).  This yielded 175 test sites in the popular category,
and 155 in the less popular category.

Next we had two evaluators visit each of these 330 websites under three browsing
configurations, for 60 seconds each.  Our decision to use 60 seconds per page
is based on prior research~\cite{liu2010understanding} finding that
that users on average spend under a minute per page.

Our evaluators first visited each site in an unmodified
Firefox browser, to determine the author-intended functionality of the website.
Second, they visited in a Firefox browser in the above mentioned conservative
configuration.  And then finally, a third time in the aggressive hardened
configuration.

For the conservative and aggressive tests, the evaluators recorded how the
modified browser configurations affected each page, using the same 1--3 scale
described in Section~\ref{subsec:per-standard-benefit}.  Our evaluators
independently gave each site the same 1--3 ranking 97.6\% of the time for
popular sites, and 98.3\% of the time for less popular sites, giving us
a high degree of confidence in their evaluations.  The ``\% Popular sites
broken'' and ``\% Less popular sites broken'' rows in Table
\ref{table:config-eval} give the results of this measurement.

To further increase our confidence the reported site-break rates, our evaluators
recorded, in text, what broken functionality they encountered.  We were then
able to randomly sample and check these textual
descriptions, and ensure that our evaluators were experiencing similar broken
functionality.  The consistency we observed through this sampling supports
the internal validity of the reported site break rates.

As Table \ref{table:config-eval} shows, the trade off between gained security
and lessened usability is non-linear.  The conservative configuration disables
code paths associated with 52\% of previous CVEs, and removes 50\% of
\gls{eloc}, while affecting the functionally of
only 3.87\%-7.14\% of sites on the internet.  Similarly, the aggressive
configuration disables 71.9\% of code paths associated with previous CVEs and
over 70\% of \gls{eloc}, while affecting the usability
of 11.61\%-15.71\% of the web.


\subsubsection{Usability Comparison}
\label{current-web:extension:usability-comparison}

\begin{table}[t]
    \centering
    % \rowcolors{2}{gray!25}{white}
    \resizebox{.5\textwidth}{!}{
      \begin{tabular}{ l | r r r }
        \toprule
          & \% Popular    & \% Less popular  & Sites tested \\
          & sites broken  & sites broken     & \\
        \midrule
          Conservative Profile & 7.14\%  & 3.87\%  & 330 \\
          Aggressive Profile   & 15.71\% & 11.61\% & 330 \\
          Tor Browser Bundle   & 16.28\% & 7.50\%  & 100 \\
          NoScript             & 40.86\% & 43.87\% & 330 \\
        \bottomrule
      \end{tabular}
    }
    \caption{Comparison of the useability of the feature-access-control imposing
    extension, compared against versus other popular browser security tools.}
    \label{table:usability-comparison}
  \end{table}


We compared the usability of our sample browser configurations against
other popular browser security tools.  We compared our conservative and
aggressive configurations first with Tor Browser and NoScript, each discussed
in Section~\ref{current-web:extension:configurations}.  We find that the conservative
configuration has the highest usability of all four tested tools, and that
the aggressive hardened configuration is roughly comparable to the default
configuration of the Tor Browser.  The results of this comparison are given in
\ref{table:usability-comparison}.

We note that this comparison is not included to imply which method is the most
secure.  The types of security problems addressed by each of these approaches
are largely intended to solve different types of problems, and all three compose
well (i.e., one could use a cost-benefit method to determine which \WASs to enable
\textit{and} harden the build environment and route traffic through the Tor
network \textit{and} apply per-origin rules to script execution).  However, as Tor Browser and NoScript are widely used security tools, comparing against them gives a good baseline for usability for security conscious users.

We tested the usability using the same technique we used for the conservative
and aggressive browser configurations, described in Section
\ref{subsec:eval-configs}; the same two evaluators visited the
same 175 popular and 155 less popular sites, but compared the page in
an unmodified Firefox browser with the default configuration of the NoScript
extension.

The same comparison was carried out for default Firefox against
the default configuration of the Tor Browser bundle\footnote{Smaller sample
sizes were used when evaluating the Tor Browser because of time constraints,
not for fundamental methodological reasons.}.  The evaluators again
reported very similar scores in their evaluation, reaching the same score
99.75\% of the time when evaluating NoScript and 90.35\% when evaluating the
Tor Browser.  We expect this lower agreement score for the Tor Browser is
a result of our evaluators being routed differently through the Tor network, and
receiving different versions of the website based on the location of their
exit nodes.\footnote{We chose to \emph{not} fix the Tor exit node in a fixed
location during this evaluation to accurately recreate the experience of using
the default configuration of the TBB.}

As Table \ref{table:usability-comparison} shows, the usability of our
conservative and aggressive configurations is as good as or better than other
popularly used browser security tools.  This suggests that, while
our \WASs cost-benefit approach has some affect on usability, it is a
cost security-sensitive users would accept.


\subsubsection{Allowing Features For Trusted Applications}
We further evaluated our approach by attempting to use several popular,
complex \JS applications in a browser in the \textbf{aggressive} hardened
configuration.  We then created application-specific configurations to allow
these applications to run, but with access to only the minimal set of
features needed for functionality.

This process of creating specific feature configurations for different
applications is roughly analogous to granting trusted applications additional
capabilities (in the context of a permissions based system), or allowing trusted
domains to run \JS code (in the context of browser security extensions, like
NoScript).

We built these application specific configurations using a tool-assisted,
trial and error process: first, we visited the application with the browser
extension in ``debug'' mode,
which caused the extension to log blocked functionality.  Next,
when we encountered a part of the web application that did not function correctly,
we reviewed the extension's log to see what blocked functionality seemed
related to the error.  We then iteratively enabled the related blocked
standards and revisited the application, to see if enabling the standard
allowed the app to function correctly.  We repeated the above steps
until the app worked as desired.

This process is would be beyond what typical web users would be
capable of, or interested in doing.  Users who were interested in improving the
security of their browser, but not interested in creating hardened app
configurations themselves, could subscribe to trusted, expert curated polices,
similar to how users of AdBlock Plus receive community created rules from
EasyList.  Section~\ref{subsec:dynamic-config} discusses ways that rulesets
could be distributed to users.

For each of the following tests, we started with a browser configured in
the previously mentioned \textbf{aggressive} hardened configuration, 
which disables 42 of the \NumStandards \WASs measured in this work.  We then created
application-specific configurations for three popular, complex web applications,
enabling only the additional standards needed to allow each application
to work correctly (as judged from the user's perspective).

First, we watched videos on YouTube, by first searching for videos on the
site's homepage, clicking on a video to watch, watching the video on its
specific page, and then expanding the video's display to full-screen.
Doing so required enabling three standards that are blocked in our 
\textbf{aggressive} configuration: the \textit{File API} standard~\footnote{YouTube
uses methods defined in this standard to create URL strings referring to media on the
page.}, the \textit{Media Source Extensions} standard~\footnote{YouTube uses the
\texttt{HTMLVideoElement.prototype.getVideoPlaybackQuality} method from this
standard to calibrate video quality based on bandwidth.}, and the
\textit{Fullscreen API} standard. Once we enabled these three standards on the
site, we were able to search for and watch videos on the site, while still
having 39 other standards disabled.

Second, we used the Google Drive application to write and save a text
document, formatting the text using the formatting features provided by
the website (creating bulleted lists, altering justifications, changing
fonts and text sizes, embedding links, etc.).  Doing so required enabling
two standards that are by default blocked in our \textbf{aggressive} configuration:
the \textit{HTML: Web Storage} standard~\footnote{Google Drive uses functionality
from this standard to track user state between pages.} and the \textit{UI Events}
standard~\footnote{Google Drive uses this standard for finer-grained detection
of where the mouse cursor is clicking in the application's interface.}.
Allowing Google Docs to access these two additional standards, but leaving
the other 40 standards disabled, allowed us create rich text documents without
any user-noticeable affect in site functionality.

Third and finally, we used the Google Maps application to map a route between
Chicago and New York.  We did so by first searching for ``Chicago, IL'',
allowing the map to zoom in on the city, clicking the ``Directions'' button,
searching for ``New York, NY'', and then selecting the ``driving directions''
option.  Once we enabled the \textit{HTML: Channel Messaging}
standard~\footnote{Which Google Maps uses to enable communication between different
sub-parts of the application.} we were able to use the site as normal.

\subsection{Real-World Deployment}
TBD
\section{Implementation}
\label{future-web:implementation}

We implemented \CDF in two parts, first as a document specification, and second
as several additions to the browser's trusted base: a \emph{parser} that converts \CDF documents into
trusted \HTML and \JS, a \emph{HTTP proxy} that converts \CDF documents for use
in web browsers, and a set of \emph{trusted \JS libraries} that run in
the browser to implement the interactive aspects of \CDF documents.

The described system was implemented to allow \CDF documents to be run in
web browsers today, with no additions or modifications needed to any recently
released browser.  The same design could be implemented by modifying a browser
to be able parse and understand \CDF documents ``natively'', though at the cost
of a much greater engineering task.

We also adopted cascading style sheets, or CSS, to handle the presentation
of \CDF applications.  We did so to minimize the engineering effort needed
to implemented the \CDF concept, and because of the relative lack of security
issues associated with CSS compared to \JS.

This section gives a high level explanation of how our implementation
of the \CDF design works.  Documentation for creating \CDF documents, including
type specifications, nesting rules, and the interactivity primitives
included in \CDF can be found in an open source implementation and accompanying
documentation\footnote{https://github.com/bitslab/cdf.}.

\subsection{Document Format}
\CDF uses JSON strings to represent documents.  \CDF documents are
trees of typed objects.  Types in \CDF fall into one of four categories.

\begin{itemize}
     \setlength{\itemsep}{-2pt}
  \item \textbf{Elements.} The structure and text of the document.
  \item \textbf{Events.} New input from the network or the user.
  \item \textbf{Behaviors.} Descriptions of what should happen when an Event has
                   triggered.
  \item \textbf{Deltas.} Changes to be applied to the document.
\end{itemize}

Each type defines the configuration it can receive (e.g. the
URL that a \texttt{image} object can refer to), and the types it
accepts as children in the tree.  For example \texttt{text} objects can be
children of \texttt{button} objects (to create labels on buttons), but
\texttt{button} objects cannot be children of \texttt{text} objects.
Since the types in \CDF are all well defined, they can be strictly checked to
ensure they will have predictable effects when rendered in the client.

Some types accept configuration parameters (e.g. the class names to add to the
element when rendered in HTML, or the local URL to post a form's information
to).  These configuration parameters are also strictly typed, and so can be
checked for safety and correctness before being rendered in the client.

Types are designed to emphasize predictable information flow and user privacy.
For example, in \CDF \texttt{form} elements are only allow to send information to
the origin domain, while in HTML applications, \texttt{<form>} elements
can be configured to send information to any domain.


% Since the types in \CDF are all well defined, they can be strictly checked to
% ensure they will have predictable effects when rendered in HTML.  This
% strict checking allows \CDF to prevent XSS and other code injection attacks,
% as the example in Figure~\ref{fig:text-example} demonstrates.

%
% \CDF contains four kinds of types, which compose into arbitrarily complex
% interactive web applications.  \textbf{Element} types describe the structure
% of a document, analogous to ``tags'' HTML documents.  Examples of structure
% types include the \texttt{ul} type, used to describe unordered lists of elements
% in a document, the \texttt{img} type, used to refer to an image resource, or the
% \texttt{text} type, the sole way of presenting text.
% These types are designed to be mostly familiar to HTML authors.  Figure~\ref{fig:cdf-list}
% shows how \CDF's types can be composed to create a simple list, and
% Figure~\ref{fig:html-list} shows how that same list could be represented in \HTML.
%
%
% \textbf{Event} types are used to describe things that should be responded to
% in a \CDF document.  Examples of event types include \texttt{timer} events and
% \texttt{mouse} events.  These have close analogues in HTML and \JS based
% applications.
%
% \textbf{Behavior} types describe things that should happen in response to
% events.  Examples include the \texttt{update} type, indicating a request
% that should be made to the originating server, optionally including information
% from the document (similar to an \texttt{AJAX} request in an HTML application),
% or the \texttt{state} type, describing a state machine that an
% application should be advanced through (state machines are sets of ``delta''
% objects, described below).
%
% Finally, \textbf{delta} types describe changes that should be made to the
% existing \CDF document.  Examples of delta types include \texttt{tree modifications},
% indicating that given element objects should be appended to, or removed from,
% part of the document, or \texttt{attribute modifications}, indicating that
% the properties of an element should be changed (e.x. the image URL a
% ``img'' element refers to should be changed).
%
% The four types in \CDF are designed to compose well.  For example, a \CDF document
% might include a \texttt{button} object (an ``element'' type) that has a \texttt{click}
% object (``event'' type) attached.  This \texttt{click} object might have an \texttt{update} object
% attached (``behavior'' type), indicating that a request should be made to the server,
% when the click occurred.  The server might then respond to this request with a
% \texttt{update subtree} object (``delta'' type), including some additional text
% to add to document.
%
% Since the types in \CDF are all well defined, they can be strictly checked to
% ensure they will have predictable effects when rendered in HTML.  This
% strict checking allows \CDF to prevent XSS and other code injection attacks,
% as the example in Figure~\ref{fig:text-example} demonstrates.
%

\subsection{Trusted Base Additions}
We implemented the \CDF design through three additions to the current trusted
web browser trusted base.  These additions, in tandem, enforce the security and
privacy properties discussed in Section~\ref{future-web:design}.

\subsubsection{Parser}
The first addition \CDF makes to the browser's trusted base is a \CDF parser.
The role of the \CDF parser is to take strings and either identify them as
invalid \CDF documents, or to render an equivalent and safe HTML and \JS string that
can be rendered in the browser. The parser also provides debugging
information as a convenience to \CDF authors.

If the parser is given a valid \CDF document, it converts it into a combination
of \HTML tags, escaped text, \texttt{<script>} tags referencing \JS libraries
that are part of the \CDF trusted base, and \texttt{<script>} tags containing
parameters to be passed to those trusted libraries.  Invalid documents
``fail closed'', and return an error code and no output.

\subsubsection{HTTP Proxy}
\label{future-web:implementation:proxy}
The second addition \CDF makes to the browser's trusted base is an HTTP proxy
that sits between the browser and the internet.  The proxy passes requests
from the browser to the destination server unchanged.  Once the server responds,
the proxy examines the response.  If the response appears to be a \CDF document,
the HTTP proxy extracts the body of the request and provides it to the parser.
If the parser accepts the response as a valid \CDF document, the proxy passes
the parser-generated \HTML and \JS back to the client.  If the parser rejects
the server's response as invalid \CDF, the proxy instead passes back an error
message to the client, informing the user that the server provided an invalid
document.
%
% In addition to \CDF documents, the proxy also needs to allow \CDF files to
% include static resources that are displayed in the resulting \HTML.
% The proxy allows this by further examining the mime-type of the server's
% response.  If that mime-type is of a type the proxy expects to be found inline
% in a \HTML document (such as an image, a video, or an audio file) the proxy
% allows the content through the client.
%
% Finally, the proxy allows \CDF documents to prompt users to download any type
% of file, but not include most types of files for display in the browser itself.
% The proxy achieves this policy as follows: If the server response was not a
% \CDF document, and was not of a type expected to be found inline, the proxy
% checks to the \texttt{Content-Disposition} header of the HTTP response.  If it
% is set as  \texttt{Attachment} (indicating that the browser should not attempt
% render the content, but only prompt the user to download it), the content is
% allowed to pass through to the client.  In all other cases the response is
% dropped.
%
\subsubsection{Client \JS Libraries}
\label{future-web:implementation:client-js}
The third addition to the browser's trusted base is in a small number of
\JS libraries (14) that implement the interactive elements of each
page.  These libraries handle all the client-side logic and functionality
needed for all of the event, behavior and delta types used in the system,
plus some plumbing code needed to route the parameters extracted by the
parser to the correct library implementations.
%
% For efficiency purposes---primarily caching---these libraries are not included
% inline in the document, but are instead included as references to \JS
% hosted on the current website's domain.  However, when the browser issues an
% HTTP request to the website's domain, that request is intercepted by the
% proxy.  The proxy contains copies of each of the \JS client libraries, and
% is able to respond to the browser's request immediately.

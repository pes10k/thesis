\section{Real-World Extension Deployment}
\label{current-web:extension-deployment}
The previous section described the design of the \WAPI blocking extension, and
how that design was based on the cost-benefit measurements from
Chapter~\ref{cost-benefit}.  This section describes discoveries that were
made once the extension was released to the public, and further
developed with the help of other privacy and security focused developers.


\subsection{Vulnerability in WebExtension Implementations}
While working on an issue that was reported against the blocking extension, we
discovered a security-related vulnerability in the WebExtension implementations
in Firefox and Chrome.  This vulnerability allows determined websites to access
browser functionality blocked by our extension, and a comprehensive defense
against it required modifications to our approach that breaks more websites
than we originally accounted for.  This same WebExtension weakness also affects
other security and privacy improving extensions (e.g. PrivacyBadger, canvas
blockers, the ``Shields Up'' defenses in the Brave browser).

This section provides some background information on what the WebExtension
standard is, describes the vulnerability in the most popular
implementations of the WebExtension standard we discovered,
and how authors of privacy and security enhancing extensions can move forward.


\subsubsection{The WebExtension Standard}
The \textit{WebExtension}~\cite{webext2018standard} standard defines a way
to write browser-modifying extensions that run on all major, modern browsers.
All of the major browser vendors have pledged support for the standard, but
Firefox and Chrome have the most complete and popular implementations.
The standard is largely based on the original Chrome Extension API, and is
managed by the \gls{w3c}.

The WebExtension standard allows authors to modify the browsing environment
in many ways.  Most relevant to our extension is the ability to inject
\JS into frames, before the frame's own \JS has executed.  This allows the
extension to modify the environment the website executes in.  Our extension
uses this technique to add access controls to \WAPI features, using a method
described in greater detail in Section~\ref{cost-benefit:intercepting-js}.
Many other privacy and security enhancing extensions use
this same technique to prevent pages from accessing parts of the \WAPI, to
protect users (e.g. to detect or prevent fingerprinting attempts).


\subsubsection{How the Vulnerability Works}
The vulnerability we discovered allows pages to access unmodified versions
of the browser environment, even after the extension's \JS has run.  The
vulnerability works by exploiting how frames interact in Chrome and Firefox,
and the timing of when WebExtension's script injection hooks execute.


\subsubsubsection{Frames in HTML Documents}
The term \textbf{frame} refers to an execution environment in the browser.
Each page loaded by the browser gets a frame, with each browser
tab depicting a single frame.  For example, loading \texttt{example.org} in
a browser tab will create a single frame, showing the \gls{html} document
returned by \texttt{example.org}.  Opening a second tab and loading
\texttt{other-example.org} will likewise create a new frame, this time
depicting the document returned from \texttt{other-example.org}.

Importantly, each frame gets its own \gls{dom} instance, each with its own
version of each \WAPI feature.  Each \WAPI feature is implemented by a \JS
object, with functions represented by objects inheriting from
\texttt{Function.prototype}.  Changing a function in one frame will have no
effect on similar objects in other frames.  Put differently, deleting the
\texttt{Document.prototype.getElementById} object in the \texttt{example.org}
frame will prevent code executing in the \texttt{example.org} frame from
querying for elements by their \texttt{id}, but that change will be invisible
to code running in the \texttt{other-example.org} frame.

In general, frames are not able to directly access the resources of other
frames.  In the above example, code running in the \texttt{example.org} frame
cannot access the \gls{dom} of the \texttt{other-example.org} frame.  However,
this rule does not hold in all cases.  In some cases, a frame can access the
\gls{dom} of its child frames (e.g. \texttt{iframes}).  When the child frame is
rendering content from the same domain as the parent frame, the parent frame
can access the \gls{dom} of the child frame through the child frame's
\texttt{contentWindow} property.


\subsubsubsection{Injecting Script from a WebExtension}
The WebExtension standard provides several opportunities for extensions to
inject \JS into frames.  Relevant to this vulnerability is that scripts can
register to run at \texttt{loading} time, which corresponds to a point when
the \gls{dom} for the frame has been prepared, but no page contents have
yet executed. The WebExtension standard defines this as the
\texttt{document\_start} hook.

Because the WebExtension standard guarantees that \texttt{document\_start} will
run before any other content is executed in the frame, many extensions use
this opportunity to inject script into a frame, to modify the DOM of the frame
to achieve some security or privacy improvement.  Our extension uses this
hook to inject \JS that interposes on the features of blocked standards.
Since the WebExtension standard guarantees that this will happen
before any page content executes, the extension can be sure that the page will
only see the \gls{dom} as its been modified by the extension, and that page code
will not be able to access the original, non-interposed-on versions of the
blocked features.

Many other extensions function similarly, and use the
\texttt{document\_start} hook to achieve security or privacy goals.
PrivacyBadger, for example, uses this opportunity to replace \WAPI functions
associated with canvas fingerprinting with new functions that record the
fingerprinting attempt.


\subsubsubsection{Exploiting the Vulnerability}
The vulnerability arises because of how the above two issues (child-frame
access and extension code injection timing) interact in common WebExtension
implementations.  One might expect that because the \texttt{document\_start}
hook runs before page content, then page content will only be able to access
the \gls{dom} after its been modified.  However, this proves to be incorrect.
While frames are unable to access \emph{their own} \gls{dom} before it is
modified by extensions, frames can access the \gls{dom} of child frames before
the \texttt{document\_start} hook fires in the child frame.

This occurs because, while the \texttt{document\_start} hook is guaranteed to
fire before a frame's content runs, there is a period of time between when the
child frame's \gls{dom} is created, and the child's \texttt{document\_start}
hook fires.  If a parent frame accesses the child frame's
\texttt{contentWindow} property during this interim period, the parent frame
will be able to access the child frame's \gls{dom} before it is modified.  The
parent frame can then extract references to blocked functionality and execute
them in the context of the parent frame.  The end result is that pages can
bypass extensions that attempt to restrict access to \WAPI features.


\subsubsection{Addressing the Vulnerability}
We addressed this vulnerability in the WebExtension standard in several ways.
First, we filed bugs with both
Firefox\footnote{\url{https://bugzilla.mozilla.org/show_bug.cgi?id=1424176}}
and
Chromium\footnote{\url{https://bugs.chromium.org/p/chromium/issues/detail?id=793217}},
notifying them of the issue.  Firefox has acknowledged the issue but so far the
issue has not been addressed.  The issue is still waiting for triage in
Chromium's system.

Second, we notified other similar privacy and security minded
projects that use the same WebExtension approach (i.e. Brave and PrivacyBadger)
of the issue.  In both cases, the developers acknowledged the issue, but are
waiting on action from the browser vendors before pursuing the matter further,
largely because currently available solutions break too many websites.

Third, we modified the extension to give users an option to protect themselves
against this vulnerability, at the cost of breaking some benign sites.  Versions
of ECMAScript (the technical name for the standard that defines \JS) 5.1
and later define a \texttt{Object.defineProperty}~\cite{ecmascript51} function,
which can be used to prevent code from reading from and writing to properties
on certain types of objects.  The extension can use this
function to prevent frames from accessing the content of child frames.
This prevents websites from bypassing the extension, with the downside of
breaking sites that benignly access child frames in this way.  While this
is not a common technique online (measured by domains that use the technique),
some very popular sites use this technique to coordinate across origins (e.g.
Google's authentication flow).

At a fundamental level, the correct solution is likely to freeze the event loop
of the parent frame until the \texttt{document\_start}
hook of the child frame has fired.  This would incur some performance
loss (since the parent frame would momentarily freeze), but the cost would
be small.  It is difficult to think of scenarios when frames need
to create and insert large numbers of child frames into documents, particularly
in performance-sensitive tasks.  However, further exploring and evaluating
solutions to this issue is beyond the scope of this dissertation.


\subsection{Feature-level Granularity}
\label{current-web:extension-deployment:feature-level}
A second insight gained from making the browser extension public, and improving
it with other privacy and security-minded developers, is that, in some cases,
the ``standard'' may be the wrong level of analysis when evaluating the
\WAPI.  In some cases, users are better served by being able to define
finer-grained policies, such as blocking a small number of features in a larger
standard.  These finer-grained policies allow dedicated users to drive down the
number of sites the extension breaks, while still preventing sites from
accessing problematic \WAPI features.

For example, many users of the extension were interested in blocking the
\textit{Canvas} standard, since it is often implicated in fingerprinting attacks.
While the functionality in the standard is rarely needed to deliver the
main content on a site, it is sometimes used when rendering peripheral but
pleasant content.  Some users wanted to be able to protect themselves against
fingerprinting attacks without giving up some of the flashier parts of the sites
they visited.

The solution was to allow users the option to block
just some features in a standard, but leave the rest of the features
unmodified.  In the \textit{Canvas} example, the standard includes 53 features,
only 4 of which are used for fingerprinting\footnote{%
\texttt{HTMLCanvasElement.prototype.toDataURL},
\texttt{HTMLCanvasElement.prototype.toBlob},
\texttt{CanvasRenderingContext2D.prototype.isPointInPath} and
\texttt{CanvasRenderingContext2D.prototype.isPointInStroke}}.  By blocking
methods that allowed for reading from a canvas, but allowing the rest of the
standard to function as normal, users were able to better protect themselves against
privacy and security violations, while still allowing trusted sites to provide
their user-benefiting functionality.

More broadly, by giving the users the option to block at the feature level,
instead of the standard level, users were able to push the extension's break
rate down without reducing their security.


\subsection{Web API Standard Growth}
Finally, maintaining the extension has emphasized how
quickly the \WAPI grows, and the need for security and privacy
researchers to consider how quickly the web changes as an application
platform.  Since the research in Chapter~\ref{cost-benefit} was conducted,
browsers have implemented partially or fully implemented \WASs for interacting
with VR headsets~\cite{webvrstandard}, interacting with USB
devices~\cite{webusbstandard}, and synthesizing and recognizing
speech~\cite{speechtandard}, among many others.

Each of these features, while adding potentially useful new capabilities to the
platform, also expands the browser's attack surface and makes the system more
complex.  More work like that described in Chapter~\ref{cost-benefit} will be
needed to understand if the benefit of each new powerful feature is worth the
cost.

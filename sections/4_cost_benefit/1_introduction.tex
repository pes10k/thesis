\section{Introduction}
\label{cost-benefit:introduction}

A second step in improving web security and privacy is to understand the
trade-offs each standard in the \WAPI carries with it.
Each feature added to the web platform brings some benefit,
by allowing websites to create new types of applications that users
may enjoy.  Each new feature also carries some cost, in the form of additional
security risk.  This risk can take a variety of forms, such as bugs in the
features implementation, or the feature being exploited to enabling new forms
of tracking or privacy loss.

This implies that web users are not best served by browsers with the maximum
set of features, but by browsers that only include features where the benefit
of doing so outweighs the risks.

This chapter presents a measurement of the costs and benefits each standard
brings to web users.  This chapter is focused only on a global measurement,
or the costs and benefits of enabling the standard in the browser for all
websites.  The more complicated question of how to deal with the fact of that
costs and benefits may differ by site (i.e. the risk of allowing
\texttt{good-folks.org} to access the \textit{Canvas} standard may differ
from the risk of allowing \texttt{evil-jerks.net} to access the standard)
is considered in Chapter~\ref{current-web}.

This Chapter builds on the feature-use work presented in
Chapter~\ref{measurement} by using \WAPI use measurements as one input to a
larger framework for assessing per-standard cost and benefit. After all, how
frequently a standard is used on the web is an important signal to understand
if a standard is beneficial to users (a \WAPI standard that is never used is
trivially not beneficial to the user), but its only one part of a more
complicated story.  A standard, for example, could be used by every site on the
internet, but only to carry out functionality the user does not desire. Or,
contra, the standard could be used by only a small number of websites, but
necessary to carry out functionality that drew the users to the website in the
first place.  Or, third, a standard could be \emph{both} very beneficial to
users, but expose the web-user to attacks and vulnerabilities so severe that
she would still be better off without it.  In short, understanding how often a
\WAPI standard is used is just one piece of information needed to asses its
impact on the browser.

On the other hand, improving browser privacy and security cannot be the
\textit{only} goal to consider when trying to improve web browsers.  If one
were to set out with so narrow a goal, then she would end up stripping out all
functionality from the browser, since, trivially, a browser with no
functionality cannot be attacked!  This is not likely to be seen by anyone but
the most Luddite-minded researcher as an improvement!

The goal then is to identify a subset of the \WAPI where the benefit to the
user outweighs the associated security and privacy risks.  This Chapter builds
towards that goal by presenting a systematic cost-benefit analysis of each
standard in the \WAPI.  By drawing on the measurements described in
Chapter~\ref{measurement}, this chapter models a standard's benefit as the
number of websites in Alexa 10k that require that standard to function
correctly.  Each standard's costs is modeled in three ways: first, as a
function of the number of previously reported \gls{cve} reports related to the
implementation of the standard, second, as the number of academic papers which
leverage the standard to carry out an attack, and third, as a function of
complexity added to the code base by implementing the standard.

The rest of this Chapter is organized as follows:
Section~\ref{cost-benefit:intercepting-js} presents a technique for removing
\WAPI features from the browser, with a minimal effect on existing code.
Section~\ref{cost-benefit:methodology} describes the full methodology used for
measuring the costs and benefits of enabling a \WAPI standard in the browser.
Section~\ref{cost-benefit:results} presents the results of applying the
cost-benefit-measurement methodology to a representative modern web browser.
Section~\ref{cost-benefit:conclusions} discusses this work's place in the
context of the larger goals of this dissertation.

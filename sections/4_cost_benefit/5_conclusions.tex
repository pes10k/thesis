\section{Conclusions}
\label{cost-benefit:conclusions}

This Chapter presented a systematic evaluation of the costs and benefits that
each standard in the \WAPI brings to browser users in low-trust,
non-authenticated situations.  This work measured the benefit of each
standard as a function of the number of sites in the \ATK that require
the standard to carry out the site's main purpose (from the perspective
of the browser user).  The work measured the cost
of each standard in three ways: first as the number of peer-reviewed papers
in top security conferences and journals that level the standard in an attack,
second as the amount of complexity that the standard's implementation adds
to the code base, and third, as the number of vulnerabilities the standard's
implementation is responsible for.

The most significance of this specific work to this dissertation's
overarching goal of improving privacy and security on the web, is two
related insights.  First, this work suggests a significant
subset of functionality in the \WAPI where the cost to users (in terms of
security risk) is much higher than the benefit (in terms of sites that need the
functionality to do what users care about).  This, in turn, suggests that web
privacy and security could be improved by restricting which sites can access
these low-benefit, high-risk features.

Second, this work documents that a small number of standards in the \WAPI
provide the majority of the benefit to browser users.  These same standards,
with a few exceptions, carry very little security risk. This suggests
that an application system based around just these safe, core-web features
would provide users with most of the benefits of modern web applications, with
significant security and privacy improvements.

The next two Chapters follow these insights to build tools and systems that
improve the privacy and security of web users.  Chapter~\ref{current-web}
pursues the first insight by exploring methods to restrict which parts of the
\WAPI existing web applications can access.  Chapter~\ref{future-web}
builds on the second insight by exploring new ways of developing and deploying
web applications that restrict websites to core web functionality, and
emphasize privacy and security over developer flexibility and application
functionality.

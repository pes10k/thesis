\section{Client Side Browser Defenses}
\label{background:related-browser-defs}

There are many techniques to ``harden'' the browser by limiting what \JS pages
are allowed to execute. These defenses can be split into two categories: those
configured by the user, and those configured by the website author.

In the user-configured category, Adblock~\cite{adblockplus} and
NoScript~\cite{noscriptwebsite} are popular browser
extensions that control what \JS is executed in the browser, based on the
\gls{url} the \JS came from.  Adblock's primary
function is to block ads for aesthetic purposes, but it can also prevent
infection by malware being served in those
ads~\cite{forbes-malware,engadget-malware}.  Adblock blocks feature use by
preventing the loading of resources from certain domains.

NoScript decides whether \JS can execute, based on the \gls{url} the code came
from.  By default, NoScript prevents \JS execution from all origins, rendering
a large swath of the web unusable.  In its default configuration, NoScript
allows \JS execution from a built-in set of trusted origins.  This built-in
allow-list has resulted in a proof of concept exploit via purchasing expired,
allowed domains~\cite{noscript_whitelist}.  Beyond these popular tools,
IceShield~\cite{heiderich2011iceshield} dynamically detects suspicious \JS
calls within the browser, and modifies the \gls{dom} to prevent attacks.

The Tor Browser~\cite{dingledine2004tor} disables, or prompts the user before
using a number of, features by default.  Tor Browser disable many \JS features,
most significantly \textit{SharedWorkers}~\cite{webworkersw3c}, and prompts
users before allowing pages to use the Canvas, GamePad API, WebGL, Battery API,
and Sensor standards~\cite{tor-features}.  These particular features are
disabled because they enable techniques which violate the Tor Browser's
security and privacy goals.

On the site-author side, \gls{csp} allows server operators to limiting what
kinds of \JS functionality, and sources of code, can be executed, through
either \gls{http} headers, or \gls{html} \textit{meta} tags.
\gls{csp} allows web developers to constrain code on their
sites so that potential attack code cannot access functionality deemed
unnecessary or dangerous~\cite{stamm2010reining}.  Conscript is another
client-side implementation which allows a hosting page to specify policies for
any third-party scripts it includes~\cite{meyerovich2010conscript}.  There are
also a number of technologies selected by the website author but enforced on
the client side,  including Google Caja~\cite{google13caja} and
GATEKEEPER~\cite{guarnieri09gatekeeper}.

There are also models for enforcing policies to limit functionality outside
of the web browser.  Mobile applications use a richer permission model,
where permission to use certain features is asked of the user at either install
or run-time~\cite{android-permissions,au2011short}.

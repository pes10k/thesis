\section{Conclusions}
\label{measurement:conclusions}

This Chapter presents an automated technique for measuring what parts of the
\WAPI is used in anonymous, typical web browsing experiences, both in
a default browser, and when a web browser is modified with common, popular
ad and tracking blocking extensions.  This Chapter also presents the results
of applying that technique to the Alexa 10k.

The most significant results of this study, as it relates to this
dissertation's overarching goal of improving privacy and security on the web,
are two related insights.  First, this study documents there are significant
portions of the \WAPI that are not used when browsing non-trusted websites
(over 50\%, as measured by \WAPI features).  This strongly suggests that there
is little benefit for browser users in enabling these features, at least until
users have logged in or otherwise established some trust.

Second, this study also documents that there are other parts of the \WAPI
that are frequently used by websites, but which go unused when advertising
and tracking is blocked.  This strongly suggests that some \WAPI functionality
implemented in web browsers is primary used for purposes that the users of those
web browsers do not approve of.  These frequently-blocked features also seem
to provide little benefit to browser users, at least in the case of non-trust,
unauthenticated browsing scenarios.

Chapter~\ref{cost-benefit} builds on these findings by quantifying the costs
and benefits all of the standards in the \WAPI bring users.  The following
Chapters then use these findings to build tools and systems that can
better protect the privacy and security of web users.

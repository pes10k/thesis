\chapter{Introduction}
\label{intro}

The \gls{www} is possibly the world's largest open system, allowing information
to be transferred, and individuals to interact, with a speed and ease that would
have been unimaginable only a generation ago.  This growth in popularity
has occurred alongside an explosion in the type of functionality provided to
websites, as both cause and effect.  Where websites were initially limited
to static documents of images and hyper text, websites now rival traditional
applications in terms of size and capability.  Web applications are frequently
megabytes in size, and can access graphics cards, web cameras, microphones,
perform low-level audio synthesis, and carry out parallel computation, just to
name a few examples.

Each of these features brings some plausible benefit to users, and from a
narrow point of view, web users benefit as the web gains more functionality.
More functionality means web site authors can create richer, more capable
applications. However, this point of view ignores half of the ledger.

In practice, feature growth brings both costs and benefits; it benefits users
by enabling new types of web applications that users may enjoy, but
harms users by adding risk to the platform.  Increased complexity
can harm users by expanding the user's trusted computing base (making bugs and
vulnerabilities more likely), broadening the attack surface of the platform,
and making information flows more difficult for users to understand.

Instead of providing websites with a maximal set of features, web users would
be best served by restricting what functionality websites can access,
and only allow websites to access features where the benefits outweigh the
costs. Put more casually, web browsers should only allow websites access to
functionality when the cargo is worth the freight.

This work presents an attempt to measure the costs and benefits of feature
growth in the browser, both to understand how to improve the web as it exists
today, and to explore alternate ways of deploying web applications with
an emphasis on security and privacy.  This approach of improving browser
privacy and security by restricting websites to a reduced feature set
makes it very different from most other research in the area, which focuses
primarily on either changing the implementations of a small number of features
in the browser or developing new methods for identifying implementation errors.


\section{Common Terms Used in Work}
\label{intro:terms}

Several terms will be used throughout this work.  They are presented upfront
to ease the reading of the rest of the work.

A browser \textbf{feature} is a piece of functionality, implemented as
\JS function, method or property, implemented in a web browser, by a browser
vendor.  Browser features are intended to be used by websites, through
\JS delivered by the website, to carry out interactivity and functionality
on the website.  Examples of browser features include
\texttt{HTMLCanvasElement.prototype.toBlob}, used to read the contents of
a \texttt{<canvas>} element in a HTML document into a string, and
\texttt{Document.prototype.getElementById}, used to query an element in a
HTML document.

A \textbf{standard} is a set of related features, created by a standards
organization, describing what features browser vendors should implement, and
how those features should function.  Standards generally define many features,
intended to be used together to accomplish related goals.  For example, the
\textit{SVG}~\cite{svg2011standard} standard defines features used for creating
and modifying SVG elements, and the \textit{WebGL}~\cite{webgl2015standard} and
\textit{WebGL 2.0}~\cite{webgl22018standard} standards define features
for performing 3d graphics operations in web pages.

The \textbf{\WAPI} is the union of all the features in all of the standards
implemented in modern web browsers. While the implemented \WAPI differs slightly
between modern browsers (for reasons such as differing organizational priorities
or different engineering capabilities), those differences tend to be temporary
and minor.

\section{Contributions}
\label{intro:contributions}

The underlying motivation, and core thesis, of this work, is that the privacy
and security of web users would be improved, with only small effects on usability,
by restricting the functionality web sites have access to.  This thesis is
explored from several angles in this work.  Specifically, this dissertation
makes the following contributions:

\begin{itemize}
    \item A web-scale measurement of how the \WAPI is used on the web today.
          This study was an automated measurement of browser functionality
          used on sites in the Alexa 10k (at time of measurement), with
          a novel method of distinguishing between \WAPI features used
          for core-site-functionality, and \WAPI features used for
          non-user-serving purposes (e.g. advertising and tracking).
    \item A comprehensive measurement of the costs and benefits of the entire
          \WAPI, broken down by standard.  This work models a standard's
          benefit as the number of sites on the web that require the standard
          to carry out the site's core functionality.  The work models
          a standard's cost in three ways: as the number of publicity
          disclosed vulnerabilities relating to the standard, the number
          of lines of code uniquely needed to implement the standard, and
          the number of papers describing attacks that leverage the standard
          in top security conferences and journals.
    \item Findings learned and vulnerabilities discovered when deploying
          tools to restrict \WAPI access on the web.  These include the
          development of a browser extension that allows for imposing
          access controls on \WAPI features in current browsers, useability
          measurements taken during an in-lab evaluation of the tool, and
          broader findings from real-world use of the extension.
    \item The design and evaluation of CDF, a system of developing and deploying
          web applications with functionality similar to most modern websites,
          but providing client-enforced protections and dataflow
          guarantees.
\end{itemize}

\section{Organization}
\label{intro:organization}

The remainder of this dissertation is organized in the following manner.

Chapter \ref{background} provides background material on attacks and defenses
related to browser security.

Chapter \ref{measurement} describes both an automated method for measuring
browser feature use on the web, and the results of applying that technique
to the Alexa 10 (as it stood at the time of measurement).  This chapter
also includes the description of a method for distinguishing user-serving
feature use from non-user-serving (e.g. advertising and tracking related)
feature use.

Chapter \ref{cost-benefit} presents a method for measuring the costs and
benefits of the \WAPI standards implemented in modern web browsers, and
the results of applying that methodology to a modern, representative web
browser.

Chapter \ref{current-web} describes efforts to apply the findings from Chapters
\ref{measurement} and \ref{cost-benefit} to the web as it exists today, in the
form of a publicly-released browser extension that is being used by
approximately 1,000 real-world users.  This chapter also includes a usability
measurement of this browser extension-approach, comparisons with other popular
web security and privacy tools, and the description of a security vulnerability
discovered in popular security and privacy tools as a result of this work.

Chapter \ref{future-web} presents the design of an alternative system for developing
and deploying web applications that, building on the findings presented in
Chapters \ref{measurement} and \ref{cost-benefit}, is intended to provide
site authors with the expressiveness needed to design interactive modern web
sites, but while providing client-enforced security and privacy guarantees to
web users.

Chapter \ref{conclusion} concludes with some discussion of these findings, and
how they could be pursued further.


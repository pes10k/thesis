\usepackage{booktabs}
% \usepackage[dvipsnames]{xcolor}
% \usepackage{xcolor}
% \usepackage{color, colortbl}
% \definecolor{linkcol}{rgb}{0,0,1}
% \definecolor{citecol}{rgb}{0,0,1}
% \definecolor{urlcol}{rgb}{0,0,1}
\usepackage{listings}
\usepackage{newlfont}
\usepackage{amsfonts}
\usepackage{amssymb}
\usepackage{euler}
\usepackage[acronym]{glossaries}
\usepackage{xspace}
\usepackage{float}
\usepackage{flushend}
\usepackage{footnote}
\usepackage{enumitem}
\usepackage{url}

\usepackage{caption}
\usepackage{graphicx}

\newacronym{http}{HTTP}{Hypertext Transfer Protocol}
\newacronym{api}{API}{Application Programming Interface}
\newacronym{www}{WWW}{World Wide Web}
\newacronym{w3c}{W3C}{World Wide Web Consortium}
\newacronym{whatwg}{WHATWG}{Web Hypertext Application Technology Working Group}
\newacronym{cdf}{CDF}{Contained Document Format}
\newacronym{eloc}{ELoC}{Effective Lines of Code}
\newacronym{html}{HTML}{Hyper Text Markup Language}
\newacronym{dom}{DOM}{Document Object Model}
\newacronym{wasm}{WASM}{WebAssembly}
\newacronym{csp}{CSP}{Content Security Policy}
\newacronym{webidl}{WebIDL}{Web Interface Definition Language}
\newacronym{io}{IO}{Input-Output}
\newacronym{xss}{XSS}{Cross-Site Scripting}
\newacronym{ecma}{ECMA}{European Computer Manufacturers Association}
\newacronym{svg}{SVG}{Scalable Vector Graphics}



%
% This is the user manual for UICTHESI CLS.
% Document date 1/10/92 (phd)
% Updated to include information on equation numbering 6/25/92 (phd)
%
% Updated to compile under LaTeX version 2e 2/21/96
%
% The material below is needed only for this document, not normally in
% a thesis prepared with this style file.
%
\def\new@fontshape#1#2#3#4#5{\expandafter
     \edef\csname#1/#2/#3\endcsname{\expandafter\noexpand
                                 \csname #4\endcsname}}
\new@fontshape{cmr}{bx}{sc}{
      <5>cmcsc8 at 5pt%
      <6>cmcsc8 at 6pt%
      <7>cmcsc8 at 7pt%
      <8>cmcsc8%
      <9>cmcsc9%
      <10>cmcsc10%
      <11>cmcsc10 at 10.95pt%
      <12>cmcsc10 at 12pt%
      <14>cmcsc10 at 14.4pt%
      <17>cmcsc10 at 17.28%
      <20>cmcsc10 at 20.736pt%
      <25>cmcsc10 at 24.8832pt%
      }{}
\mathversion{normal}
\newcommand{\ams}{{$\cal{A}\cal{M}\cal{S}$}}
\newcommand{\amslatex}{{$\cal{A}\cal{M}\cal{S}$-\LaTeX{}}}
\newcommand{\amstex}{{$\cal{A}\cal{M}\cal{S}$-\TeX{}}}
\newcommand{\BibTeX}{{\rm B\kern-.05em{\sc i\kern-.025em b}\kern-.08em
    T\kern-.1667em\lower.7ex\hbox{E}\kern-.125emX}}
\newcommand{\uicthesi}{{$\mathbb{UICTHESI}$}}

\newcommand\bs{\char '134 }   % A backslash character for \tt font
\newcommand{\lb}{\char '173 } % A left brace character for \tt font
\newcommand{\rb}{\char '175 } % A right brace character for \tt font


% one or two other commands
\def\newfont#1#2{\@ifdefinable #1{\font #1=#2\relax}}
\def\symbol#1{\char #1\relax}
\setacronymstyle{long-short}
\makeglossaries
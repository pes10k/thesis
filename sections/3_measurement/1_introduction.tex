\section{Introduction}
\label{measurement:overview}

The first step in understanding the security and privacy implications
of the current (and quickly expanding) \WAPI is to understand what features
are being used on the web, and for what purposes.  If every site on the
web uses a given \WAPI feature, thats a suggestive (though not determinative)
signal that a feature may be useful to web users.  Conversely, if a user
never visits a website that uses a given \WAPI feature, then that feature
is (trivially) providing no direct benefit to the user.

The design and development of the web makes these kinds of conceptually simple
measurements complex to carry in practice.  First, with the exception of
trivial cases, one cannot simply ``download a website'' and count invocations
of functions calls, the way one might be able to approximate with traditional
applications.  Web sites are downloaded (and sometimes, generated) dynamically,
one portion at a time, depending on user input. The server side of an
application can even change while the client is interacting with the
application!  All of which means, there is no way to meaningfully know if
you've downloaded the entire web application, making static analysis of a
website difficult, if not impossible.

Similarly, the highly-dynamic nature of \JS (how the client-side of web
applications are implemented) means that the language is difficult to
statically analyze; even if one could download all of the \JS code that
comprises a web application, its a non-trivial task to determine what \WAPI
functions are called in a piece of code, let alone determining which code paths
the user would execute during a likely interaction with the website.

To confound matters more, all feature invocations on a website are not equally
desirable to the user.  A user may benefit if a feature is being used to render
a news story she wishes to read, while the user might experience harm if the
feature is being used to fingerprinting her browser for tracking purposes.  A
useful measurement of \WAPI use on the web should distinguish between these
kinds of cases.

Finally, the scale of the internet means that manual interaction is not
feasible to measure a representatively large portion of the web.  An automated
technique is needed.

For these reasons, determining what \WAPI features users are likely
to encounter on the web, and distinguishing harmful from beneficial feature
invocations, is a difficult problem.

This work presents a solution to this problem, in the form of an automated
measurement technique that interacts with websites in a manner that
approximates how humans would interact with the site.  By counting which
features are invoked during each automated interaction, we're able to estimate
which features real users would encounter when interacting with a website. We
then repeated this automated measurement technique with popular advertising and
tracking-blocking extensions installed.  By comparing the difference in
features that are executed under these configurations allows us to to
distinguish user-serving \WAPI use from non-user-serving \WAPI use.

Applying this automated measurement technique to the Alexa 10k gives an
answer to the original question, namely, which \WAPI features do web users use when
browsing the web.  We find, for example, that 50\% of the \WAPI features in the
browser are never used by the ten thousand most popular websites, when users
are browsing in low trust, non-authenticated scenarios.

We were also able to identify features that are used
predominantly for non-user-serving purposes; 10\% of \WAPI features in the
browser are blocked 90\% of the time when ad and tracking-blocking extensions
are installed,  and over 83\% of features in are executed on less than 1\% of
websites in the presence of these popular blocking extensions.

The data described in this work has been publicity shared and is freely
available.  The dataset contains our measurements of what \JS features are used
in the Alexa 10k, both by default browsers, and when ad and tracking blocking
extensions are installed.  The database with these measurements, along with
documentation describing the database's schema, is available
online~\cite{snyderp2016webapidata}.

The rest of this Chapter is organized as follows.
Section~\ref{measurement:data-sources} describes the data sources used to
conduct this work. Section~\ref{measurement:methodology} describes the
methodology used in this Chapter in greater detail, and
Section~\ref{measurement:results} presents the results of this automated
measurement, including how frequently features are used, and which features are
blocked by popular blocking extensions.  Section~\ref{measurement:validity}
describes steps taken verify that our results are correct, and
Section~\ref{measurement:conclusions} concludes with some discussion of the
significance of these results.

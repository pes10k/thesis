\section{Ads and Tracking Blocking}
\label{background:ads-tracking-blocking}

Researchers have previously investigated how people use ad blockers.
Pujol et al. measured AdBlock usage in the wild, discovering that while a
significant fraction of web users use AdBlock, most users
primarily use its ad blocking, and not its privacy-preserving
features~\cite{pujolannoyed}.

User tracking is an insidious part of the modern web. Recent work by Radler
found that users were less aware of cross-website tracking than they were about
of data collection by first party sites, like Facebook and Google. Radler also
found that users who were aware of it had greater concerns about unwanted
access to private information than those who weren't
aware~\cite{rader2014awareness}.  Tracking users' web browsing activity across
websites is largely unregulated, and a complex network of mechanisms and
businesses have sprung up to provide services in this
space~\cite{falahrastegar2014anatomy}.  Krishnamurthy and Willis found that
aggregation of user-related data is both growing and becoming more
concentrated, i.e. being conducted by a smaller number of
companies~\cite{krishnamurthy2009privacy}.

Tracking was traditionally done via client-side cookies, giving users a measure
of control over how much they are tracked (i.e. they can always delete
cookies).  However, a wide variety of non-cookie tracking measures have been
developed that take this control away from users. A variety of tracking
blockers prevent these non-cookie tracking mechanisms, including browser
fingerprinting~\cite{eckersley2010unique}, \JS
fingerprinting~\cite{mowery2011fingerprinting,mulazzani2013fast}, Canvas
fingerprinting~\cite{mowery2012pixel}, clock skew
fingerprinting~\cite{kohno2005remote}, history
sniffing~\cite{jang2010empirical}, cross origin timing
attacks~\cite{van2015clock}, evercookies~\cite{evercookies}, and Flash cookie
respawning~\cite{soltani2010flash,ayenson2011flash}.  A variety of these
tracking behaviors have been observed in widespread use in the
wild~\cite{soltani2010flash,ayenson2011flash,acar2014web,nikiforakis2013cookieless,mcdonald2011survey,olejnik2014selling,sorensen2013zombie}.

Especially relevant to this work is the use of \JS \gls{api}s for tracking.
While some \gls{api}~s, such as the \textit{Beacon standard}~\cite{beaconapi},
are designed specifically for tracking, other \gls{api}s were designed to support
benign functionality, but has been co-opted into tracking
purposes~\cite{mowery2012pixel,fingerprintjs2}.  Balebako et al. evaluated
tools which purport to prevent tracking and found that blocking add-ons were
effective~\cite{balebako2012measuring}.

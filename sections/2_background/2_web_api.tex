\section{Web API Standardization and Growth}
\label{background:webapi-growth}

The current \WAPI is the result of many years of growth and standardization.
The standardization process aims to
ensure that each browser's implementation  of the \WAPI is compatible, and that
web developers only need to support a single code base to have their applications
work in all modern browsers.

The current standardization process grew out of frustrations and incompatibilities
in earlier web browsers.  The two early major browsers, Netscape's ``Netscape
Navigator'' and Microsoft's ``Internet Explorer'', initially provided web sites
with very different systems for building interactive websites.  These
models differed in ways that were both trivial
(e.g. different names for properties and methods that provided identical
functionality) and fundamental (e.g. inverse event delegation models).

To keep the browsers' \gls{api}~s from drifting further apart, and
to make the web a more appealing platform for developers,
the tasks of standardizing and growing the \WAPI was moved from the browser
vendors to standards organizations.  The two most significant standards
organizations for the web are the \gls{w3c}, which oversaw the original web
standards, and is the main body overseeing the development of new \WAPI
standards, and the \gls{whatwg}, which was formed as a response to what was seen
as slow progress in the \gls{w3c}.  Other groups, such as the Kronos Group and
\gls{ecma}, also manage relevant standards (the
\textit{WebGL}~\cite{webgl2015standard} and \textit{\JS}~\cite{ecmascript}
standards, respectively).
